%versi 2 (8-10-2016)
\chapter{Landasan Teori}
\label{chap:teori}


 
Pada bab ini akan dibahas mengenai dasar teori yang digunakan pada penyusunan tugas akhir. Pembahasan pertama mencakup hal-hal yang berkaitan dengan pengertian kewirausahaan dari umum sampai khusus yaitu kewirausahaan menurut GEM. Pembahasan kedua yaitu tentang teori dan aplikasi dari CA (Cellular Automata) khususnya tentang ECA (Entrepreneur Cellular Automata). Pembahasan terakhir tentang hal-hal lain yang mendukung implementasi perangkat lunak seperti bahasa pemrograman java.


\section{Arti Kewirausahaan}
\label{sec:artiwirausaha}

\graphicspath{{images/}}

Wirausaha berasal dari kata wira dan usaha. Wira artinya unggul, mulia, luhur sedangkan usaha berarti kemampuan melakukan usaha atas kekuatan diri sendiri. Jadi wirausaha adalah manusia yang unggul yang memiliki kemampuan membangun usaha sendiri. Kewirausahaan sendiri merupakan kepribadian wirausaha. Wirausaha merupakan orang atau manusia yang memperjuangkan kemajuan terutama pada bidang ekonomi demi masyarakat seperti menciptakan lapangan pekerjaan, membantu memenuhi kebutuhan masyarakat yang semakin meningkat dan berusaha mengurangi ketergantungan dari luar negeri. Istilah kewirausahaan pada umumnya merupakan suatu ilmu yang mempelajari tentang kemampuan seseorang dalam menghadapi tantangan hidup untuk memperoleh peluang dan menghadapi segala risiko yang ada dengan mengandalkan kekuatan diri sendiri tanpa bergantung pada orang lain. \cite{artiwirausaha} 


%Kewirausahaan menurut GEM merupakan proses yang terdiri dari fase-fase berbeda mulai dari niat mendirikan suatu usaha, menjalankan suatu usaha baru atau sudah berdiri dan sampai usaha tersebut berhenti. Proses ini dimulai dengan keterlibatan individu yang berpotensi untuk menjadi wirausaha, yaitu mereka yang percaya bahwa mereka mempunyai kemampuan untuk memulai suatu usaha, individu yang melihat kesempatan untuk berwirausaha dan individu yang tidak takut gagal dalam memulai suatu usaha. \cite{wirausahaGEM}

Kewirausahaan menurut GEM merupakan sebuah proses yang memiliki tahapan-tahapan yang berbeda (Gambar \ref{fig:fasewirausaha}). Tahapan-tahapannya antara lain adalah dimulai dari niat mendirikan usaha, menjalankan usaha dan yang terakhir adalah berhentinya usaha yang dibuat. Tahapan pertama yaitu wirausaha \textit{nascent}. Wirausaha \textit{nascent} ini merupakan tahapan dimana seseorang memulai usahanya yang waktunya kurang dari tiga bulan. Tahapan kedua yaitu wirausaha yang sedang menjalankan usahanya dan sudah bisa menggaji orang lain, waktunya lebih dari tiga bulan tetapi kurang dari tiga tahun. Wirausaha \textit{nascent} dan wirausaha yang sedang menjalankan usahanya masuk ke dalam TEA (Total Early-Stage Entrepreneurial Activity). TEA merupakan persentase populasi antara usia 18 sampai 64 tahun yang berada pada tahap memulai usaha maupun pemilik bisnis yang waktunya kurang dari 42 bulan \cite{wirausahaGEM}. Tahapan terakhir adalah wirausaha mapan (\textit{established entrepreneur}) yaitu seseorang yang sudah menjalankan usahanya lebih dari tiga tahun dan tentunya sudah bisa menggaji orang.\cite{GEM2013}

GEM melakukan penelitiannya berdasarkan pada beberapa premis.Pertama, keadaan ekonomi suatu negara. Jika keadaan ekonomi suatu negara sedang sulit itu artinya dengan adanya wirausaha dapat membantu memperluas lapangan pekerjaan (memotivasi orang untuk menjadi seorang wirausaha juga lebih meningkat), sedangkan jika keadaan ekonomi suatu negara sudah baik keberadaan wirausaha tidak terlalu dibutuhkan (memotivasi orang untuk menjadi seorang wirausaha sudah kurang menarik). Kedua, kemampuan dan motivasi individu untuk memulai sebuah usaha dan pandangan masyarakat tentang wirausaha. Ketiga, pertumbuhan tinggi kewirausahaan dan persaingan antar negara tentang seberapa inovatif usaha tersebut. \cite{GEM2013}
%GEM melakukan penelitiannya dengan didasarkan pada premis-premis berikut. Pertama, kemakmuran sebuah ekonomi sangat tergantung pada sektor kewirausahaan yang dinamis. Sifat dari kegiatan ini dapat beragam karakter dan dampaknya. Kewirausahaan yang didorong oleh kebutuhan khususnya di wilayah yang kurang berkembang atau wilayah yang sedang mengalami penurunan lapangan kerja, dapat membantu ekonomi suatu negara jika memang lapangan pekerjaan terbatas. Di sisi lain, pada wilayah yang lebih berkembang kesempatan wirausaha terjadi lebih akibat dari kemakmuran dan kemampuan inovasi mereka. Kedua, kapasitas kewirausahaan sebuah ekonomi didasarkan pada kemampuan dan motivasi individunya untuk memulai suatu usaha dan dapat diperkuat oleh persepsi positif masyarakat tentang kewirausahaan. Terakhir, pertumbuhan tinggi kewirausahaan adalah kontributor utama untuk penyedia lapangan pekerjaan dan persaingan antar negara bergantung pada usaha yang inovatif.
\begin{figure} [H]
	\centering  
	\includegraphics[width=14cm, height=6cm]{GEM2016-wirausaha}  
	\caption[Fase Wirausaha]{Fase Wirausaha} 
	\label{fig:fasewirausaha} 
\end{figure}


GEM mempertimbangkan beberapa atribut atau indikator yang mempengaruhi berlangsungnya kegiatan berwirausaha. Atribut-atributnya yaitu Perceived Opportunities, Perceived Capabilities, Entreprenurial Intention dan Fear of Failure Rate \cite{wirausahaGEM}. Penjelasan beberapa indikator akan dijelaskan pada tabel \ref{tabelindikator}

\begin{table}[H]
\centering
\caption{Tabel Indikator GEM}
\begin{tabular}{|c|p{8cm}|}
\hline
Indikator & Deskripsi\\
\hline
Perceived Opportunities & Persentase penduduk antara usia 18-64 tahun yang melihat peluang baik untuk memulai usaha. \\
\hline
Perceived Capabilities & Persentase penduduk antara usia 18-64 tahun yang percaya bahwa mereka mempunyai kemampuan untuk memulai suatu usaha. \\
\hline
Entreprenurial Intention & Persentase penduduk antara usia 18-64 tahun (selain orang yang berwirausaha) yang bertekad untuk mendirikan usaha dalam waktu tiga tahun kedepan.\\
\hline
Fear of Failure Rate & Persentase penduduk antara usia 18-64 tahun dapat melihat peluang baik yang mengindikasikan bahwa takut akan gagal akan menjauhkan mereka dari mendirikan usaha. \\
\hline
\end{tabular}
\label{tabelindikator}
\end{table}



Indikator-indikator menurut GEM yang mempengaruhi perkembangan kewirausahaan di Indonesia yaitu Perceived Capabilities, Role Model, Perceived Opportunity dan Fear of Failure. Berikut data pendidikan dan wilayah Indonesia dari GEM tentang Perceived Capabilities yang diambil pada tahun 2015 \cite{dataGEM}.


\begin{figure} [H]
	\centering  
	\includegraphics[width=14cm, height=7cm]{PCPendidikan} 
	\caption[Komposisi perceived capabilities untuk tingkat pendidikan yang berbeda]{Komposisi perceived capabilities untuk tingkat pendidikan yang berbeda} 
	\label{fig:PCPendidikan} 
\end{figure}

Dapat dilihat pada gambar \ref{fig:PCPendidikan} dijelaskan bahwa individu yang memiliki kemampuan berwirausaha tertinggi yaitu pada individu yang berpendidikan sekolah menengah ke atas (SMA). Pria mempunyai peluang yang lebih unggul (58.5\%) daripada wanita (55.4\%). Peluang yang paling rendah untuk menjadi wirausaha yaitu pada individu yang berpendidikan sampai S-3 yaitu 0.0\% untuk wanita dan 0.1\% untuk pria. 

\begin{figure} [H]
	\centering  
	\includegraphics[width=11cm, height=6cm]{PCRegion} 
	\caption[Komposisi perceived capabilities untuk wilayah Indonesia]{Komposisi perceived capabilities untuk wilayah Indonesia} 
	\label{fig:PCRegion} 
\end{figure}

Dapat dilihat pada gambar \ref{fig:PCRegion} dijelaskan bahwa individu yang memiliki kemampuan berwirausaha tertinggi yaitu pada wanita yang berada pada wilayah Indonesia Timur sebesar 53.5\% sedangkan pria yang berpeluang tinggi untuk menjadi wirausaha berada pada wilayah Indonesia Tengah sebesar 51.0\%. Individu yang memiliki kemampuan berwirausaha terendah yaitu untuk wanita berada pada wilayah Indonesia Tengah sebesar 49.0\% dan untuk pria berada pada wilayah Indonesia Timur sebesar 46.5\%. Data kedua yaitu data Role Model tentang perbedaan tingkat wirausaha antara perempuan dan laki-laki serta yang kedua adalah data pendidikan.

\begin{figure} [H]
	\centering  
	\includegraphics[width=12cm, height=7cm]{RMfemalemale} 
	\caption[Komposisi role model untuk wanita dan pria]{Komposisi role model untuk wanita dan pria} 
	\label{fig:RMfemalemale} 
\end{figure}


Pada gambar \ref{fig:RMfemalemale} dijelaskan individu yang memulai bisnis dalam 2 tahun terakhir. Peluang individu yang memulai bisnis dalam 2 tahun terakhir tertinggi yaitu pada wanita usia 25 sampai 34 tahun sebesar 30.1\% sedangkan pria sebesar 28.9\%. Peluang terendah yaitu pada wanita usia 55 sampai 64 tahun sebesar 10.4\% sedangkan pria sebesar 9.8\%.


\begin{figure} [H]
	\centering  
	\includegraphics[width=13cm, height=7cm]{RMpendidikan} 
	\caption[Komposisi role model untuk tingkat pendidikan yang berbeda]{Komposisi role model untuk tingkat pendidikan yang berbeda} 
	\label{fig:RMpendidikan} 
\end{figure}  


Pada gambar \ref{fig:RMpendidikan} dijelaskan individu yang memulai bisnis dalam 2 tahun terakhir. Peluang individu yang memulai bisnis dalam 2 tahun terakhir tertinggi pada individu yang mempunyai tingkat pendidikan sekolah menengah ke atas (SMA). Pria memperoleh persentase sebesar 58.3\% dan wanita sebesar 54.4\%. Individu yang mempunyai peluang terendah yaitu individu yang berpendidikan S-3. Pria memperoleh persentase sebesar 0.1\% dan wanita memperoleh persentase sebesar 0.0\%. Data ketiga yaitu data Perceived Opportunities tentang perbedaan tingkat wirausaha antara perempuan dan laki-laki serta yang kedua adalah data pendidikan.

\begin{figure} [H]
	\centering  
	\includegraphics[width=12cm, height=6cm]{POfemalemale} 
	\caption[Komposisi role model untuk wanita dan pria]{Komposisi role model untuk wanita dan pria} 
	\label{fig:POfemalemale} 
\end{figure} 

Pada gambar \ref{fig:POfemalemale} dijelaskan kemampuan individu antara pria dan wanita dalam melihat peluang berwirausaha. Peluang tertinggi yaitu pada pria berusia 25 sampai 34 tahun yang memiliki persentase sebesar 30.1\% dan wanita sebesar 28.2\%. Peluang terendah yaitu pada pria berusia 55 sampai 64 tahun sebesar 10.8\% dan wanita sebesar 10.9\%. 

\begin{figure} [H]
	\centering  
	\includegraphics[width=14cm, height=6cm]{POpendidikan} 
	\caption[Komposisi perceived opportunities untuk tingkat pendidikan yang berbeda]{Komposisi perceived opportunities untuk tingkat pendidikan yang berbeda} 
	\label{fig:POpendidikan} 
\end{figure}  

Gambar \ref{fig:POpendidikan} menjelaskan kemampuan individu dalam melihat peluang. Kemampuan melihat peluang berwirausaha tertinggi yaitu pada individu yang berpendidikan sekolah menengah ke atas (SMA). Persentase pria sebesar 59.3\% dan wanita sebesar 56.7\%. Kemampuan melihat peluang berwirausaha terendah yaitu pada individu yang berpendidikan S-3. Persentase pria sebesar 0.2\% dan wanita sebesar 0.0\%. Data keempat yaitu data Fear of Failure tentang perbedaan tingkat wirausaha antara perempuan dan laki-laki.

\begin{figure} [H]
	\centering  
	\includegraphics[width=13cm, height=6cm]{FOFfemalemale} 
	\caption[Komposisi fear of failure untuk wanita dan pria]{Komposisi fear of failure untuk wanita dan pria} 
	\label{fig:FOF} 
\end{figure}  

Gambar \ref{fig:FOF} menjelaskan perbedaan Fear of Failure antara pria dan wanita. Persentase Fear of Failure yang tertinggi yaitu pada wanita berusia 25 sampai 34 tahun sebesar 29.6\% dan pria sebesar 29.3\%. Persentase Fear of Failure terendah yaitu pada wanita usia 55 sampai 64 tahun sebesar 11.5\% dan pria sebesar 11.3\%. 

\section{Cellular Automata}
\label{sec:cellularautomata}

Cellular Automata (CA) diperkenalkan pertama kali oleh Ulam dan von Neumann pada tahun 1940. Cellular Automata sendiri merupakan model matematis untuk sistem dimana banyak komponen sederhana bertindak bersama untuk menghasilkan pola perilaku yang rumit \cite{referensiCA2}. Sebuah CA terdiri atas sekumpulan sel, tersusun dalam larik-larik (\textit{grid}). Setiap sel mempunyai satu dari sejumlah \textit{state} (kondisi) yang mungkin. \textit{State} dapat berubah sesuai dengan aturan tertentu. Perubahan \textit{state} dari sebuah sel dipengaruhi oleh \textit{state} dari sel-sel di sekitarnya atau disebut dengan sel tetangga.

\subsection{Karakteristik CA}
\begin{enumerate}
	\item Dimensi pada CA
		\begin{enumerate}
			\item CA Satu Dimensi
			
				\begin{figure} [H]
					\centering  
					\includegraphics[width=4cm, height=2cm]{CA1D} 
					\caption[CA 1 Dimensi]{CA 1 Dimensi} 
					\label{fig:CA1D} 
				\end{figure}
			
			Cellular Automata satu dimensi adalah cellular automata yang ruang selnya berupa array satu dimensi, sehingga masing-masing sel hanya memiliki dua tetangga yang tepat bersebelahan, kecuali sel paling pinggir yang hanya mempunyai satu tetangga.
			
			\item CA Dua Dimensi
			
			\begin{figure} [H]
					\centering  
					\includegraphics[width=4cm, height=2cm]{CA2D} 
					\caption[CA 2 Dimensi]{CA 2 Dimensi} 
					\label{fig:CA2D} 
				\end{figure}
			
			Cellular Automata dua dimensi adalah cellular automata yang ruang selnya biasanya berupa matriks, sehingga masing-masing sel memiliki lebih dari dua tetangga. CA dua dimensi yang sangat terkenal adalah Conway's \textit{Game of Life}. Setiap sel pada CA menggambarkan suatu individu yang dapat berada pada \textit{state} hidup atau mati.
			
			\item CA Tiga Dimensi
			
			\begin{figure} [H]
					\centering  
					\includegraphics[width=4cm, height=4cm]{CA3D} 
					\caption[CA 3 Dimensi]{CA 3 Dimensi} 
					\label{fig:CA3D} 
				\end{figure}
			
			Cellular Automata tiga dimensi adalah cellular automata yang ruang selnya memiliki baris, kolom dan kedalaman, sehingga jumlah tetangga setiap sel bisa lebih banyak lagi. \cite{referensiCA1}
		\end{enumerate}
		
	\item Aplikasi CA
		
		\begin{enumerate}
			\item Bidang Transportasi
			
			CA banyak digunakan untuk memodelkan lalu lintas, dengan tujuan utama biasanya adalah untuk mempelajari beban dari jalan-jalan di area tertentu. Contoh aplikasi CA dibidang transportasi ini adalah simulasi pengaturan lampu lalu lintas. Model dalam penelitian ini menggunakan CA 1 dimensi. Pada pergerakan  atau perpindahan lajur kendaraan, terdapat beberapa aturan yaitu : 

	\begin{figure} [H]
		\centering  
		\includegraphics[width=12cm, height=4cm]{aplikasi1} 
		\caption[Ilustrasi dua jalur]{Ilustrasi dua jalur} 
		\label{fig:aplikasiCA1} 
	\end{figure}
				\begin{enumerate}
				\item Kendaraan di depannya terlalu dekat (kecepatan >   \textit{gap}).
				\item Jalur di sebelahnya kosong.
				\item f\_gap $\geq$ kecepatan.
				\item b\_gap $\geq$ kecepatan\_maksimum.
				\item Peluang untuk pindah terpenuhi (rand() $\leq$ peluang\_pindah). \cite{referensiCA3}
			
			\end{enumerate}
			
			
			\item Bidang Kesehatan
			
			Pada bidang kesehatan, CA juga sering digunakan untuk pemodelan penyebaran penyakit. Biasanya masalah penyebaran penyakit dimodelkan dengan CA dua dimensi dan menggunakan aturan Game of Life dari Conway.
			
			\item Bidang Lingkungan / Ekologi
			
			CA juga dapat digunakan untuk pemodelan pada bidang lingkungan. Sebagai contoh Guy Engelen menggunakan CA dua dimensi untuk memodelkan perubahan penggunaan lahan akibat dorongan sosial-ekonomi, lingkungan dan kebijakan.
			
			\item Bidang Sains
			
			Pada bidang sains, khususnya fisika CA dapat digunakan untuk memodelkan pergerakan partikel dan juga permasalahan lainnya terkait dengan fisika kuantum. Pada bidang biologi, CA digunakan untuk memodelkan sel biologis.
		\end{enumerate}
		
\end{enumerate}




