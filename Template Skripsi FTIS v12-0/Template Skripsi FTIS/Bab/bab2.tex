%versi 2 (8-10-2016)
\chapter{Landasan Teori}
\label{chap:teori}


 
Pada bab ini akan dibahas mengenai dasar teori yang digunakan pada penyusunan tugas akhir. Pembahasan pertama mencakup hal-hal yang berkaitan dengan pengertian kewirausahaan dari umum sampai khusus yaitu kewirausahaan menurut GEM. Pembahasan kedua yaitu tentang teori dan aplikasi dari CA (Cellular Automata) khususnya tentang ECA (Entrepreneur Cellular Automata). Pembahasan terakhir tentang hal-hal lain yang mendukung implementasi perangkat lunak seperti... 


\section{Arti Kewirausahaan}
\label{sec:artiwirausaha}

\graphicspath{{images/}}

Secara umum arti kewirausahaan merupakan suatu proses dalam mengerjakan sesuatu yang baru dan berbeda yang bermanfaat bagi orang lain atau diri sendiri. Orang yang melakukan proses kewirausahaan adalah wirausaha. Ciri-ciri wirausaha antara lain yaitu berani mengambil risiko, memiliki semangat dan kemauan keras, memiliki jiwa pemimpin, dsb. Tujuan wirausaha sendiri yaitu menciptakan lapangan kerja yang baru dan meningkatkan jumlah para wirausaha di suatu negara.


Kewirausahaan menurut GEM merupakan proses yang terdiri dari fase-fase berbeda mulai dari niat mendirikan suatu usaha, menjalankan suatu usaha baru atau sudah berdiri, sampai dengan penghentian sebuah usaha. Proses ini dimulai dengan keterlibatan individu yang berpotensi untuk menjadi wirausaha, yaitu mereka yang percaya bahwa mereka mempunyai kemampuan untuk memulai suatu usaha, individu yang melihat kesempatan untuk berwirausaha dan individu yang tidak takut gagal dalam memulai suatu usaha.
\begin{figure} 
	\centering  
	\includegraphics[width=14cm, height=6cm]{Capture}  
	\caption[Fase Wirausaha]{Fase Wirausaha} 
	\label{fig:artiwirausaha} 
\end{figure}

Pada gambar \ref{fig:artiwirausaha}, dijelaskan fase pertama dari ilustrasi GEM adalah wirausaha \textit{nascent}. Wirausaha \textit{nascent} adalah mereka yang telah memulai suatu usaha baru namun masih sangat dini (< 3 bulan). Setelah lebih dari tiga bulan, wirausaha \textit{nascent} ini disebut Pemilik Usaha Baru (\textit{new business owner}). Fase ini dijalani sampai individu tersebut telah tiga setengah tahun tahun terlibat dalam kewirausahaan. Kegiatan pada fase wirausaha \textit{nascent} dan pemilik usaha baru masuk kedalam kelompok Total Early Stage Entrepreneurial Activity (TEA). Fase selanjutnya adalah fase dimana wirausaha disebut sebagai Pemilik Usaha Mapan (\textit{owner-manager of an established business}). 

GEM mempertimbangkan beberapa indikator yang mempengaruhi berlangsungnya kewirausahaan di suatu negara yaitu \textit{Entrepreneurial Intention}, \textit{Fear of Failure}, \textit{perceived opportunities} dan \textit{Perceived Capabilities}. \textit{Entrepreneurial Intention} mendeskripsikan populasi yang bertekad untuk mendirikan suatu usaha dalam waktu tiga tahun kedepan. \textit{Fear of Failure} mendeskripsikan populasi yang positif yang mengindikasikan bahwa takutnya gagal dalam menghambat mereka dalam mendirikan suatu usaha. \textit{Perceived Opportunities} mendeskripsikan populasi yang melihat kesempatan bagus untuk memulai suatu usaha di daerah tempat tinggal mereka. \textit{Perceived Capabilities} mendeskripsikan populasi yang merasa mempunyai kemampuan dan pengetahuan yang cukup untuk mendirikan suatu usaha.

GEM melihat penduduk yang berpotensi menjadi wirausaha di Indonesia
dilihat dari tiga indikator yaitu \textit{perceived opportunities}, \textit{perceived capabilities} dan \textit{role model}. \textit{Perceived Opportunities} mengukur persentase dari orang dewasa antara usia 18 sampai 64 tahun yang melihat kesempatan bagus untuk memulai usaha di tempat mereka tinggal. Seperti pada gambar \ref{fig:perceivedopportunity}, diantara semuanya yang melihat adanya kesempatan baik untuk memulai usaha baru, pria muda (antara 25 sampai 34 tahun) memiliki perceived opportunities lebih tinggi dari wanita yang seusianya. Namun, untuk wanita diatas usia 35 tahun melihat adanya kesempatan lebih tinggi dari pria pada kelompok usia yang sama.

\begin{figure} [H]
	\centering  
	\includegraphics[width=14cm, height=6cm]{POumur} 
	\caption[Komposisi perceived opportunity untuk kelompok usia yang berbeda]{Komposisi perceived opportunity untuk kelompok usia yang berbeda} 
	\label{fig:perceivedopportunity} 
\end{figure}

Menurut data GEM, orang dewasa yang berpendidikan sekolah menengah atas memiliki perceived opportunities paling tinggi di antara orang dewasa Indonesia. Ketika mereka menjalani pendidikan di universitas untuk pendidikan yang lebih tinggi, perceived opportunities mereka cenderung menurun (Lihat Gambar \ref{fig:POPendidikan}). Umumnya Jakarta memiliki persepsi yang lebih tinggi tentang perceived opportunities, diikuti kota Bandung, Surabaya, Semarang dan Surakarta. Kota-kota tersebut terletak di pulau Jawa dimana aktivitas ekonomi terkonsentrasi di sana. Kota yang terletak di pulau lain cenderung menerima persepsi adanya kesempatan yang lebih rendah (< 1\%) adalah Banda Aceh dan Pontianak (Lihat Gambar \ref{fig:POdomisili}). Berdasarkan tingkat pendapatan dan berdasarkan jenis kelamin, tidak ada perbedaan jenis kelamin pada perceived opportunities antara tingkat pendapatan. Terdapat 83,8\% dan 84,2\% wanita dengan pendapatan per bulan dibawah 5 juta rupiah yang mempertimbangkan adanya kesempatan yang baik untuk memulai usaha (Lihat Gambar \ref{fig:POpendapatan}).

\begin{figure} [H]
	\centering  
	\includegraphics[width=14cm, height=6cm]{POPendidikan} 
	\caption[Komposisi perceived opportunity untuk tingkat pendidikan yang berbeda] {Komposisi perceived opportunity untuk tingkat pendidikan yang berbeda} 
	\label{fig:POpendidikan} 
\end{figure}

\begin{figure} [H]
	\centering  
	\includegraphics[width=15cm, height=10cm]{POdomisili} 
	\caption[Komposisi perceived opportunity berdasarkan domisili]{Komposisi perceived opportunity berdasarkan domisili} 
	\label{fig:POdomisili} 
\end{figure}

\begin{figure} [H]
	\centering  
	\includegraphics[width=14cm, height=7cm]{POpendapatan} 
	\caption[Komposisi perceived opportunity berdasarkan pendapatan]{Komposisi perceived opportunity berdasarkan pendapatan} 
	\label{fig:POpendapatan} 
\end{figure}


 
