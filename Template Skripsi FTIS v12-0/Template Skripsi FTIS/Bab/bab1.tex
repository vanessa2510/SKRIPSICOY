%versi 2 (8-10-2016) 
\chapter{Pendahuluan}
\label{chap:intro}
   
\section{Latar Belakang}
\label{sec:label}

Pada saat ini, lapangan kerja pada suatu negara tidak bisa kita prediksi, tetapi kenyataan yang kita ketahui adalah lapangan kerja dari tahun ke tahun semakin terbatas. Dengan melihat situasi tersebut maka bisa dipastikan tingkat pengangguran di suatu negara akan semakin tinggi. Solusi terbaik untuk mengurangi permasalahan tersebut adalah dengan berwirausaha. Kewirausahaan adalah kemampuan seseorang untuk membuat suatu usaha yang dimulai dari 0 atau dimulai dari bawah yang dirintis hingga usaha tersebut benar-benar sukses. Tentu saja hal ini memberikan pengaruh positif terhadap pertumbuhan ekonomi suatu negara, karena kewirausahaan juga sekaligus membuka lapangan kerja bagi masyarakat dengan terbentuknya usaha yang semakin besar maka pasti dibutuhkan pekerja-pekerja untuk membantu usaha tersebut menjadi semakin besar.

 
Pada jaman sekarang, sudah banyak sekali orang yang lebih memilih untuk berwirausaha daripada bekerja di kantor atau di sebuah perusahaan. Seperti di Indonesia contohnya, Indonesia memiliki kemampuan yang tinggi untuk memulai bisnis baru. Alasan mengapa banyak orang lebih memilih berwirausaha pun bervariasi contohnya orang tersebut tidak terlalu menyukai waktu kerjanya diatur oleh orang lain melainkan ia lebih menyukai waktu kerjanya diatur oleh dirinya sendiri. Tidak hanya pada jaman sekarang, dari jaman dahulu juga sudah ada wirausaha yang namanya tidak asing lagi didengar oleh telinga kita salah satunya yaitu Bob Sadino.


Kesuksesan dalam berwirausaha juga dapat dilihat dari usaha yang sudah dilakukan. Proses untuk meraih usaha tersebut tentunya mengalami jatuh-bangun, sehingga keuletan dan semangat juang yang tinggi sangat diperlukan dalam berwirausaha. Akan tetapi, usaha yang sudah sukses juga harus terus dijaga kekonsistenannya. Jika tidak dijaga, usaha tersebut akan mengalami kebangkrutan. 


Melalui \textit{Cellular Automata} (CA) ini dapat dibuat sebuah model untuk mengetahui hubungan antara berbagai faktor yang mempengaruhi pertumbuhan kewirausahaan di suatu negara. \textit{Cellular Automata} menurut John von Neumann merupakan sistem dinamis diskrit yang memodelkan perilaku kompleks berdasarkan peraturan lokal sederhana yang menggerakkan sel pada kisi. Sebuah CA terdiri atas sekumpulan sel, tersusun dalam larik-larik (\textit{grid}). Setiap sel mempunyai satu dari sejumlah \textit{state} (kondisi) yang mungkin. \textit{State} dapat berubah menurut aturan tertentu. Perubahan \textit{state} dari sebuah sel dipengaruhi oleh \textit{state} dari sel-sel disekitarnya atau bisa disebut dengan sel tetangga.


\section{Rumusan Masalah}
\label{sec:rumusan}
Berikut adalah susunan permasalahan yang akan dibahas pada penelitian ini:


\begin{enumerate}
	\item Faktor apa saja yang mempengaruhi pertumbuhan kewirausahaan di suatu negara?
	\item Informasi apa saja yang disediakan oleh GEM yang berguna dalam pengembangan model pertumbuhan wirausaha?
	\item Bagaimana memodelkan pertumbuhan wirausaha dengan \textit{cellular automata}?
\end{enumerate}



\section{Tujuan}
\label{sec:tujuan}
Berdasarkan rumusan masalah yang telah dibuat, maka tujuan penelitian ini dijelaskan ke dalam poin-poin sebagai berikut :


\begin{enumerate}
	\item Mengetahui faktor apa saja yang mempengaruhi pertumbuhan kewirausahaan di suatu negara.
	\item Mengetahui cara kerja algoritma \textit{Cellular Automata}.
	\item Memodelkan \textit{Cellular Automata} untuk meningkatkan pertumbuhan kewirausahaan di suatu negara.
\end{enumerate}



