%versi 3 (18-12-2016)
\chapter{Kode Program}
\label{lamp:A}

%terdapat 2 cara untuk memasukkan kode program
% 1. menggunakan perintah \lstinputlisting (kode program ditempatkan di folder yang sama dengan file ini)
% 2. menggunakan environment lstlisting (kode program dituliskan di dalam file ini)
% Perhatikan contoh yang diberikan!!
%
% untuk keduanya, ada parameter yang harus diisi:
% - language: bahasa dari kode program (pilihan: Java, C, C++, PHP, Matlab, C#, HTML, R, Python, SQL, dll)
% - caption: nama file dari kode program yang akan ditampilkan di dokumen akhir
%
% Perhatian: Abaikan warning tentang textasteriskcentered!!
%

\begin{lstlisting}[language=Java, caption=CA.java]
/*
 * To change this license header, choose License Headers in Project Properties.
 * To change this template file, choose Tools | Templates
 * and open the template in the editor.
 */
package ecasimulatorjframe;

import java.util.Random;
import java.io.BufferedReader;
import java.io.BufferedWriter;
import java.io.FileReader;
import java.io.FileWriter;
import java.io.IOException;
import java.io.PrintWriter;

/**
 *
 * @author Vanessa
 */
public class CA {

    int popSize;
    int neighSize;
    double threshold; //ambang minimum untuk usaha berlanjut
    Entrepreneurs[] E;
    Neighborhoods N;
    PublicFactor pub;
    int[] S; 
    float P;
    float[] delta;
    float[] sigma;
    int numOfMonth;
    CA(int n, int m, int pf) {
        popSize = n;
        neighSize = m;
        E = new Entrepreneurs[n];
        N = new Neighborhoods(n, m);
        pub = new PublicFactor(pf);
        this.numOfMonth = 1; // bulan ke 1
    }
    /*
    * Method untuk menentukan perubahan individu wirausaha
    */
    Entrepreneurs[] stateTransition(CA model, double[] composition) {
        int size = model.popSize;
        Entrepreneurs[] nextEnt = new Entrepreneurs[size];

        for (int i = 0; i < size; i++) {
            nextEnt[i] = new Entrepreneurs();
            model.E[i].copy(nextEnt[i]);
            if (this.numOfMonth % 12 == 0) {
                nextEnt[i].age++; // tiap kelipatan 12 umurnya nambah
            }
            nextEnt[i].b_age++;
            nextLevel(nextEnt[i], i, model, composition);
        }
        this.numOfMonth++;
        return nextEnt;
    }

    /*
    * Method untuk menghitung kondisi ketetanggaan
    */
    double getNeighborIndex(CA model, int idxEnt) {
        int size = model.neighSize;
        double sum = 0.0;
        for (int i = 0; i < size; i++) {
            double sum1 = 0.0;
            for (int j = 0; j < model.popSize; j++) {
                sum1 = sum1 + model.N.neighbors[i].neighborMatrix[idxEnt][j];
            }
            sum = sum + sum1 / (model.popSize - 1) * model.N.weight[i];
        }
        return sum;
    }
    /*
    * Method untuk menentukan level wirausaha
    */
    void nextLevel(Entrepreneurs ne, int i, CA model, double[] composition) {
        //kasus umur yang sudah lebih dari 64th
        if (ne.age > (64)) {
            ne.level = State.RETIRED;
            ne.b_age = 0;
        } else {
            double idx = getIndex(i, model, composition);
            threshold = Double.parseDouble(InputDataHandler.getValue("threshold"));
            if (idx < threshold) {
                ne.level = State.POTENTIAL;
                ne.b_age = 0;
            } else {
                switch (ne.level) {
                    case 0: //potential
                        ne.level = State.NASCENT;
                        break;
                    case 1: //nascent
                        if (ne.b_age > 3) {
                            ne.level = State.NEW_BOM;
                            break;
                        }
                    case 2: //new_bm
                        if (ne.b_age > 42) {
                            ne.level = State.ESTABLISH_BOM;
                            break;
                        }
                }
            }
        }
    }
    /*
    * Method untuk menghitung Continuity Index
    */
    double getIndex(int i, CA model, double[] composition) {
        double hasil = composition[0] * model.E[i].point + composition[1] * this.getNeighborIndex(model, i) + composition[2] * this.pub.getPublicIdx();
        System.out.println("total hasil : "+hasil);
        return hasil;
    }

    /*
    * Method untuk mendefinisikan ketetanggaan
    * 0 jika sama dengan
    * 1 jika kurang dari sama dengan
    * 2 jika lebih dari sama dengan
    */
    //perubahan -> ditambahin casenya
    void NeighborhoodDefinition() {
        int n = this.N.numNeighbor;
        int ng = this.popSize;
        for (int i = 0; i < n; i++) {
            for (int j = 0; j < ng; j++) {
                for (int k = 0; k < ng; k++) {
                    switch (i) {
                        case 0: // level
                            // kalau relasinya sama dengan
                            if ((this.N.relation[i] == 0) && (this.E[j].level == this.E[k].level)) {
                                this.N.neighbors[i].neighborMatrix[j][k] = 1.0;
                            }
                            // kalau relasinya kurang dari sama dengan
                            if ((this.N.relation[i] == 1) && (this.E[j].level <= this.E[k].level)) {
                                this.N.neighbors[i].neighborMatrix[j][k] = 1.0;
                            }
                            // kalau relasinya lebih dari sama dengan
                            if ((this.N.relation[i] == 2) && (this.E[j].level >= this.E[k].level)) {
                                this.N.neighbors[i].neighborMatrix[j][k] = 1.0;
                            }
                            break;
                        case 1: // b_area
                            // kalau relasinya sama dengan
                            if ((this.N.relation[i] == 0) && (this.E[j].b_category == this.E[k].b_category) && (this.E[j].b_area == this.E[k].b_area)) {
                                this.N.neighbors[i].neighborMatrix[j][k] = 1.0;
                            }
                            break;
                        case 2: //location
                            // kalau relasinya sama dengan
                            if ((this.N.relation[i] == 0) && (this.E[j].location == this.E[k].location)) {
                                this.N.neighbors[i].neighborMatrix[j][k] = 1.0;
                            }
                            break;
                        case 3: // jenis kelamin
                            if ((this.N.relation[i] == 0) && (this.E[j].sex == this.E[k].sex)) {
                                this.N.neighbors[i].neighborMatrix[j][k] = 1.0;
                            }
                            break;
                        case 4: // umur
                            if ((this.N.relation[i] == 0) && (this.E[j].age == this.E[k].age)) {
                                this.N.neighbors[i].neighborMatrix[j][k] = 1.0;
                            }
                            // kalau relasinya kurang dari sama dengan
                            if ((this.N.relation[i] == 1) && (this.E[j].age <= this.E[k].age)) {
                                this.N.neighbors[i].neighborMatrix[j][k] = 1.0;
                            }
                            // kalau relasinya lebih dari sama dengan
                            if ((this.N.relation[i] == 2) && (this.E[j].age >= this.E[k].age)) {
                                this.N.neighbors[i].neighborMatrix[j][k] = 1.0;
                            }
                            break;

                        case 5: // pendidikan
                            if ((this.N.relation[i] == 0) && (this.E[j].education == this.E[k].education)) {
                                this.N.neighbors[i].neighborMatrix[j][k] = 1.0;
                            }
                            // kalau relasinya kurang dari sama dengan
                            if ((this.N.relation[i] == 1) && (this.E[j].education <= this.E[k].education)) {
                                this.N.neighbors[i].neighborMatrix[j][k] = 1.0;
                            }
                            // kalau relasinya lebih dari sama dengan
                            if ((this.N.relation[i] == 2) && (this.E[j].education >= this.E[k].education)) {
                                this.N.neighbors[i].neighborMatrix[j][k] = 1.0;
                            }
                            break;
                        case 6: // pendapatan
                            if ((this.N.relation[i] == 0) && (this.E[j].income == this.E[k].income)) {
                                this.N.neighbors[i].neighborMatrix[j][k] = 1.0;
                            }
                            // kalau relasinya kurang dari sama dengan
                            if ((this.N.relation[i] == 1) && (this.E[j].income <= this.E[k].income)) {
                                this.N.neighbors[i].neighborMatrix[j][k] = 1.0;
                            }
                            // kalau relasinya lebih dari sama dengan
                            if ((this.N.relation[i] == 2) && (this.E[j].income >= this.E[k].income)) {
                                this.N.neighbors[i].neighborMatrix[j][k] = 1.0;
                            }
                            break;
                    }
                }
            }
        }
    }

    void genDummyEntrepreneurs() {
        this.E[0] = new Entrepreneurs(false, 18 * 12, 0, 0, 3, 4, 3, 4, 0, 0.0);
        this.E[1] = new Entrepreneurs(true, 35 * 12, 0, 0, 2, 4, 0, 4, 1, 0.0);
        this.E[2] = new Entrepreneurs(false, 55 * 12, 0, 0, 1, 4, 0, 4, 2, 0.0);
        this.E[3] = new Entrepreneurs(true, 27 * 12, 0, 0, 3, 4, 1, 4, 1, 0.0);
        this.E[4] = new Entrepreneurs(true, 30 * 12, 0, 0, 1, 4, 1, 4, 0, 0.0);
        this.E[5] = new Entrepreneurs(false, 45 * 12, 0, 0, 1, 4, 2, 4, 4, 0.0);
        this.E[6] = new Entrepreneurs(false, 33 * 12, 0, 0, 2, 4, 3, 4, 2, 0.0);
        this.E[7] = new Entrepreneurs(true, 20 * 12, 0, 0, 3, 4, 2, 4, 0, 0.0);
        this.E[8] = new Entrepreneurs(false, 38 * 12, 0, 0, 5, 4, 3, 4, 1, 0.0);
        this.E[9] = new Entrepreneurs(false, 41 * 12, 0, 0, 5, 4, 0, 4, 0, 0.0);
    }

    void genSimulationData() {
        int nSim = this.popSize;
        Random r = new Random();
        int n;
        for (int i = 0; i < nSim; i++) {
            this.E[i] = new Entrepreneurs();
            n = r.nextInt(nSim);
            if (n < nSim * 0.6) {
                this.E[i].sex = State.FEMALE;
            } else {
                this.E[i].sex = State.MALE;
            }

            //location
            n = r.nextInt(16);
            this.E[i].location = n;
            //category business, ada 3 dan area bisnis
            n = r.nextInt();
            this.E[i].b_category = n;
            switch (this.E[i].b_category) {
                case 0:
                    this.E[i].b_area = r.nextInt(3);
                    break;
                case 1:
                    this.E[i].b_area = r.nextInt(12);
                    break;
                case 2:
                    this.E[i].b_area = r.nextInt(16);
                    break;
            }

            //income
            n = r.nextInt(6);
            this.E[i].income = n;

            //education
            n = r.nextInt(6);
            this.E[i].education = n;

            int m = r.nextInt(100);
            if (m > 80) { // 18-24
                this.E[i].age = (r.nextInt(7) + 18) * 12;
            } else if (m > 60) { // 25-34
                this.E[i].age = (r.nextInt(10) + 25) * 12;
            } else if (m > 40) { // 35-44
                this.E[i].age = (r.nextInt(10) + 35) * 12;
            } else if (m > 20) { // 45-54
                this.E[i].age = (r.nextInt(10) + 45) * 12;
            } else if (m > 5) { // 55-64
                this.E[i].age = (r.nextInt(10) + 55) * 12;
            } else {
                this.E[i].age = 65 * 12;
            }

            if (this.E[i].age > 64 * 12) {
                this.E[i].level = State.RETIRED;
            } else {
                if (this.E[i].age < 25 * 12) {
                    n = r.nextInt(100);
                    if (n < 70) {
                        this.E[i].level = State.POTENTIAL;
                    } else if (n < 90) {
                        this.E[i].level = State.NASCENT;
                    } else {
                        this.E[i].level = State.NEW_BOM;
                    }
                } else {
                    if (this.E[i].age < 35 * 12) {
                        n = r.nextInt(100);
                        if (n < 60) {
                            this.E[i].level = State.POTENTIAL;
                        } else if (n < 85) {
                            this.E[i].level = State.NASCENT;
                        } else if (n < 95) {
                            this.E[i].level = State.NEW_BOM;
                        } else {
                            this.E[i].level = State.ESTABLISH_BOM;
                        }
                    } else {
                        if (this.E[i].age < 45 * 12) {
                            n = r.nextInt(100);
                            if (n < 45) {
                                this.E[i].level = State.POTENTIAL;
                            } else if (n < 60) {
                                this.E[i].level = State.NASCENT;
                            } else if (n < 80) {
                                this.E[i].level = State.NEW_BOM;
                            } else {
                                this.E[i].level = State.ESTABLISH_BOM;
                            }
                        } else {
                            if (this.E[i].age < 55 * 12) {
                                n = r.nextInt(100);
                                if (n < 25) {
                                    this.E[i].level = State.POTENTIAL;
                                } else if (n < 55) {
                                    this.E[i].level = State.NASCENT;
                                } else if (n < 75) {
                                    this.E[i].level = State.NEW_BOM;
                                } else {
                                    this.E[i].level = State.ESTABLISH_BOM;
                                }
                            } else {
                                n = r.nextInt(100);
                                if (n < 10) {
                                    this.E[i].level = State.POTENTIAL;
                                } else if (n < 20) {
                                    this.E[i].level = State.NASCENT;
                                } else if (n < 50) {
                                    this.E[i].level = State.NEW_BOM;
                                } else {
                                    this.E[i].level = State.ESTABLISH_BOM;
                                }
                            }
                            //umur bisnis
                            switch (this.E[i].level) {
                                case 1:
                                    this.E[i].b_age = r.nextInt(3) + 1;
                                    break;
                                case 2:
                                    this.E[i].b_age = r.nextInt(39) + 4;
                                    break;
                                case 3:
                                    this.E[i].b_age = r.nextInt(this.E[i].age = 18 * 12) + 1;
                                    break;
                            }
                        }
                    }
                }
            }
        }
    }

    void writeSimulationData(String namaFile) {
        String teks = "";
        try {
            PrintWriter out = new PrintWriter(new BufferedWriter(new FileWriter(namaFile, true)));
            System.out.println(this.popSize);
            for (int i = 0; i < this.popSize; i++) {
                teks = "";
                if (this.E[i].sex) {
                    teks = teks + "1, ";
                } else {
                    teks = teks + "0, ";
                }
                teks = teks + this.E[i].age + ", ";
                teks = teks + this.E[i].b_age + ", ";
                teks = teks + this.E[i].b_category + ", ";
                teks = teks + this.E[i].b_area + ", ";
                teks = teks + this.E[i].education + ", ";
                teks = teks + this.E[i].location + ", ";
                teks = teks + this.E[i].income + ", ";
                teks = teks + this.E[i].level;
                System.out.println(teks);
            }
            System.out.close();
        } catch (IOException e) {
            System.out.println("Gagal menulis ke file " + namaFile);
            e.printStackTrace();
        }
    }

    void readSimulationData(String fileName) {
        String line = "";
        String separator = ", ";
        BufferedReader br = null;
        String teks = "";
        String[] jm;

        try {
            br = new BufferedReader(new FileReader(fileName));
            System.out.println("proses baca file...");
            line = br.readLine();
            line = br.readLine();
            this.popSize = Integer.parseInt(line.trim());
            for (int i = 0; i < this.popSize; i++) {
                line = br.readLine();
                jm = line.split(separator);

                this.E[i] = new Entrepreneurs();
                // sex, false = 0, true = 1
                if (Integer.parseInt(jm[0].trim()) == 0) {
                    this.E[i].sex = false;
                } else {
                    this.E[i].sex = true;
                }
                this.E[i].age = Integer.parseInt(jm[1].trim());
                this.E[i].b_age = Integer.parseInt(jm[2].trim());
                this.E[i].b_category = Integer.parseInt(jm[3].trim());
                this.E[i].b_area = Integer.parseInt(jm[4].trim());
                this.E[i].education = Integer.parseInt(jm[5].trim());
                this.E[i].location = Integer.parseInt(jm[6].trim());
                this.E[i].income = Integer.parseInt(jm[7].trim());
                this.E[i].level = Integer.parseInt(jm[8].trim());
                this.E[i].point = 0.0;
            }
            br.close();

        } catch (IOException e) {
            System.out.println("Gagal membaca dari file " + fileName);
            e.printStackTrace();
        }

    }
    /*
    * Method untuk mengeluarkan jumlah wirausaha pada level tertentu
    */
    String print(int iter) {
        int l0 = 0;
        int l1 = 0;
        int l2 = 0;
        int l3 = 0;
        int l4 = 0;
        for (int i = 0; i < this.popSize; i++) {
            switch (this.E[i].level) {
                case 0:
                    l0++;
                    break;
                case 1:
                    l1++;
                    break;
                case 2:
                    l2++;
                    break;
                case 3:
                    l3++;
                    break;
                case 4:
                    l4++;
                    break;
            }
        }
        return (iter + ", " + l0 + ", " + l1 + ", " + l2 + ", " + l3 + ", " + l4);
    }
    /*
    * Method untuk menghitung kondisi internal wirausaha
    * parameternya berisi dengan nilai-nilai atribut psikologis dari GEM 2013
    */
    // perubahan : ditambahin faktor psikologisnya
    void calculatePoint(double[] POAm, double[] POAf, double[] POEm, double[] POEf, double[] POLm, double[] POLf, double[] POIm, double[] POIf, double[] PCAm, double[] PCAf, double[] PCEm, double[] PCEf, double[] PCLm, double[] PCLf, double[] PCIm, double[] PCIf, double[] RMAm, double[] RMAf, double[] RMIm, double[] RMIf, double[] FFAf, double[] FFAm, double[] FFEf, double[] FFEm, double[] FFLf, double[] FFLm, double[] MALf, double[] MALm, double[] MAIf, double[] MAIm, double[] HSSIf, double[] HSSIm, double[] HSSLf, double[] HSSLm, double[] HSSAf, double[] HSSAm, double[] HSSEf, double[] HSSEm) {
        for (int i = 0; i < this.popSize; i++) {
            int a = getAgeRange(E[i].age);
            if (this.E[i].sex) {
                E[i].point = (POAm[a] + POEm[E[i].education] + POLm[E[i].location] + POIm[E[i].income]) * 0.2 + (PCAm[a] + PCEm[E[i].education] + PCLm[E[i].location] + PCIm[E[i].income]) * 0.25 + (RMAm[a] + RMIm[E[i].income]) * 0.3 + (FFAm[a] + FFEm[E[i].education] + FFLm[E[i].location]) * 0.1 + (MALm[E[i].location] + MAIm[E[i].income]) * 0.05 + (HSSAm[a] + HSSIm[E[i].income] + HSSLm[E[i].location] + HSSEm[E[i].education]) * 0.1;
            } else {
                E[i].point = (POAf[a] + POEf[E[i].education] + POLf[E[i].location] + POIf[E[i].income]) * 0.2 + (PCAf[a] + PCEf[E[i].education] + PCLf[E[i].location] + PCIf[E[i].income]) * 0.25 + (RMAf[a] + RMIf[E[i].income]) * 0.3 + (FFAf[a] + FFEf[E[i].education] + FFLf[E[i].location]) * 0.1 + (MALf[E[i].location] + MAIf[E[i].income]) * 0.05 + (HSSAf[a] + HSSIf[E[i].income] + HSSLf[E[i].location] + HSSEf[E[i].education]) * 0.1;
            }
        }
    }
    /*
    * Method untuk mengelompokkan rentang umur sesuai dengan GEM 2013
    * a merupakan umur dari wirausaha
    */
    int getAgeRange(int a) {
        int ageC = -1;
        if (a >= 55 && a <= 64) {
            ageC = 0;
        }
        if (a >= 45 && a <= 54 ) {
            ageC = 1;
        }
        if (a >= 35  && a <= 44 ) {
            ageC = 2;
        }
        if (a >= 25  && a <= 34 ) {
            ageC = 3;
        }
        if (a >= 18  && a <= 24) {
            ageC = 4;
        }
        return ageC;
    }

}

\end{lstlisting}
\begin{lstlisting}[language=Java, caption=Entrepreneurs.java]

/*
 * To change this license header, choose License Headers in Project Properties.
 * To change this template file, choose Tools | Templates
 * and open the template in the editor.
 */
package ecasimulatorjframe;

/**
 *
 * @author Vanessa
 */
public class Entrepreneurs {
    int level;
    int age;
    boolean sex;
    int b_age;
    int b_category; // bidang usaha, misal makanan
    int b_area;// makanan ringan, makanan berat
    int education;
    int location;
    int income;
    double point;
    //penambahan
    
    Entrepreneurs(){
        sex=false;
        age = 0;
        b_age = 0;
        b_category = 0;
        b_area = 0;
        education = 0;
        location = 0;
        income = 0;
        level = 0;
        point = 0.0;
    }
    
    Entrepreneurs(boolean s, int a, int ba, int cat,int area, int edu, int loc, int inc, int l, double p){
        this.sex = s;
        this.age = a;
        this.b_age = ba;
        this.b_category = cat;
        this.b_area = area;
        this.education = edu;
        this.location = loc;
        this.income = inc; //pendapatan
        this.level = l;
        this.point = p;
    }
    
    void copy(Entrepreneurs e){
        e.sex = this.sex;
        e.age = this.age;
        e.b_age = this.b_age;
        e.b_category = this.b_category;
        e.b_area = this.b_area;
        e.education = this.education;
        e.location = this.location;
        e.income = this.income;
        e.level = this.level;
    }
    /*
    * method untuk mengeluarkan hasil perubahan individu wirausahawan
    */
    public String toString2(){
        return sex +" , "+ age + " , "+b_age+" , "+b_category+" , "+b_area+" , "+education +" , "+ location+" , "+ income +" , "+ level;
    }  
    
    void genDummy(CA model){
    
    }
}


\end{lstlisting}

\begin{lstlisting}[language=Java, caption=Neighbor.java]
/*
 * To change this license header, choose License Headers in Project Properties.
 * To change this template file, choose Tools | Templates
 * and open the template in the editor.
 */
package ecasimulatorjframe;

/**
 *
 * @author Vanessa
 */
public class Neighbor {
    double[][] neighborMatrix;
    /*
    * Method untuk membuat matriks berdasarkan atribut tetangga
    * n merupakan jumlah wirausaha
    */
    Neighbor(int n){
        neighborMatrix = new double[n][n];
        for (int i = 0; i < n; i++) {
            for (int j = 0; j < n; j++) {
                neighborMatrix[i][j] = 0.0; 
            }
        }
    }
}
\end{lstlisting}

\begin{lstlisting}[language=Java, caption=Neighborhoods.java]
/*
 * To change this license header, choose License Headers in Project Properties.
 * To change this template file, choose Tools | Templates
 * and open the template in the editor.
 */
package ecasimulatorjframe;

/**
 *
 * @author Vanessa
 * himpunan ketetanggaan tersusun atas sejumlah ketetanggaan
 */
public class Neighborhoods {
    int numNeighbor; // banyaknya ketetanggaan
    Neighbor[] neighbors;
    double[] weight;
    int[] relation; // jenis hubungan ketetanggaan, sama dengan, lebih kecil atau yang lain --> perlu didefinisikan
    /*
    * konstruktor untuk membuat matriks neighbor berdasarkan banyaknya tetangga
    * n untuk jumlah wirausaha
    * m untuk banyaknya tetangga
    */
    Neighborhoods(int n, int m){
        this.numNeighbor = m;
        neighbors = new Neighbor[m];
        weight = new double[m];
        relation = new int[m];
        
        for (int i = 0; i < m; i++) {
            this.numNeighbor = m;
            neighbors[i] = new Neighbor(n);
            weight[i] = 0.0;
            relation[i] = 0;
        }
    }

    public void setWeight(double[] weight) {
        this.weight = weight;
    }

    public void setRelation(int[] relation) {
        this.relation = relation;
    }

    public void setNumNeighbor(int numNeighbor) {
        this.numNeighbor = numNeighbor;
    }
    
    
}
\end{lstlisting}

\begin{lstlisting}[language=Java, caption=PublicFactor.java]
/*
 * To change this license header, choose License Headers in Project Properties.
 * To change this template file, choose Tools | Templates
 * and open the template in the editor.
 */
package ecasimulatorjframe;

/**
 *
 * @author Vanessa
 */
public class PublicFactor {
    
    double[] factors;
    double[] weights;

    PublicFactor(int n) {
        factors = new double[n];
        weights = new double[n];
    }

    public void setFactors(double[] f) {
        this.factors = f;
    }

    public void setWeights(double[] w) {
        this.weights = w;
    }
    /*
    * Method untuk menghitung hasil faktor publik
    */
    double getPublicIdx() {
        double idx = 0.0;
        for (int i = 0; i < factors.length; i++) {
            idx = idx + factors[i] * weights[i];
        }
        return idx / factors.length;
    }

}

\end{lstlisting}

\begin{lstlisting}[language=Java, caption=State.java]
/*
 * To change this license header, choose License Headers in Project Properties.
 * To change this template file, choose Tools | Templates
 * and open the template in the editor.
 */
package ecasimulatorjframe;

/**
 *
 * @author Vanessa
 */
public class State {
    public static int POTENTIAL = 0;
    public static int NASCENT = 1;
    public static int NEW_BOM = 2;
    public static int ESTABLISH_BOM = 3;
    public static int RETIRED = 4;
    public static boolean FEMALE = false;
    public static boolean MALE = true;
}


\end{lstlisting}

\begin{lstlisting}[language=Java, caption=TampilanBobotKetetanggaan.java]
/*
 * To change this license header, choose License Headers in Project Properties.
 * To change this template file, choose Tools | Templates
 * and open the template in the editor.
 */
package ecasimulatorjframe;

import javax.swing.JOptionPane;

/**
 *
 * @author Vanessa
 */
public class TampilanBobotKetetanggaan extends javax.swing.JFrame {

    /**
     * Creates new form ECASimulator
     */
    public TampilanBobotKetetanggaan() {
        initComponents();
        nilaiUmurInternal.setEnabled(false);
        nilaiLokasiInternal.setEnabled(false);
        nilaiPendapatanInternal.setEnabled(false);
        nilaiPendidikanInternal.setEnabled(false);
        nilaiUsahaInternal.setEnabled(false);
        nilaiJenisKelaminInternal.setEnabled(false);
        nilaiLevelInternal.setEnabled(false);

    }

    /**
     * This method is called from within the constructor to initialize the form.
     * WARNING: Do NOT modify this code. The content of this method is always
     * regenerated by the Form Editor.
     */
    @SuppressWarnings("unchecked")
    // <editor-fold defaultstate="collapsed" desc="Generated Code">                          
    private void initComponents() {

        jPanel1 = new javax.swing.JPanel();
        jLabel1 = new javax.swing.JLabel();
        jLabel2 = new javax.swing.JLabel();
        umurCBInternal = new javax.swing.JCheckBox();
        levelCBInternal = new javax.swing.JCheckBox();
        pendidikanCBInternal = new javax.swing.JCheckBox();
        pendapatanCBInternal = new javax.swing.JCheckBox();
        jenisKelaminCBInternal = new javax.swing.JCheckBox();
        lokasiCBInternal = new javax.swing.JCheckBox();
        bUsahaCBInternal = new javax.swing.JCheckBox();
        nilaiUmurInternal = new javax.swing.JTextField();
        nilaiLevelInternal = new javax.swing.JTextField();
        nilaiPendidikanInternal = new javax.swing.JTextField();
        nilaiPendapatanInternal = new javax.swing.JTextField();
        nilaiJenisKelaminInternal = new javax.swing.JTextField();
        nilaiLokasiInternal = new javax.swing.JTextField();
        nilaiUsahaInternal = new javax.swing.JTextField();
        jLabel3 = new javax.swing.JLabel();
        jLabel4 = new javax.swing.JLabel();
        jLabel5 = new javax.swing.JLabel();
        jLabel6 = new javax.swing.JLabel();
        jLabel7 = new javax.swing.JLabel();
        jLabel8 = new javax.swing.JLabel();
        jLabel9 = new javax.swing.JLabel();
        nextButton = new javax.swing.JButton();

        setDefaultCloseOperation(javax.swing.WindowConstants.EXIT_ON_CLOSE);

        jLabel1.setFont(new java.awt.Font("Tahoma", 1, 14)); // NOI18N
        jLabel1.setText("SIMULATOR ECA");

        jLabel2.setText("Bobot Ketetanggaan Wirausaha :");

        umurCBInternal.setText("UMUR :");
        umurCBInternal.addMouseListener(new java.awt.event.MouseAdapter() {
            public void mouseClicked(java.awt.event.MouseEvent evt) {
                umurCBInternalMouseClicked(evt);
            }
        });
        umurCBInternal.addActionListener(new java.awt.event.ActionListener() {
            public void actionPerformed(java.awt.event.ActionEvent evt) {
                umurCBInternalActionPerformed(evt);
            }
        });

        levelCBInternal.setText("LEVEL :");
        levelCBInternal.addMouseListener(new java.awt.event.MouseAdapter() {
            public void mouseClicked(java.awt.event.MouseEvent evt) {
                levelCBInternalMouseClicked(evt);
            }
        });

        pendidikanCBInternal.setText("PENDIDIKAN :");
        pendidikanCBInternal.addMouseListener(new java.awt.event.MouseAdapter() {
            public void mouseClicked(java.awt.event.MouseEvent evt) {
                pendidikanCBInternalMouseClicked(evt);
            }
        });

        pendapatanCBInternal.setText("PENDAPATAN :");
        pendapatanCBInternal.addMouseListener(new java.awt.event.MouseAdapter() {
            public void mouseClicked(java.awt.event.MouseEvent evt) {
                pendapatanCBInternalMouseClicked(evt);
            }
        });
        pendapatanCBInternal.addActionListener(new java.awt.event.ActionListener() {
            public void actionPerformed(java.awt.event.ActionEvent evt) {
                pendapatanCBInternalActionPerformed(evt);
            }
        });

        jenisKelaminCBInternal.setText("JENIS KELAMIN :");
        jenisKelaminCBInternal.addMouseListener(new java.awt.event.MouseAdapter() {
            public void mouseClicked(java.awt.event.MouseEvent evt) {
                jenisKelaminCBInternalMouseClicked(evt);
            }
        });

        lokasiCBInternal.setText("LOKASI :");
        lokasiCBInternal.addMouseListener(new java.awt.event.MouseAdapter() {
            public void mouseClicked(java.awt.event.MouseEvent evt) {
                lokasiCBInternalMouseClicked(evt);
            }
        });

        bUsahaCBInternal.setText("BIDANG USAHA :");
        bUsahaCBInternal.addMouseListener(new java.awt.event.MouseAdapter() {
            public void mouseClicked(java.awt.event.MouseEvent evt) {
                bUsahaCBInternalMouseClicked(evt);
            }
        });

        nilaiUmurInternal.addContainerListener(new java.awt.event.ContainerAdapter() {
            public void componentAdded(java.awt.event.ContainerEvent evt) {
                nilaiUmurInternalComponentAdded(evt);
            }
        });
        nilaiUmurInternal.addMouseListener(new java.awt.event.MouseAdapter() {
            public void mouseClicked(java.awt.event.MouseEvent evt) {
                nilaiUmurInternalMouseClicked(evt);
            }
        });
        nilaiUmurInternal.addInputMethodListener(new java.awt.event.InputMethodListener() {
            public void caretPositionChanged(java.awt.event.InputMethodEvent evt) {
            }
            public void inputMethodTextChanged(java.awt.event.InputMethodEvent evt) {
                nilaiUmurInternalInputMethodTextChanged(evt);
            }
        });
        nilaiUmurInternal.addActionListener(new java.awt.event.ActionListener() {
            public void actionPerformed(java.awt.event.ActionEvent evt) {
                nilaiUmurInternalActionPerformed(evt);
            }
        });

        nilaiLevelInternal.addActionListener(new java.awt.event.ActionListener() {
            public void actionPerformed(java.awt.event.ActionEvent evt) {
                nilaiLevelInternalActionPerformed(evt);
            }
        });

        nilaiPendapatanInternal.addActionListener(new java.awt.event.ActionListener() {
            public void actionPerformed(java.awt.event.ActionEvent evt) {
                nilaiPendapatanInternalActionPerformed(evt);
            }
        });

        nilaiLokasiInternal.addActionListener(new java.awt.event.ActionListener() {
            public void actionPerformed(java.awt.event.ActionEvent evt) {
                nilaiLokasiInternalActionPerformed(evt);
            }
        });

        jLabel3.setText("%");

        jLabel4.setText("%");

        jLabel5.setText("%");

        jLabel6.setText("%");

        jLabel7.setText("%");

        jLabel8.setText("%");

        jLabel9.setText("%");

        nextButton.setText("NEXT");
        nextButton.addMouseListener(new java.awt.event.MouseAdapter() {
            public void mouseClicked(java.awt.event.MouseEvent evt) {
                nextButtonMouseClicked(evt);
            }
        });
        nextButton.addActionListener(new java.awt.event.ActionListener() {
            public void actionPerformed(java.awt.event.ActionEvent evt) {
                nextButtonActionPerformed(evt);
            }
        });

        javax.swing.GroupLayout jPanel1Layout = new javax.swing.GroupLayout(jPanel1);
        jPanel1.setLayout(jPanel1Layout);
        jPanel1Layout.setHorizontalGroup(
            jPanel1Layout.createParallelGroup(javax.swing.GroupLayout.Alignment.LEADING)
            .addGroup(jPanel1Layout.createSequentialGroup()
                .addGroup(jPanel1Layout.createParallelGroup(javax.swing.GroupLayout.Alignment.LEADING)
                    .addGroup(jPanel1Layout.createSequentialGroup()
                        .addGroup(jPanel1Layout.createParallelGroup(javax.swing.GroupLayout.Alignment.LEADING)
                            .addGroup(jPanel1Layout.createSequentialGroup()
                                .addContainerGap()
                                .addComponent(jLabel2))
                            .addGroup(jPanel1Layout.createSequentialGroup()
                                .addGap(27, 27, 27)
                                .addGroup(jPanel1Layout.createParallelGroup(javax.swing.GroupLayout.Alignment.LEADING)
                                    .addGroup(jPanel1Layout.createParallelGroup(javax.swing.GroupLayout.Alignment.TRAILING)
                                        .addComponent(nilaiLevelInternal, javax.swing.GroupLayout.PREFERRED_SIZE, 53, javax.swing.GroupLayout.PREFERRED_SIZE)
                                        .addGroup(jPanel1Layout.createSequentialGroup()
                                            .addGroup(jPanel1Layout.createParallelGroup(javax.swing.GroupLayout.Alignment.LEADING)
                                                .addComponent(umurCBInternal)
                                                .addComponent(levelCBInternal)
                                                .addComponent(pendidikanCBInternal)
                                                .addComponent(pendapatanCBInternal))
                                            .addGroup(jPanel1Layout.createParallelGroup(javax.swing.GroupLayout.Alignment.LEADING)
                                                .addGroup(jPanel1Layout.createSequentialGroup()
                                                    .addGap(34, 34, 34)
                                                    .addComponent(nilaiUmurInternal, javax.swing.GroupLayout.PREFERRED_SIZE, 53, javax.swing.GroupLayout.PREFERRED_SIZE))
                                                .addGroup(javax.swing.GroupLayout.Alignment.TRAILING, jPanel1Layout.createSequentialGroup()
                                                    .addPreferredGap(javax.swing.LayoutStyle.ComponentPlacement.RELATED)
                                                    .addGroup(jPanel1Layout.createParallelGroup(javax.swing.GroupLayout.Alignment.LEADING)
                                                        .addComponent(nilaiPendidikanInternal, javax.swing.GroupLayout.Alignment.TRAILING, javax.swing.GroupLayout.PREFERRED_SIZE, 53, javax.swing.GroupLayout.PREFERRED_SIZE)
                                                        .addComponent(nilaiPendapatanInternal, javax.swing.GroupLayout.Alignment.TRAILING, javax.swing.GroupLayout.PREFERRED_SIZE, 53, javax.swing.GroupLayout.PREFERRED_SIZE)))))
                                        .addComponent(nilaiJenisKelaminInternal, javax.swing.GroupLayout.PREFERRED_SIZE, 53, javax.swing.GroupLayout.PREFERRED_SIZE)
                                        .addComponent(nilaiLokasiInternal, javax.swing.GroupLayout.PREFERRED_SIZE, 53, javax.swing.GroupLayout.PREFERRED_SIZE)
                                        .addComponent(nilaiUsahaInternal, javax.swing.GroupLayout.PREFERRED_SIZE, 53, javax.swing.GroupLayout.PREFERRED_SIZE))
                                    .addGroup(jPanel1Layout.createParallelGroup(javax.swing.GroupLayout.Alignment.LEADING)
                                        .addGroup(javax.swing.GroupLayout.Alignment.TRAILING, jPanel1Layout.createSequentialGroup()
                                            .addComponent(jenisKelaminCBInternal)
                                            .addGap(68, 68, 68))
                                        .addComponent(lokasiCBInternal)
                                        .addComponent(bUsahaCBInternal)))
                                .addGap(18, 18, 18)
                                .addGroup(jPanel1Layout.createParallelGroup(javax.swing.GroupLayout.Alignment.LEADING)
                                    .addComponent(jLabel3)
                                    .addComponent(jLabel4)
                                    .addComponent(jLabel5)
                                    .addComponent(jLabel6)
                                    .addComponent(jLabel7)
                                    .addComponent(jLabel8)
                                    .addComponent(jLabel9)))
                            .addGroup(jPanel1Layout.createSequentialGroup()
                                .addGap(131, 131, 131)
                                .addComponent(jLabel1)))
                        .addGap(0, 129, Short.MAX_VALUE))
                    .addGroup(javax.swing.GroupLayout.Alignment.TRAILING, jPanel1Layout.createSequentialGroup()
                        .addGap(0, 0, Short.MAX_VALUE)
                        .addComponent(nextButton)))
                .addContainerGap())
        );
        jPanel1Layout.setVerticalGroup(
            jPanel1Layout.createParallelGroup(javax.swing.GroupLayout.Alignment.LEADING)
            .addGroup(jPanel1Layout.createSequentialGroup()
                .addContainerGap()
                .addComponent(jLabel1)
                .addGap(36, 36, 36)
                .addComponent(jLabel2)
                .addGap(18, 18, 18)
                .addGroup(jPanel1Layout.createParallelGroup(javax.swing.GroupLayout.Alignment.BASELINE)
                    .addComponent(umurCBInternal)
                    .addComponent(nilaiUmurInternal, javax.swing.GroupLayout.PREFERRED_SIZE, javax.swing.GroupLayout.DEFAULT_SIZE, javax.swing.GroupLayout.PREFERRED_SIZE)
                    .addComponent(jLabel3))
                .addGap(18, 18, 18)
                .addGroup(jPanel1Layout.createParallelGroup(javax.swing.GroupLayout.Alignment.BASELINE)
                    .addComponent(levelCBInternal)
                    .addComponent(nilaiLevelInternal, javax.swing.GroupLayout.PREFERRED_SIZE, javax.swing.GroupLayout.DEFAULT_SIZE, javax.swing.GroupLayout.PREFERRED_SIZE)
                    .addComponent(jLabel4))
                .addGap(18, 18, 18)
                .addGroup(jPanel1Layout.createParallelGroup(javax.swing.GroupLayout.Alignment.BASELINE)
                    .addComponent(pendidikanCBInternal)
                    .addComponent(nilaiPendidikanInternal, javax.swing.GroupLayout.PREFERRED_SIZE, javax.swing.GroupLayout.DEFAULT_SIZE, javax.swing.GroupLayout.PREFERRED_SIZE)
                    .addComponent(jLabel5))
                .addGap(18, 18, 18)
                .addGroup(jPanel1Layout.createParallelGroup(javax.swing.GroupLayout.Alignment.BASELINE)
                    .addComponent(pendapatanCBInternal)
                    .addComponent(nilaiPendapatanInternal, javax.swing.GroupLayout.PREFERRED_SIZE, javax.swing.GroupLayout.DEFAULT_SIZE, javax.swing.GroupLayout.PREFERRED_SIZE)
                    .addComponent(jLabel6))
                .addGap(18, 18, 18)
                .addGroup(jPanel1Layout.createParallelGroup(javax.swing.GroupLayout.Alignment.BASELINE)
                    .addComponent(jenisKelaminCBInternal)
                    .addComponent(nilaiJenisKelaminInternal, javax.swing.GroupLayout.PREFERRED_SIZE, javax.swing.GroupLayout.DEFAULT_SIZE, javax.swing.GroupLayout.PREFERRED_SIZE)
                    .addComponent(jLabel7))
                .addGap(18, 18, 18)
                .addGroup(jPanel1Layout.createParallelGroup(javax.swing.GroupLayout.Alignment.BASELINE)
                    .addComponent(lokasiCBInternal)
                    .addComponent(nilaiLokasiInternal, javax.swing.GroupLayout.PREFERRED_SIZE, javax.swing.GroupLayout.DEFAULT_SIZE, javax.swing.GroupLayout.PREFERRED_SIZE)
                    .addComponent(jLabel8))
                .addGap(18, 18, 18)
                .addGroup(jPanel1Layout.createParallelGroup(javax.swing.GroupLayout.Alignment.BASELINE)
                    .addComponent(bUsahaCBInternal)
                    .addComponent(nilaiUsahaInternal, javax.swing.GroupLayout.PREFERRED_SIZE, javax.swing.GroupLayout.DEFAULT_SIZE, javax.swing.GroupLayout.PREFERRED_SIZE)
                    .addComponent(jLabel9))
                .addPreferredGap(javax.swing.LayoutStyle.ComponentPlacement.RELATED, 26, Short.MAX_VALUE)
                .addComponent(nextButton)
                .addGap(21, 21, 21))
        );

        javax.swing.GroupLayout layout = new javax.swing.GroupLayout(getContentPane());
        getContentPane().setLayout(layout);
        layout.setHorizontalGroup(
            layout.createParallelGroup(javax.swing.GroupLayout.Alignment.LEADING)
            .addGroup(layout.createSequentialGroup()
                .addGap(21, 21, 21)
                .addComponent(jPanel1, javax.swing.GroupLayout.PREFERRED_SIZE, javax.swing.GroupLayout.DEFAULT_SIZE, javax.swing.GroupLayout.PREFERRED_SIZE)
                .addContainerGap(22, Short.MAX_VALUE))
        );
        layout.setVerticalGroup(
            layout.createParallelGroup(javax.swing.GroupLayout.Alignment.LEADING)
            .addGroup(layout.createSequentialGroup()
                .addContainerGap()
                .addComponent(jPanel1, javax.swing.GroupLayout.DEFAULT_SIZE, javax.swing.GroupLayout.DEFAULT_SIZE, Short.MAX_VALUE)
                .addContainerGap())
        );

        pack();
    }// </editor-fold>                        

    private void nilaiPendapatanInternalActionPerformed(java.awt.event.ActionEvent evt) {                                                        
        // TODO add your handling code here:
    }                                                       

    private void nilaiLevelInternalActionPerformed(java.awt.event.ActionEvent evt) {                                                   
        // TODO add your handling code here:
    }                                                  

    private void pendapatanCBInternalActionPerformed(java.awt.event.ActionEvent evt) {                                                     
        // TODO add your handling code here:
    }                                                    

    private void nilaiLokasiInternalActionPerformed(java.awt.event.ActionEvent evt) {                                                    
        // TODO add your handling code here:
    }                                                   

    private void nextButtonActionPerformed(java.awt.event.ActionEvent evt) {                                           


    }                                          

    private void umurCBInternalActionPerformed(java.awt.event.ActionEvent evt) {                                               

    }                                              

    private void nilaiUmurInternalActionPerformed(java.awt.event.ActionEvent evt) {                                                  

    }                                                 

    private void nextButtonMouseClicked(java.awt.event.MouseEvent evt) {                                        
        double nilaiUmur = 0.0;
        double nilaiLevel = 0.0;
        double nilaiPendidikan = 0.0;
        double nilaiPendapatan = 0.0;
        double nilaiLokasi = 0.0;
        double nilaiUsaha = 0.0;
        double nilaiJenisKelamin = 0.0;
        boolean checker = true;
        if (umurCBInternal.isSelected()) {
            InputDataHandler.jmlChecklist();
            if (nilaiUmurInternal.getText().equals("")) {
                InputDataHandler.inputDataInternal("umurInternal", null);
                checker = false; // false karena nilainya null
            } else {
                nilaiUmur = Double.parseDouble(nilaiUmurInternal.getText()) / 100.0;
                String nilaiU = Double.toString(nilaiUmur);
//                System.out.println(nilaiU);
                InputDataHandler.inputDataInternal("umurInternal", nilaiU);
            }

        }

        if (levelCBInternal.isSelected()) {
            InputDataHandler.jmlChecklist();
            if (nilaiLevelInternal.getText().equals("")) {
                InputDataHandler.inputDataInternal("levelInternal", null);
                checker = false; // false karena nilainya null
            } else {
                nilaiLevel = Double.parseDouble(nilaiLevelInternal.getText()) / 100.0;
                String nilaiL = Double.toString(nilaiLevel);
//                System.out.println(nilaiL);
                InputDataHandler.inputDataInternal("levelInternal", nilaiL);
            }
        }

        if (pendidikanCBInternal.isSelected()) {
            InputDataHandler.jmlChecklist();
            if (nilaiPendidikanInternal.getText().equals("")) {
                InputDataHandler.inputDataInternal("pendidikanInternal", null);
                checker = false; // false karena nilainya null
            } else {
                nilaiPendidikan = Double.parseDouble(nilaiPendidikanInternal.getText()) / 100.0;
                String nilaiPendi = Double.toString(nilaiPendidikan);
//                System.out.println(nilaiPendi);
                InputDataHandler.inputDataInternal("pendidikanInternal", nilaiPendi);
            }
        }

        if (pendapatanCBInternal.isSelected()) {
            InputDataHandler.jmlChecklist();
            if (nilaiPendapatanInternal.getText().equals("")) {
                InputDataHandler.inputDataInternal("pendapatanInternal", null);
                checker = false; // false karena nilainya null
            } else {
                nilaiPendapatan = Double.parseDouble(nilaiPendapatanInternal.getText()) / 100.0;
                String nilaiPenda= Double.toString(nilaiPendapatan);
//                System.out.println(nilaiU);
                InputDataHandler.inputDataInternal("pendapatanInternal", nilaiPenda);
            }
        }

        if (jenisKelaminCBInternal.isSelected()) {
            InputDataHandler.jmlChecklist();
            if (nilaiJenisKelaminInternal.getText().equals("")) {
                InputDataHandler.inputDataInternal("jenisKelaminInternal", null);
                checker = false; // false karena nilainya null
            } else {
                nilaiJenisKelamin = Double.parseDouble(nilaiJenisKelaminInternal.getText()) / 100.0;
                String nilaiJK = Double.toString(nilaiJenisKelamin);
//                System.out.println(nilaiJK);
                InputDataHandler.inputDataInternal("jenisKelaminInternal", nilaiJK);
            }
        }

        if (lokasiCBInternal.isSelected()) {
            InputDataHandler.jmlChecklist();
            if (nilaiLokasiInternal.getText().equals("")) {
                InputDataHandler.inputDataInternal("lokasiInternal", null);
                checker = false; // false karena nilainya null
            } else {
                nilaiLokasi = Double.parseDouble(nilaiLokasiInternal.getText()) / 100.0;
                String nilaiL = Double.toString(nilaiLokasi);
//                System.out.println(nilaiU);
                InputDataHandler.inputDataInternal("lokasiInternal", nilaiL);
            }
        }

        if (bUsahaCBInternal.isSelected()) {
            InputDataHandler.jmlChecklist();
            if (nilaiUsahaInternal.getText().equals("")) {
                InputDataHandler.inputDataInternal("usahaInternal", null);
                checker = false; // false karena nilainya null
            } else {
                nilaiUsaha = Double.parseDouble(nilaiUsahaInternal.getText()) / 100.0;
                String nilaiUs = Double.toString(nilaiUsaha);
//                System.out.println(nilaiU);
                InputDataHandler.inputDataInternal("usahaInternal", nilaiUs);
            }
        }
        if (InputDataHandler.getKetetanggaan() == 0) {
            checker = false;
        }
        double umur = 0.0;
        double pendidikan = 0.0;
        double level = 0.0;
        double pendapatan = 0.0;
        double jenisKelamin = 0.0;
        double lokasi = 0.0;
        double usaha = 0.0;

        double[] kumpulanBobot = new double[InputDataHandler.getKetetanggaan()];
        int m = 0;
        // kalau umur dichecklist, dimasukkan ke variabel umur
        if (InputDataHandler.checkKey("umurInternal")) {
            umur = Double.parseDouble(InputDataHandler.getValue("umurInternal"));
            kumpulanBobot[m] = umur;
            m++;
        }
        if (InputDataHandler.checkKey("pendidikanInternal")) {
            pendidikan = Double.parseDouble(InputDataHandler.getValue("pendidikanInternal"));
            kumpulanBobot[m] = pendidikan;
            m++;
        }
        if (InputDataHandler.checkKey("pendapatanInternal")) {
            pendapatan = Double.parseDouble(InputDataHandler.getValue("pendapatanInternal"));
            kumpulanBobot[m] = pendapatan;
            m++;
        }
        if (InputDataHandler.checkKey("levelInternal")) {
            level = Double.parseDouble(InputDataHandler.getValue("levelInternal"));
            kumpulanBobot[m] = level;
            m++;
        }
        if (InputDataHandler.checkKey("jenisKelaminInternal")) {
            jenisKelamin = Double.parseDouble(InputDataHandler.getValue("jenisKelaminInternal"));
            kumpulanBobot[m] = jenisKelamin;
            m++;
        }
        if (InputDataHandler.checkKey("lokasiInternal")) {
            lokasi = Double.parseDouble(InputDataHandler.getValue("lokasiInternal"));
            kumpulanBobot[m] = lokasi;
            m++;
        }
        if (InputDataHandler.checkKey("usahaInternal")) {
            usaha = Double.parseDouble(InputDataHandler.getValue("usahaInternal"));
            kumpulanBobot[m] = usaha;
            m++;
        }
        
        int totalNilai=0;
        for (int i = 0; i < kumpulanBobot.length; i++) {
            totalNilai+=kumpulanBobot[i]*100;

        }
        if (totalNilai != 100) {
            JOptionPane.showMessageDialog(null, "The sum of text fields must 100%!");
            checker = false;
        }
        InputDataHandler.setBobot(kumpulanBobot);
        
        
        
        if (checker == true) {
            this.hide();
            TampilanKondisiKetetanggaan kk = new TampilanKondisiKetetanggaan();
            kk.setVisible(true);
        } else {
            JOptionPane.showMessageDialog(null, "You cannot move to the other page because you must fill text field first!");
        }

    }                                       

    private void umurCBInternalMouseClicked(java.awt.event.MouseEvent evt) {                                            
        if (umurCBInternal.isSelected()) {
            nilaiUmurInternal.setEnabled(true);
        } else {
            nilaiUmurInternal.setEnabled(false);
        }
    }                                           

    private void levelCBInternalMouseClicked(java.awt.event.MouseEvent evt) {                                             
        if (levelCBInternal.isSelected()) {
            nilaiLevelInternal.setEnabled(true);
        } else {
            nilaiLevelInternal.setEnabled(false);
        }
    }                                            

    private void pendidikanCBInternalMouseClicked(java.awt.event.MouseEvent evt) {                                                  

        if (pendidikanCBInternal.isSelected()) {
            nilaiPendidikanInternal.setEnabled(true);
        } else {
            nilaiPendidikanInternal.setEnabled(false);
        }
    }                                                 

    private void pendapatanCBInternalMouseClicked(java.awt.event.MouseEvent evt) {                                                  

        if (pendapatanCBInternal.isSelected()) {
            nilaiPendapatanInternal.setEnabled(true);
        } else {
            nilaiPendapatanInternal.setEnabled(false);
        }
    }                                                 

    private void jenisKelaminCBInternalMouseClicked(java.awt.event.MouseEvent evt) {                                                    
        if (jenisKelaminCBInternal.isSelected()) {
            nilaiJenisKelaminInternal.setEnabled(true);
        } else {
            nilaiJenisKelaminInternal.setEnabled(false);
        }
    }                                                   

    private void lokasiCBInternalMouseClicked(java.awt.event.MouseEvent evt) {                                              
        if (lokasiCBInternal.isSelected()) {
            nilaiLokasiInternal.setEnabled(true);
        } else {
            nilaiLokasiInternal.setEnabled(false);
        }

    }                                             

    private void bUsahaCBInternalMouseClicked(java.awt.event.MouseEvent evt) {                                              
        if (bUsahaCBInternal.isSelected()) {
            nilaiUsahaInternal.setEnabled(true);
        } else {
            nilaiUsahaInternal.setEnabled(false);
        }
    }                                             

    private void nilaiUmurInternalInputMethodTextChanged(java.awt.event.InputMethodEvent evt) {                                                         

    }                                                        

    private void nilaiUmurInternalMouseClicked(java.awt.event.MouseEvent evt) {                                               

    }                                              

    private void nilaiUmurInternalComponentAdded(java.awt.event.ContainerEvent evt) {                                                 

    }                                                

    /**
     * @param args the command line arguments
     */
    public static void main(String args[]) {
        /* Set the Nimbus look and feel */
        //<editor-fold defaultstate="collapsed" desc=" Look and feel setting code (optional) ">
        /* If Nimbus (introduced in Java SE 6) is not available, stay with the default look and feel.
         * For details see http://download.oracle.com/javase/tutorial/uiswing/lookandfeel/plaf.html 
         */
        try {
            for (javax.swing.UIManager.LookAndFeelInfo info : javax.swing.UIManager.getInstalledLookAndFeels()) {
                if ("Nimbus".equals(info.getName())) {
                    javax.swing.UIManager.setLookAndFeel(info.getClassName());
                    break;
                }
            }
        } catch (ClassNotFoundException ex) {
            java.util.logging.Logger.getLogger(TampilanBobotKetetanggaan.class.getName()).log(java.util.logging.Level.SEVERE, null, ex);
        } catch (InstantiationException ex) {
            java.util.logging.Logger.getLogger(TampilanBobotKetetanggaan.class.getName()).log(java.util.logging.Level.SEVERE, null, ex);
        } catch (IllegalAccessException ex) {
            java.util.logging.Logger.getLogger(TampilanBobotKetetanggaan.class.getName()).log(java.util.logging.Level.SEVERE, null, ex);
        } catch (javax.swing.UnsupportedLookAndFeelException ex) {
            java.util.logging.Logger.getLogger(TampilanBobotKetetanggaan.class.getName()).log(java.util.logging.Level.SEVERE, null, ex);
        }
        //</editor-fold>
        //</editor-fold>
        //</editor-fold>
        //</editor-fold>

        /* Create and display the form */
        java.awt.EventQueue.invokeLater(new Runnable() {
            public void run() {
                new TampilanBobotKetetanggaan().setVisible(true);
            }
        });
    }

    // Variables declaration - do not modify                     
    private javax.swing.JCheckBox bUsahaCBInternal;
    private javax.swing.JLabel jLabel1;
    private javax.swing.JLabel jLabel2;
    private javax.swing.JLabel jLabel3;
    private javax.swing.JLabel jLabel4;
    private javax.swing.JLabel jLabel5;
    private javax.swing.JLabel jLabel6;
    private javax.swing.JLabel jLabel7;
    private javax.swing.JLabel jLabel8;
    private javax.swing.JLabel jLabel9;
    private javax.swing.JPanel jPanel1;
    private javax.swing.JCheckBox jenisKelaminCBInternal;
    private javax.swing.JCheckBox levelCBInternal;
    private javax.swing.JCheckBox lokasiCBInternal;
    public javax.swing.JButton nextButton;
    private javax.swing.JTextField nilaiJenisKelaminInternal;
    private javax.swing.JTextField nilaiLevelInternal;
    private javax.swing.JTextField nilaiLokasiInternal;
    private javax.swing.JTextField nilaiPendapatanInternal;
    private javax.swing.JTextField nilaiPendidikanInternal;
    private javax.swing.JTextField nilaiUmurInternal;
    private javax.swing.JTextField nilaiUsahaInternal;
    private javax.swing.JCheckBox pendapatanCBInternal;
    private javax.swing.JCheckBox pendidikanCBInternal;
    private javax.swing.JCheckBox umurCBInternal;
    // End of variables declaration                   
}

\end{lstlisting}

\begin{lstlisting}[language=Java, caption=TampilanKondisiKetetanggaan.java]
/*
 * To change this license header, choose License Headers in Project Properties.
 * To change this template file, choose Tools | Templates
 * and open the template in the editor.
 */
package ecasimulatorjframe;

import javax.swing.ButtonGroup;
import javax.swing.JOptionPane;
import javax.swing.JPanel;

/**
 *
 * @author Vanessa
 */
public class TampilanKondisiKetetanggaan extends javax.swing.JFrame {

    /**
     * Creates new form TampilanKondisiKetetanggaan
     */
    JPanel[] kumpulanJPanel;

    //int jmlCheckListInternal = 0;
    //double[] bobot;
    public TampilanKondisiKetetanggaan() {
        initComponents();

        kumpulanJPanel = new JPanel[]{jUmur, jLevel, jPendidikan, jPendapatan, jJenisKelamin, jLokasi, jbidangUsaha};
        for (int i = 0; i < kumpulanJPanel.length; i++) {
            kumpulanJPanel[i].setVisible(false);
            kumpulanJPanel[i].setLocation(29, 103); //ditumpuk di jUmur
        }
        int i = 0;
        if (InputDataHandler.checkKey("umurInternal")) {
            kumpulanJPanel[0].setVisible(true);
        }
        if (InputDataHandler.checkKey("levelInternal")) {
            kumpulanJPanel[1].setVisible(true);
        }
        if (InputDataHandler.checkKey("pendidikanInternal")) {
            kumpulanJPanel[2].setVisible(true);
        }
        if (InputDataHandler.checkKey("pendapatanInternal")) {
            kumpulanJPanel[3].setVisible(true);
        }
        if (InputDataHandler.checkKey("jenisKelaminInternal")) {
            kumpulanJPanel[4].setVisible(true);
        }
        if (InputDataHandler.checkKey("lokasiInternal")) {
            kumpulanJPanel[5].setVisible(true);
        }
        if (InputDataHandler.checkKey("usahaInternal")) {
            kumpulanJPanel[6].setVisible(true);
        }
        
        ButtonGroup group1 = new ButtonGroup();
        group1.add(umurLbhDr);
        group1.add(umurSmDgn);
        group1.add(umurKrgDr);
        
        ButtonGroup group2 = new ButtonGroup();
        group2.add(levelLbhDr);
        group2.add(levelSmDgn);
        group2.add(levelKrgDr);
        
        ButtonGroup group3 = new ButtonGroup();
        group3.add(pendapatanLbhDr);
        group3.add(pendapatanSmDgn);
        group3.add(pendapatanKrgDr);
        
        ButtonGroup group4 = new ButtonGroup();
        group4.add(pendidikanLbhDr);
        group4.add(pendidikanSmDgn);
        group4.add(pendidikanKrgDr);

    }

    /**
     * This method is called from within the constructor to initialize the form.
     * WARNING: Do NOT modify this code. The content of this method is always
     * regenerated by the Form Editor.
     */
    @SuppressWarnings("unchecked")
    // <editor-fold defaultstate="collapsed" desc="Generated Code">                          
    private void initComponents() {

        jPanel1 = new javax.swing.JPanel();
        jLabel1 = new javax.swing.JLabel();
        jLabel2 = new javax.swing.JLabel();
        jLabel3 = new javax.swing.JLabel();
        jUmur = new javax.swing.JPanel();
        umurKrgDr = new javax.swing.JRadioButton();
        umurSmDgn = new javax.swing.JRadioButton();
        umurLbhDr = new javax.swing.JRadioButton();
        jLabel4 = new javax.swing.JLabel();
        jLevel = new javax.swing.JPanel();
        levelKrgDr = new javax.swing.JRadioButton();
        levelSmDgn = new javax.swing.JRadioButton();
        levelLbhDr = new javax.swing.JRadioButton();
        jLabel5 = new javax.swing.JLabel();
        jPendidikan = new javax.swing.JPanel();
        pendidikanKrgDr = new javax.swing.JRadioButton();
        pendidikanSmDgn = new javax.swing.JRadioButton();
        pendidikanLbhDr = new javax.swing.JRadioButton();
        jLabel6 = new javax.swing.JLabel();
        jPendapatan = new javax.swing.JPanel();
        pendapatanKrgDr = new javax.swing.JRadioButton();
        pendapatanSmDgn = new javax.swing.JRadioButton();
        pendapatanLbhDr = new javax.swing.JRadioButton();
        jLabel7 = new javax.swing.JLabel();
        jPanel2 = new javax.swing.JPanel();
        jJenisKelamin = new javax.swing.JPanel();
        jLabel8 = new javax.swing.JLabel();
        jbidangUsaha = new javax.swing.JPanel();
        jLabel10 = new javax.swing.JLabel();
        jLokasi = new javax.swing.JPanel();
        jLabel9 = new javax.swing.JLabel();
        jPendapatan2 = new javax.swing.JPanel();
        pendapatanCBNeg2 = new javax.swing.JCheckBox();
        pendapatanKrgDr2 = new javax.swing.JRadioButton();
        pendapatanSmDgn2 = new javax.swing.JRadioButton();
        pendapatanLbhDr2 = new javax.swing.JRadioButton();
        jPanel4 = new javax.swing.JPanel();
        backButton = new javax.swing.JButton();
        nextButton = new javax.swing.JButton();

        setDefaultCloseOperation(javax.swing.WindowConstants.EXIT_ON_CLOSE);

        jLabel1.setFont(new java.awt.Font("Tahoma", 1, 14)); // NOI18N
        jLabel1.setText("SIMULATOR ECA");

        jLabel2.setText("Parameter Setting Kondisi Ketetanggaan :");

        jLabel3.setText("Berdasarkan Relasi :");

        jUmur.setBorder(javax.swing.BorderFactory.createLineBorder(new java.awt.Color(0, 0, 0)));

        umurKrgDr.setText("<=");

        umurSmDgn.setText("=");

        umurLbhDr.setText(">=");

        jLabel4.setText("UMUR");

        javax.swing.GroupLayout jUmurLayout = new javax.swing.GroupLayout(jUmur);
        jUmur.setLayout(jUmurLayout);
        jUmurLayout.setHorizontalGroup(
            jUmurLayout.createParallelGroup(javax.swing.GroupLayout.Alignment.LEADING)
            .addGroup(jUmurLayout.createSequentialGroup()
                .addContainerGap()
                .addComponent(jLabel4)
                .addPreferredGap(javax.swing.LayoutStyle.ComponentPlacement.RELATED, javax.swing.GroupLayout.DEFAULT_SIZE, Short.MAX_VALUE)
                .addComponent(umurKrgDr)
                .addGap(26, 26, 26)
                .addComponent(umurSmDgn)
                .addGap(18, 18, 18)
                .addComponent(umurLbhDr)
                .addContainerGap())
        );
        jUmurLayout.setVerticalGroup(
            jUmurLayout.createParallelGroup(javax.swing.GroupLayout.Alignment.LEADING)
            .addGroup(jUmurLayout.createSequentialGroup()
                .addContainerGap()
                .addGroup(jUmurLayout.createParallelGroup(javax.swing.GroupLayout.Alignment.BASELINE)
                    .addComponent(umurKrgDr)
                    .addComponent(umurSmDgn)
                    .addComponent(umurLbhDr)
                    .addComponent(jLabel4))
                .addContainerGap(javax.swing.GroupLayout.DEFAULT_SIZE, Short.MAX_VALUE))
        );

        jLevel.setBorder(javax.swing.BorderFactory.createLineBorder(new java.awt.Color(0, 0, 0)));

        levelKrgDr.setText("<=");

        levelSmDgn.setText("=");
        levelSmDgn.addActionListener(new java.awt.event.ActionListener() {
            public void actionPerformed(java.awt.event.ActionEvent evt) {
                levelSmDgnActionPerformed(evt);
            }
        });

        levelLbhDr.setText(">=");
        levelLbhDr.addActionListener(new java.awt.event.ActionListener() {
            public void actionPerformed(java.awt.event.ActionEvent evt) {
                levelLbhDrActionPerformed(evt);
            }
        });

        jLabel5.setText("LEVEL");

        javax.swing.GroupLayout jLevelLayout = new javax.swing.GroupLayout(jLevel);
        jLevel.setLayout(jLevelLayout);
        jLevelLayout.setHorizontalGroup(
            jLevelLayout.createParallelGroup(javax.swing.GroupLayout.Alignment.LEADING)
            .addGroup(jLevelLayout.createSequentialGroup()
                .addContainerGap()
                .addComponent(jLabel5)
                .addPreferredGap(javax.swing.LayoutStyle.ComponentPlacement.RELATED, javax.swing.GroupLayout.DEFAULT_SIZE, Short.MAX_VALUE)
                .addComponent(levelKrgDr)
                .addGap(26, 26, 26)
                .addComponent(levelSmDgn)
                .addGap(18, 18, 18)
                .addComponent(levelLbhDr)
                .addContainerGap())
        );
        jLevelLayout.setVerticalGroup(
            jLevelLayout.createParallelGroup(javax.swing.GroupLayout.Alignment.LEADING)
            .addGroup(javax.swing.GroupLayout.Alignment.TRAILING, jLevelLayout.createSequentialGroup()
                .addGap(11, 11, 11)
                .addGroup(jLevelLayout.createParallelGroup(javax.swing.GroupLayout.Alignment.BASELINE)
                    .addComponent(levelKrgDr)
                    .addComponent(levelSmDgn)
                    .addComponent(levelLbhDr, javax.swing.GroupLayout.DEFAULT_SIZE, javax.swing.GroupLayout.DEFAULT_SIZE, Short.MAX_VALUE)
                    .addComponent(jLabel5))
                .addContainerGap())
        );

        jPendidikan.setBorder(javax.swing.BorderFactory.createLineBorder(new java.awt.Color(0, 0, 0)));

        pendidikanKrgDr.setText("<=");

        pendidikanSmDgn.setText("=");

        pendidikanLbhDr.setText(">=");

        jLabel6.setText("PENDIDIKAN");

        javax.swing.GroupLayout jPendidikanLayout = new javax.swing.GroupLayout(jPendidikan);
        jPendidikan.setLayout(jPendidikanLayout);
        jPendidikanLayout.setHorizontalGroup(
            jPendidikanLayout.createParallelGroup(javax.swing.GroupLayout.Alignment.LEADING)
            .addGroup(jPendidikanLayout.createSequentialGroup()
                .addContainerGap()
                .addComponent(jLabel6)
                .addGap(145, 145, 145)
                .addComponent(pendidikanKrgDr)
                .addGap(26, 26, 26)
                .addComponent(pendidikanSmDgn)
                .addGap(18, 18, Short.MAX_VALUE)
                .addComponent(pendidikanLbhDr)
                .addContainerGap())
        );
        jPendidikanLayout.setVerticalGroup(
            jPendidikanLayout.createParallelGroup(javax.swing.GroupLayout.Alignment.LEADING)
            .addGroup(jPendidikanLayout.createSequentialGroup()
                .addContainerGap()
                .addGroup(jPendidikanLayout.createParallelGroup(javax.swing.GroupLayout.Alignment.BASELINE)
                    .addComponent(pendidikanKrgDr)
                    .addComponent(pendidikanSmDgn)
                    .addComponent(pendidikanLbhDr)
                    .addComponent(jLabel6))
                .addContainerGap(javax.swing.GroupLayout.DEFAULT_SIZE, Short.MAX_VALUE))
        );

        jPendapatan.setBorder(javax.swing.BorderFactory.createLineBorder(new java.awt.Color(0, 0, 0)));

        pendapatanKrgDr.setText("<=");

        pendapatanSmDgn.setText("=");

        pendapatanLbhDr.setText(">=");

        jLabel7.setText("PENDAPATAN");

        javax.swing.GroupLayout jPendapatanLayout = new javax.swing.GroupLayout(jPendapatan);
        jPendapatan.setLayout(jPendapatanLayout);
        jPendapatanLayout.setHorizontalGroup(
            jPendapatanLayout.createParallelGroup(javax.swing.GroupLayout.Alignment.LEADING)
            .addGroup(jPendapatanLayout.createSequentialGroup()
                .addContainerGap()
                .addComponent(jLabel7)
                .addGap(139, 139, 139)
                .addComponent(pendapatanKrgDr)
                .addGap(27, 27, 27)
                .addComponent(pendapatanSmDgn)
                .addGap(18, 18, 18)
                .addComponent(pendapatanLbhDr, javax.swing.GroupLayout.DEFAULT_SIZE, javax.swing.GroupLayout.DEFAULT_SIZE, Short.MAX_VALUE)
                .addContainerGap())
        );
        jPendapatanLayout.setVerticalGroup(
            jPendapatanLayout.createParallelGroup(javax.swing.GroupLayout.Alignment.LEADING)
            .addGroup(javax.swing.GroupLayout.Alignment.TRAILING, jPendapatanLayout.createSequentialGroup()
                .addContainerGap(javax.swing.GroupLayout.DEFAULT_SIZE, Short.MAX_VALUE)
                .addGroup(jPendapatanLayout.createParallelGroup(javax.swing.GroupLayout.Alignment.BASELINE)
                    .addComponent(pendapatanKrgDr)
                    .addComponent(pendapatanSmDgn)
                    .addComponent(pendapatanLbhDr)
                    .addComponent(jLabel7))
                .addContainerGap())
        );

        jJenisKelamin.setBorder(javax.swing.BorderFactory.createLineBorder(new java.awt.Color(0, 0, 0)));

        jLabel8.setText("JENIS KELAMIN");

        javax.swing.GroupLayout jJenisKelaminLayout = new javax.swing.GroupLayout(jJenisKelamin);
        jJenisKelamin.setLayout(jJenisKelaminLayout);
        jJenisKelaminLayout.setHorizontalGroup(
            jJenisKelaminLayout.createParallelGroup(javax.swing.GroupLayout.Alignment.LEADING)
            .addGroup(jJenisKelaminLayout.createSequentialGroup()
                .addContainerGap()
                .addComponent(jLabel8)
                .addContainerGap(283, Short.MAX_VALUE))
        );
        jJenisKelaminLayout.setVerticalGroup(
            jJenisKelaminLayout.createParallelGroup(javax.swing.GroupLayout.Alignment.LEADING)
            .addGroup(jJenisKelaminLayout.createSequentialGroup()
                .addContainerGap()
                .addComponent(jLabel8)
                .addContainerGap(javax.swing.GroupLayout.DEFAULT_SIZE, Short.MAX_VALUE))
        );

        jbidangUsaha.setBorder(javax.swing.BorderFactory.createLineBorder(new java.awt.Color(0, 0, 0)));

        jLabel10.setText("BIDANG USAHA");

        javax.swing.GroupLayout jbidangUsahaLayout = new javax.swing.GroupLayout(jbidangUsaha);
        jbidangUsaha.setLayout(jbidangUsahaLayout);
        jbidangUsahaLayout.setHorizontalGroup(
            jbidangUsahaLayout.createParallelGroup(javax.swing.GroupLayout.Alignment.LEADING)
            .addGroup(jbidangUsahaLayout.createSequentialGroup()
                .addContainerGap()
                .addComponent(jLabel10)
                .addContainerGap(javax.swing.GroupLayout.DEFAULT_SIZE, Short.MAX_VALUE))
        );
        jbidangUsahaLayout.setVerticalGroup(
            jbidangUsahaLayout.createParallelGroup(javax.swing.GroupLayout.Alignment.LEADING)
            .addGroup(javax.swing.GroupLayout.Alignment.TRAILING, jbidangUsahaLayout.createSequentialGroup()
                .addContainerGap(javax.swing.GroupLayout.DEFAULT_SIZE, Short.MAX_VALUE)
                .addComponent(jLabel10)
                .addContainerGap())
        );

        jLokasi.setBorder(javax.swing.BorderFactory.createLineBorder(new java.awt.Color(0, 0, 0)));

        jLabel9.setText("LOKASI");

        javax.swing.GroupLayout jLokasiLayout = new javax.swing.GroupLayout(jLokasi);
        jLokasi.setLayout(jLokasiLayout);
        jLokasiLayout.setHorizontalGroup(
            jLokasiLayout.createParallelGroup(javax.swing.GroupLayout.Alignment.LEADING)
            .addGroup(jLokasiLayout.createSequentialGroup()
                .addContainerGap()
                .addComponent(jLabel9)
                .addContainerGap(javax.swing.GroupLayout.DEFAULT_SIZE, Short.MAX_VALUE))
        );
        jLokasiLayout.setVerticalGroup(
            jLokasiLayout.createParallelGroup(javax.swing.GroupLayout.Alignment.LEADING)
            .addGroup(jLokasiLayout.createSequentialGroup()
                .addContainerGap()
                .addComponent(jLabel9)
                .addContainerGap(javax.swing.GroupLayout.DEFAULT_SIZE, Short.MAX_VALUE))
        );

        javax.swing.GroupLayout jPanel2Layout = new javax.swing.GroupLayout(jPanel2);
        jPanel2.setLayout(jPanel2Layout);
        jPanel2Layout.setHorizontalGroup(
            jPanel2Layout.createParallelGroup(javax.swing.GroupLayout.Alignment.LEADING)
            .addGroup(jPanel2Layout.createSequentialGroup()
                .addGroup(jPanel2Layout.createParallelGroup(javax.swing.GroupLayout.Alignment.LEADING, false)
                    .addComponent(jLokasi, javax.swing.GroupLayout.DEFAULT_SIZE, javax.swing.GroupLayout.DEFAULT_SIZE, Short.MAX_VALUE)
                    .addComponent(jbidangUsaha, javax.swing.GroupLayout.DEFAULT_SIZE, javax.swing.GroupLayout.DEFAULT_SIZE, Short.MAX_VALUE)
                    .addComponent(jJenisKelamin, javax.swing.GroupLayout.PREFERRED_SIZE, javax.swing.GroupLayout.DEFAULT_SIZE, javax.swing.GroupLayout.PREFERRED_SIZE))
                .addGap(0, 83, Short.MAX_VALUE))
        );
        jPanel2Layout.setVerticalGroup(
            jPanel2Layout.createParallelGroup(javax.swing.GroupLayout.Alignment.LEADING)
            .addGroup(jPanel2Layout.createSequentialGroup()
                .addGap(7, 7, 7)
                .addComponent(jJenisKelamin, javax.swing.GroupLayout.PREFERRED_SIZE, javax.swing.GroupLayout.DEFAULT_SIZE, javax.swing.GroupLayout.PREFERRED_SIZE)
                .addPreferredGap(javax.swing.LayoutStyle.ComponentPlacement.UNRELATED)
                .addComponent(jLokasi, javax.swing.GroupLayout.PREFERRED_SIZE, javax.swing.GroupLayout.DEFAULT_SIZE, javax.swing.GroupLayout.PREFERRED_SIZE)
                .addPreferredGap(javax.swing.LayoutStyle.ComponentPlacement.UNRELATED)
                .addComponent(jbidangUsaha, javax.swing.GroupLayout.PREFERRED_SIZE, javax.swing.GroupLayout.DEFAULT_SIZE, javax.swing.GroupLayout.PREFERRED_SIZE)
                .addContainerGap(45, Short.MAX_VALUE))
        );

        jPendapatan2.setBorder(javax.swing.BorderFactory.createLineBorder(new java.awt.Color(0, 0, 0)));

        pendapatanCBNeg2.setText("PENDAPATAN");
        pendapatanCBNeg2.addActionListener(new java.awt.event.ActionListener() {
            public void actionPerformed(java.awt.event.ActionEvent evt) {
                pendapatanCBNeg2ActionPerformed(evt);
            }
        });

        pendapatanKrgDr2.setText("<=");

        pendapatanSmDgn2.setText("=");

        pendapatanLbhDr2.setText(">=");

        javax.swing.GroupLayout jPendapatan2Layout = new javax.swing.GroupLayout(jPendapatan2);
        jPendapatan2.setLayout(jPendapatan2Layout);
        jPendapatan2Layout.setHorizontalGroup(
            jPendapatan2Layout.createParallelGroup(javax.swing.GroupLayout.Alignment.LEADING)
            .addGroup(jPendapatan2Layout.createSequentialGroup()
                .addContainerGap()
                .addComponent(pendapatanCBNeg2)
                .addGap(92, 92, 92)
                .addComponent(pendapatanKrgDr2)
                .addGap(27, 27, 27)
                .addComponent(pendapatanSmDgn2)
                .addGap(18, 18, 18)
                .addComponent(pendapatanLbhDr2)
                .addContainerGap(javax.swing.GroupLayout.DEFAULT_SIZE, Short.MAX_VALUE))
        );
        jPendapatan2Layout.setVerticalGroup(
            jPendapatan2Layout.createParallelGroup(javax.swing.GroupLayout.Alignment.LEADING)
            .addGroup(javax.swing.GroupLayout.Alignment.TRAILING, jPendapatan2Layout.createSequentialGroup()
                .addContainerGap(javax.swing.GroupLayout.DEFAULT_SIZE, Short.MAX_VALUE)
                .addGroup(jPendapatan2Layout.createParallelGroup(javax.swing.GroupLayout.Alignment.BASELINE)
                    .addComponent(pendapatanCBNeg2)
                    .addComponent(pendapatanKrgDr2)
                    .addComponent(pendapatanSmDgn2)
                    .addComponent(pendapatanLbhDr2))
                .addContainerGap())
        );

        javax.swing.GroupLayout jPanel1Layout = new javax.swing.GroupLayout(jPanel1);
        jPanel1.setLayout(jPanel1Layout);
        jPanel1Layout.setHorizontalGroup(
            jPanel1Layout.createParallelGroup(javax.swing.GroupLayout.Alignment.LEADING)
            .addGroup(jPanel1Layout.createSequentialGroup()
                .addGroup(jPanel1Layout.createParallelGroup(javax.swing.GroupLayout.Alignment.LEADING)
                    .addGroup(jPanel1Layout.createSequentialGroup()
                        .addGap(157, 157, 157)
                        .addComponent(jLabel1))
                    .addGroup(jPanel1Layout.createSequentialGroup()
                        .addContainerGap()
                        .addComponent(jLabel2))
                    .addGroup(jPanel1Layout.createSequentialGroup()
                        .addGap(29, 29, 29)
                        .addGroup(jPanel1Layout.createParallelGroup(javax.swing.GroupLayout.Alignment.LEADING)
                            .addComponent(jLabel3)
                            .addComponent(jPanel2, javax.swing.GroupLayout.PREFERRED_SIZE, javax.swing.GroupLayout.DEFAULT_SIZE, javax.swing.GroupLayout.PREFERRED_SIZE)
                            .addGroup(jPanel1Layout.createParallelGroup(javax.swing.GroupLayout.Alignment.TRAILING, false)
                                .addComponent(jPendapatan, javax.swing.GroupLayout.Alignment.LEADING, javax.swing.GroupLayout.DEFAULT_SIZE, javax.swing.GroupLayout.DEFAULT_SIZE, Short.MAX_VALUE)
                                .addComponent(jUmur, javax.swing.GroupLayout.Alignment.LEADING, javax.swing.GroupLayout.DEFAULT_SIZE, javax.swing.GroupLayout.DEFAULT_SIZE, Short.MAX_VALUE)
                                .addComponent(jLevel, javax.swing.GroupLayout.Alignment.LEADING, javax.swing.GroupLayout.DEFAULT_SIZE, javax.swing.GroupLayout.DEFAULT_SIZE, Short.MAX_VALUE)
                                .addComponent(jPendidikan, javax.swing.GroupLayout.Alignment.LEADING, javax.swing.GroupLayout.DEFAULT_SIZE, javax.swing.GroupLayout.DEFAULT_SIZE, Short.MAX_VALUE)))))
                .addContainerGap(javax.swing.GroupLayout.DEFAULT_SIZE, Short.MAX_VALUE))
            .addGroup(jPanel1Layout.createParallelGroup(javax.swing.GroupLayout.Alignment.LEADING)
                .addGroup(jPanel1Layout.createSequentialGroup()
                    .addGap(131, 131, 131)
                    .addComponent(jPendapatan2, javax.swing.GroupLayout.PREFERRED_SIZE, javax.swing.GroupLayout.DEFAULT_SIZE, javax.swing.GroupLayout.PREFERRED_SIZE)
                    .addContainerGap(javax.swing.GroupLayout.DEFAULT_SIZE, Short.MAX_VALUE)))
        );
        jPanel1Layout.setVerticalGroup(
            jPanel1Layout.createParallelGroup(javax.swing.GroupLayout.Alignment.LEADING)
            .addGroup(jPanel1Layout.createSequentialGroup()
                .addContainerGap()
                .addComponent(jLabel1)
                .addGap(30, 30, 30)
                .addComponent(jLabel2)
                .addPreferredGap(javax.swing.LayoutStyle.ComponentPlacement.RELATED)
                .addComponent(jLabel3)
                .addPreferredGap(javax.swing.LayoutStyle.ComponentPlacement.UNRELATED)
                .addComponent(jUmur, javax.swing.GroupLayout.DEFAULT_SIZE, javax.swing.GroupLayout.DEFAULT_SIZE, Short.MAX_VALUE)
                .addPreferredGap(javax.swing.LayoutStyle.ComponentPlacement.UNRELATED)
                .addComponent(jLevel, javax.swing.GroupLayout.PREFERRED_SIZE, javax.swing.GroupLayout.DEFAULT_SIZE, javax.swing.GroupLayout.PREFERRED_SIZE)
                .addPreferredGap(javax.swing.LayoutStyle.ComponentPlacement.UNRELATED)
                .addComponent(jPendidikan, javax.swing.GroupLayout.DEFAULT_SIZE, javax.swing.GroupLayout.DEFAULT_SIZE, Short.MAX_VALUE)
                .addPreferredGap(javax.swing.LayoutStyle.ComponentPlacement.UNRELATED)
                .addComponent(jPendapatan, javax.swing.GroupLayout.DEFAULT_SIZE, javax.swing.GroupLayout.DEFAULT_SIZE, Short.MAX_VALUE)
                .addPreferredGap(javax.swing.LayoutStyle.ComponentPlacement.RELATED)
                .addComponent(jPanel2, javax.swing.GroupLayout.PREFERRED_SIZE, javax.swing.GroupLayout.DEFAULT_SIZE, javax.swing.GroupLayout.PREFERRED_SIZE)
                .addGap(3821, 3821, 3821))
            .addGroup(jPanel1Layout.createParallelGroup(javax.swing.GroupLayout.Alignment.LEADING)
                .addGroup(jPanel1Layout.createSequentialGroup()
                    .addGap(2122, 2122, 2122)
                    .addComponent(jPendapatan2, javax.swing.GroupLayout.PREFERRED_SIZE, javax.swing.GroupLayout.DEFAULT_SIZE, javax.swing.GroupLayout.PREFERRED_SIZE)
                    .addContainerGap(2122, Short.MAX_VALUE)))
        );

        backButton.setText("BACK");
        backButton.addActionListener(new java.awt.event.ActionListener() {
            public void actionPerformed(java.awt.event.ActionEvent evt) {
                backButtonActionPerformed(evt);
            }
        });

        nextButton.setText("NEXT");
        nextButton.addMouseListener(new java.awt.event.MouseAdapter() {
            public void mouseClicked(java.awt.event.MouseEvent evt) {
                nextButtonMouseClicked(evt);
            }
        });
        nextButton.addActionListener(new java.awt.event.ActionListener() {
            public void actionPerformed(java.awt.event.ActionEvent evt) {
                nextButtonActionPerformed(evt);
            }
        });

        javax.swing.GroupLayout jPanel4Layout = new javax.swing.GroupLayout(jPanel4);
        jPanel4.setLayout(jPanel4Layout);
        jPanel4Layout.setHorizontalGroup(
            jPanel4Layout.createParallelGroup(javax.swing.GroupLayout.Alignment.LEADING)
            .addGroup(jPanel4Layout.createSequentialGroup()
                .addGap(20, 20, 20)
                .addComponent(backButton)
                .addPreferredGap(javax.swing.LayoutStyle.ComponentPlacement.RELATED, 212, Short.MAX_VALUE)
                .addComponent(nextButton)
                .addContainerGap())
        );
        jPanel4Layout.setVerticalGroup(
            jPanel4Layout.createParallelGroup(javax.swing.GroupLayout.Alignment.LEADING)
            .addGroup(jPanel4Layout.createSequentialGroup()
                .addContainerGap()
                .addGroup(jPanel4Layout.createParallelGroup(javax.swing.GroupLayout.Alignment.BASELINE)
                    .addComponent(backButton)
                    .addComponent(nextButton))
                .addContainerGap(20, Short.MAX_VALUE))
        );

        javax.swing.GroupLayout layout = new javax.swing.GroupLayout(getContentPane());
        getContentPane().setLayout(layout);
        layout.setHorizontalGroup(
            layout.createParallelGroup(javax.swing.GroupLayout.Alignment.LEADING)
            .addGroup(layout.createSequentialGroup()
                .addGap(28, 28, 28)
                .addComponent(jPanel1, javax.swing.GroupLayout.PREFERRED_SIZE, 419, javax.swing.GroupLayout.PREFERRED_SIZE)
                .addContainerGap(33, Short.MAX_VALUE))
            .addGroup(javax.swing.GroupLayout.Alignment.TRAILING, layout.createSequentialGroup()
                .addContainerGap(javax.swing.GroupLayout.DEFAULT_SIZE, Short.MAX_VALUE)
                .addComponent(jPanel4, javax.swing.GroupLayout.PREFERRED_SIZE, javax.swing.GroupLayout.DEFAULT_SIZE, javax.swing.GroupLayout.PREFERRED_SIZE)
                .addGap(60, 60, 60))
        );
        layout.setVerticalGroup(
            layout.createParallelGroup(javax.swing.GroupLayout.Alignment.LEADING)
            .addGroup(layout.createSequentialGroup()
                .addContainerGap()
                .addComponent(jPanel1, javax.swing.GroupLayout.PREFERRED_SIZE, 461, javax.swing.GroupLayout.PREFERRED_SIZE)
                .addGap(18, 18, 18)
                .addComponent(jPanel4, javax.swing.GroupLayout.DEFAULT_SIZE, javax.swing.GroupLayout.DEFAULT_SIZE, Short.MAX_VALUE)
                .addContainerGap())
        );

        pack();
    }// </editor-fold>                        

    private void levelLbhDrActionPerformed(java.awt.event.ActionEvent evt) {                                           
        // TODO add your handling code here:
    }                                          

    private void backButtonActionPerformed(java.awt.event.ActionEvent evt) {                                           
        // TODO add your handling code here:
        this.hide();
        TampilanBobotKetetanggaan ki = new TampilanBobotKetetanggaan();
        ki.setVisible(true);
    }                                          

    private void nextButtonActionPerformed(java.awt.event.ActionEvent evt) {                                           
        // TODO add your handling code here:

    }                                          

    private void levelSmDgnActionPerformed(java.awt.event.ActionEvent evt) {                                           
        // TODO add your handling code here:
    }                                          

    private void pendapatanCBNeg2ActionPerformed(java.awt.event.ActionEvent evt) {                                                 
        // TODO add your handling code here:
    }                                                

    private void nextButtonMouseClicked(java.awt.event.MouseEvent evt) {                                        
        boolean checker = true;
        String nilaiRB = "";

        double[] bobot = new double[InputDataHandler.getKetetanggaan()];

        // set relasi 0 kalau sama dengan
        // set relasi 1 kalau kurang dari sama dengan
        // set relasi 2 kalau lebih dari sama dengan
        int n = 0;
        int[] kumpulanNilaiRelasi = new int[InputDataHandler.getKetetanggaan()];
        
        //umur
        if (InputDataHandler.checkKey("umurInternal")) {
            if (umurLbhDr.isSelected()) {
                InputDataHandler.inputDataKetetanggaan("umurLbhDr", umurLbhDr.getText());
                nilaiRB = InputDataHandler.getValue("umurLbhDr");
                if (nilaiRB.equals(">=")) {
                    nilaiRB = "2";
                }
                kumpulanNilaiRelasi[n] = Integer.parseInt(nilaiRB);
                n++;
                checker = true;
            } else {
                if (umurSmDgn.isSelected()) {
                    InputDataHandler.inputDataKetetanggaan("umurSmDgn", umurSmDgn.getText());
                    nilaiRB = InputDataHandler.getValue("umurSmDgn");
                    if (nilaiRB.equals("=")) {
                        nilaiRB = "0";
                    }
                    kumpulanNilaiRelasi[n] = Integer.parseInt(nilaiRB);
                    n++;
                    checker = true;
                } else {
                    if (umurKrgDr.isSelected()) {
                        InputDataHandler.inputDataKetetanggaan("umurKurangDari", umurKrgDr.getText());
                        nilaiRB = InputDataHandler.getValue("umurKurangDari");
                        if (nilaiRB.equals("<=")) {
                            nilaiRB = "1";
                        }
                        kumpulanNilaiRelasi[n] = Integer.parseInt(nilaiRB);
                        n++;
                        checker = true;
                    }
                }
            }

            if (!umurLbhDr.isSelected()) {
                if (!umurSmDgn.isSelected()) {
                    if (!umurKrgDr.isSelected()) {
                        checker = false;
                    }
                }
            }
        }

        // level
        if (InputDataHandler.checkKey("levelInternal")) {
            if (levelLbhDr.isSelected()) {
                InputDataHandler.inputDataKetetanggaan("levelLbhDr", levelLbhDr.getText());
                nilaiRB = InputDataHandler.getValue("levelLbhDr");
                if (nilaiRB.equals(">=")) {
                    nilaiRB = "2";
                }
                kumpulanNilaiRelasi[n] = Integer.parseInt(nilaiRB);
                n++;
                checker = true;
            } else {
                if (levelSmDgn.isSelected()) {
                    InputDataHandler.inputDataKetetanggaan("levelSmDgn", levelSmDgn.getText());
                    nilaiRB = InputDataHandler.getValue("levelSmDgn");
                    if (nilaiRB.equals("=")) {
                        nilaiRB = "0";
                    }
                    kumpulanNilaiRelasi[n] = Integer.parseInt(nilaiRB);
                    n++;
                    checker = true;
                } else {
                    if (levelKrgDr.isSelected()) {
                        InputDataHandler.inputDataKetetanggaan("levelKurangDari", levelKrgDr.getText());
                        nilaiRB = InputDataHandler.getValue("levelKurangDari");
                        if (nilaiRB.equals("<=")) {
                            nilaiRB = "1";
                        }
                        kumpulanNilaiRelasi[n] = Integer.parseInt(nilaiRB);
                        n++;
                        checker = true;
                    }
                }
            }

            if (!levelLbhDr.isSelected()) {
                if (!levelSmDgn.isSelected()) {
                    if (!levelKrgDr.isSelected()) {
                        checker = false;
                    }
                }
            }
        }

        //pendidikan
        if (InputDataHandler.checkKey("pendidikanInternal")) {
            if (pendidikanLbhDr.isSelected()) {
                InputDataHandler.inputDataKetetanggaan("pendidikanLbhDr", pendidikanLbhDr.getText());
                nilaiRB = InputDataHandler.getValue("pendidikanLbhDr");
                if (nilaiRB.equals(">=")) {
                    nilaiRB = "2";
                }
                kumpulanNilaiRelasi[n] = Integer.parseInt(nilaiRB);
                n++;
                checker = true;
            } else {
                if (pendidikanSmDgn.isSelected()) {
                    InputDataHandler.inputDataKetetanggaan("pendidikanSmDgn", pendidikanSmDgn.getText());
                    nilaiRB = InputDataHandler.getValue("pendidikanSmDgn");
                    if (nilaiRB.equals("=")) {
                        nilaiRB = "0";
                    }
                    kumpulanNilaiRelasi[n] = Integer.parseInt(nilaiRB);
                    n++;
                    checker = true;
                } else {
                    if (pendidikanKrgDr.isSelected()) {
                        InputDataHandler.inputDataKetetanggaan("pendidikanKurangDari", pendidikanKrgDr.getText());
                        nilaiRB = InputDataHandler.getValue("pendidikanKurangDari");
                        if (nilaiRB.equals("<=")) {
                            nilaiRB = "1";
                        }
                        kumpulanNilaiRelasi[n] = Integer.parseInt(nilaiRB);
                        n++;
                        checker = true;
                    }
                }
            }
            if (!pendidikanLbhDr.isSelected()) {
                if (!pendidikanSmDgn.isSelected()) {
                    if (!pendidikanKrgDr.isSelected()) {
                        checker = false;
                    }
                }
            }
        }

        // pendapatan
        if (InputDataHandler.checkKey("pendapatanInternal")) {
            if (pendapatanLbhDr.isSelected()) {
                InputDataHandler.inputDataKetetanggaan("pendapatanLbhDr", pendapatanLbhDr.getText());
                nilaiRB = InputDataHandler.getValue("pendapatanLbhDr");
                if (nilaiRB.equals(">=")) {
                    nilaiRB = "2";
                }
                kumpulanNilaiRelasi[n] = Integer.parseInt(nilaiRB);
                n++;
                checker = true;
            } else {
                if (pendapatanSmDgn.isSelected()) {
                    InputDataHandler.inputDataKetetanggaan("pendapatanSmDgn", pendapatanSmDgn.getText());
                    nilaiRB = InputDataHandler.getValue("pendapatanSmDgn");
                    if (nilaiRB.equals("=")) {
                        nilaiRB = "0";
                    }
                    kumpulanNilaiRelasi[n] = Integer.parseInt(nilaiRB);
                    n++;
                    checker = true;
                } else {
                    if (pendapatanKrgDr.isSelected()) {
                        InputDataHandler.inputDataKetetanggaan("pendapatanKurangDari", pendapatanKrgDr.getText());
                        nilaiRB = InputDataHandler.getValue("pendapatanKurangDari");
                        if (nilaiRB.equals("<=")) {
                            nilaiRB = "1";
                        }
                        kumpulanNilaiRelasi[n] = Integer.parseInt(nilaiRB);
                        n++;
                        checker = true;
                    }
                }
            }
            if (!pendapatanLbhDr.isSelected()) {
                if (!pendapatanSmDgn.isSelected()) {
                    if (!pendapatanKrgDr.isSelected()) {
                        checker = false;
                    }
                }
            }
        }

        InputDataHandler.setRelation(kumpulanNilaiRelasi);
        if (checker == true) {
            this.hide();
            TampilanKondisiEksternal ke = new TampilanKondisiEksternal();
            ke.setVisible(true);
        } else {
            JOptionPane.showMessageDialog(null, "You cannot move to the other page because you must fill radio button first!");
        }
    }                                       

    /**
     * @param args the command line arguments
     */
    public static void main(String args[]) {
        /* Set the Nimbus look and feel */
        //<editor-fold defaultstate="collapsed" desc=" Look and feel setting code (optional) ">
        /* If Nimbus (introduced in Java SE 6) is not available, stay with the default look and feel.
         * For details see http://download.oracle.com/javase/tutorial/uiswing/lookandfeel/plaf.html 
         */
        try {
            for (javax.swing.UIManager.LookAndFeelInfo info : javax.swing.UIManager.getInstalledLookAndFeels()) {
                if ("Nimbus".equals(info.getName())) {
                    javax.swing.UIManager.setLookAndFeel(info.getClassName());
                    break;
                }
            }
        } catch (ClassNotFoundException ex) {
            java.util.logging.Logger.getLogger(TampilanKondisiKetetanggaan.class.getName()).log(java.util.logging.Level.SEVERE, null, ex);
        } catch (InstantiationException ex) {
            java.util.logging.Logger.getLogger(TampilanKondisiKetetanggaan.class.getName()).log(java.util.logging.Level.SEVERE, null, ex);
        } catch (IllegalAccessException ex) {
            java.util.logging.Logger.getLogger(TampilanKondisiKetetanggaan.class.getName()).log(java.util.logging.Level.SEVERE, null, ex);
        } catch (javax.swing.UnsupportedLookAndFeelException ex) {
            java.util.logging.Logger.getLogger(TampilanKondisiKetetanggaan.class.getName()).log(java.util.logging.Level.SEVERE, null, ex);
        }
        //</editor-fold>

        /* Create and display the form */
        java.awt.EventQueue.invokeLater(new Runnable() {
            public void run() {
                new TampilanKondisiKetetanggaan().setVisible(true);
            }
        });
    }

    // Variables declaration - do not modify                     
    private javax.swing.JButton backButton;
    private javax.swing.JPanel jJenisKelamin;
    private javax.swing.JLabel jLabel1;
    private javax.swing.JLabel jLabel10;
    private javax.swing.JLabel jLabel2;
    private javax.swing.JLabel jLabel3;
    private javax.swing.JLabel jLabel4;
    private javax.swing.JLabel jLabel5;
    private javax.swing.JLabel jLabel6;
    private javax.swing.JLabel jLabel7;
    private javax.swing.JLabel jLabel8;
    private javax.swing.JLabel jLabel9;
    private javax.swing.JPanel jLevel;
    private javax.swing.JPanel jLokasi;
    private javax.swing.JPanel jPanel1;
    private javax.swing.JPanel jPanel2;
    private javax.swing.JPanel jPanel4;
    private javax.swing.JPanel jPendapatan;
    private javax.swing.JPanel jPendapatan2;
    private javax.swing.JPanel jPendidikan;
    private javax.swing.JPanel jUmur;
    private javax.swing.JPanel jbidangUsaha;
    private javax.swing.JRadioButton levelKrgDr;
    private javax.swing.JRadioButton levelLbhDr;
    private javax.swing.JRadioButton levelSmDgn;
    public javax.swing.JButton nextButton;
    private javax.swing.JCheckBox pendapatanCBNeg2;
    private javax.swing.JRadioButton pendapatanKrgDr;
    private javax.swing.JRadioButton pendapatanKrgDr2;
    private javax.swing.JRadioButton pendapatanLbhDr;
    private javax.swing.JRadioButton pendapatanLbhDr2;
    private javax.swing.JRadioButton pendapatanSmDgn;
    private javax.swing.JRadioButton pendapatanSmDgn2;
    private javax.swing.JRadioButton pendidikanKrgDr;
    private javax.swing.JRadioButton pendidikanLbhDr;
    private javax.swing.JRadioButton pendidikanSmDgn;
    private javax.swing.JRadioButton umurKrgDr;
    private javax.swing.JRadioButton umurLbhDr;
    private javax.swing.JRadioButton umurSmDgn;
    // End of variables declaration                   
}

\end{lstlisting}

\begin{lstlisting}[language=Java, caption=TampilanKondisiEksternal.java]
/*
 * To change this license header, choose License Headers in Project Properties.
 * To change this template file, choose Tools | Templates
 * and open the template in the editor.
 */
package ecasimulatorjframe;

import javax.swing.JOptionPane;

/**
 *
 * @author Vanessa
 */
public class TampilanKondisiEksternal extends javax.swing.JFrame {

    /**
     * Creates new form TampilanKondisiEksternal
     */
    double[] bobotPF;
    PublicFactor pf;

    public TampilanKondisiEksternal() {
        initComponents();
    }

    /**
     * This method is called from within the constructor to initialize the form.
     * WARNING: Do NOT modify this code. The content of this method is always
     * regenerated by the Form Editor.
     */
    @SuppressWarnings("unchecked")
    // <editor-fold defaultstate="collapsed" desc="Generated Code">                          
    private void initComponents() {

        jPanel1 = new javax.swing.JPanel();
        jLabel1 = new javax.swing.JLabel();
        jLabel2 = new javax.swing.JLabel();
        jLabel3 = new javax.swing.JLabel();
        jLabel4 = new javax.swing.JLabel();
        jLabel5 = new javax.swing.JLabel();
        jLabel6 = new javax.swing.JLabel();
        jLabel7 = new javax.swing.JLabel();
        nilaiPP = new javax.swing.JTextField();
        nilaiDP = new javax.swing.JTextField();
        nilaiNSB = new javax.swing.JTextField();
        nilaiIFA = new javax.swing.JTextField();
        nextButton = new javax.swing.JButton();
        jLabel8 = new javax.swing.JLabel();
        jLabel9 = new javax.swing.JLabel();
        jLabel10 = new javax.swing.JLabel();
        jLabel11 = new javax.swing.JLabel();
        backButton = new javax.swing.JButton();
        jLabel12 = new javax.swing.JLabel();
        nilaiKeterbukaanPasar = new javax.swing.JTextField();
        jLabel13 = new javax.swing.JLabel();
        jLabel14 = new javax.swing.JLabel();
        jLabel15 = new javax.swing.JLabel();
        nilaiInfrastrukturKomersial = new javax.swing.JTextField();
        jLabel16 = new javax.swing.JLabel();
        jLabel17 = new javax.swing.JLabel();
        nilaiTransferPenelitian = new javax.swing.JTextField();
        jLabel18 = new javax.swing.JLabel();
        jLabel19 = new javax.swing.JLabel();
        nilaiPendidikanSMK = new javax.swing.JTextField();
        jLabel20 = new javax.swing.JLabel();
        jLabel21 = new javax.swing.JLabel();
        nilaiPendidikanSDSMP = new javax.swing.JTextField();
        jLabel22 = new javax.swing.JLabel();
        jLabel23 = new javax.swing.JLabel();
        nilaiKPPajak = new javax.swing.JTextField();
        jLabel24 = new javax.swing.JLabel();
        jLabel25 = new javax.swing.JLabel();
        nilaiKPEkonomi = new javax.swing.JTextField();
        jLabel26 = new javax.swing.JLabel();
        jLabel27 = new javax.swing.JLabel();
        nilaiKeuanganKewirausahaan = new javax.swing.JTextField();
        jLabel28 = new javax.swing.JLabel();

        setDefaultCloseOperation(javax.swing.WindowConstants.EXIT_ON_CLOSE);

        jLabel1.setFont(new java.awt.Font("Tahoma", 1, 14)); // NOI18N
        jLabel1.setText("SIMULATOR ECA");

        jLabel2.setText("Parameter Setting Kondisi Eksternal :");

        jLabel3.setText("Faktor Publik :");

        jLabel4.setText("Program Pemerintah :");

        jLabel5.setText("Dinamika Pasar :");

        jLabel6.setText("Norma, Sosial dan Budaya :");

        jLabel7.setText("Infrastruktur Fisik dan Akses Layanan :");

        nilaiPP.addActionListener(new java.awt.event.ActionListener() {
            public void actionPerformed(java.awt.event.ActionEvent evt) {
                nilaiPPActionPerformed(evt);
            }
        });

        nilaiDP.addActionListener(new java.awt.event.ActionListener() {
            public void actionPerformed(java.awt.event.ActionEvent evt) {
                nilaiDPActionPerformed(evt);
            }
        });

        nilaiNSB.addActionListener(new java.awt.event.ActionListener() {
            public void actionPerformed(java.awt.event.ActionEvent evt) {
                nilaiNSBActionPerformed(evt);
            }
        });

        nilaiIFA.addActionListener(new java.awt.event.ActionListener() {
            public void actionPerformed(java.awt.event.ActionEvent evt) {
                nilaiIFAActionPerformed(evt);
            }
        });

        nextButton.setText("NEXT");
        nextButton.addMouseListener(new java.awt.event.MouseAdapter() {
            public void mouseClicked(java.awt.event.MouseEvent evt) {
                nextButtonMouseClicked(evt);
            }
        });
        nextButton.addActionListener(new java.awt.event.ActionListener() {
            public void actionPerformed(java.awt.event.ActionEvent evt) {
                nextButtonActionPerformed(evt);
            }
        });

        jLabel8.setText("%");

        jLabel9.setText("%");

        jLabel10.setText("%");

        jLabel11.setText("%");

        backButton.setText("BACK");
        backButton.addActionListener(new java.awt.event.ActionListener() {
            public void actionPerformed(java.awt.event.ActionEvent evt) {
                backButtonActionPerformed(evt);
            }
        });

        jLabel12.setText("Keterbukaan Pasar : ");

        nilaiKeterbukaanPasar.addActionListener(new java.awt.event.ActionListener() {
            public void actionPerformed(java.awt.event.ActionEvent evt) {
                nilaiKeterbukaanPasarActionPerformed(evt);
            }
        });

        jLabel13.setText("%");

        jLabel15.setText("Infrastruktur Komersial dan Legal : ");

        nilaiInfrastrukturKomersial.addActionListener(new java.awt.event.ActionListener() {
            public void actionPerformed(java.awt.event.ActionEvent evt) {
                nilaiInfrastrukturKomersialActionPerformed(evt);
            }
        });

        jLabel16.setText("%");

        jLabel17.setText("Transfer Penelitian dan Pengembangan : ");

        jLabel18.setText("%");

        jLabel19.setText("Pendidikan Kewirausahaan pada SMK, Profesional dan Universitas : ");

        jLabel20.setText("%");

        jLabel21.setText("Pendidikan Kewirausahaan pada SD dan SMP :");

        nilaiPendidikanSDSMP.addActionListener(new java.awt.event.ActionListener() {
            public void actionPerformed(java.awt.event.ActionEvent evt) {
                nilaiPendidikanSDSMPActionPerformed(evt);
            }
        });

        jLabel22.setText("%");

        jLabel23.setText("Kebijakan Pemerintah terkait Pajak :");

        jLabel24.setText("%");

        jLabel25.setText("Kebijakan Pemerintah terkait Ekonomi :");

        jLabel26.setText("%");

        jLabel27.setText("Keuangan terkait Kewirausahaan :");

        jLabel28.setText("%");

        javax.swing.GroupLayout jPanel1Layout = new javax.swing.GroupLayout(jPanel1);
        jPanel1.setLayout(jPanel1Layout);
        jPanel1Layout.setHorizontalGroup(
            jPanel1Layout.createParallelGroup(javax.swing.GroupLayout.Alignment.LEADING)
            .addGroup(javax.swing.GroupLayout.Alignment.TRAILING, jPanel1Layout.createSequentialGroup()
                .addContainerGap(javax.swing.GroupLayout.DEFAULT_SIZE, Short.MAX_VALUE)
                .addComponent(nextButton)
                .addGap(29, 29, 29))
            .addGroup(jPanel1Layout.createSequentialGroup()
                .addGroup(jPanel1Layout.createParallelGroup(javax.swing.GroupLayout.Alignment.LEADING)
                    .addGroup(jPanel1Layout.createSequentialGroup()
                        .addGap(34, 34, 34)
                        .addGroup(jPanel1Layout.createParallelGroup(javax.swing.GroupLayout.Alignment.LEADING)
                            .addComponent(jLabel3)
                            .addGroup(jPanel1Layout.createSequentialGroup()
                                .addGroup(jPanel1Layout.createParallelGroup(javax.swing.GroupLayout.Alignment.LEADING)
                                    .addComponent(jLabel7)
                                    .addComponent(backButton))
                                .addGap(18, 18, 18)
                                .addComponent(nilaiIFA, javax.swing.GroupLayout.PREFERRED_SIZE, 48, javax.swing.GroupLayout.PREFERRED_SIZE)
                                .addPreferredGap(javax.swing.LayoutStyle.ComponentPlacement.UNRELATED)
                                .addComponent(jLabel11))
                            .addGroup(jPanel1Layout.createSequentialGroup()
                                .addComponent(jLabel6)
                                .addGap(18, 18, 18)
                                .addComponent(nilaiNSB, javax.swing.GroupLayout.PREFERRED_SIZE, 48, javax.swing.GroupLayout.PREFERRED_SIZE)
                                .addPreferredGap(javax.swing.LayoutStyle.ComponentPlacement.RELATED)
                                .addComponent(jLabel10))
                            .addGroup(jPanel1Layout.createSequentialGroup()
                                .addComponent(jLabel12)
                                .addPreferredGap(javax.swing.LayoutStyle.ComponentPlacement.UNRELATED)
                                .addComponent(nilaiKeterbukaanPasar, javax.swing.GroupLayout.PREFERRED_SIZE, 51, javax.swing.GroupLayout.PREFERRED_SIZE)
                                .addPreferredGap(javax.swing.LayoutStyle.ComponentPlacement.UNRELATED)
                                .addComponent(jLabel13))
                            .addComponent(jLabel14)
                            .addGroup(jPanel1Layout.createSequentialGroup()
                                .addComponent(jLabel5)
                                .addGap(18, 18, 18)
                                .addComponent(nilaiDP, javax.swing.GroupLayout.PREFERRED_SIZE, 48, javax.swing.GroupLayout.PREFERRED_SIZE)
                                .addPreferredGap(javax.swing.LayoutStyle.ComponentPlacement.UNRELATED)
                                .addComponent(jLabel9))
                            .addGroup(jPanel1Layout.createSequentialGroup()
                                .addComponent(jLabel15)
                                .addPreferredGap(javax.swing.LayoutStyle.ComponentPlacement.UNRELATED)
                                .addComponent(nilaiInfrastrukturKomersial, javax.swing.GroupLayout.PREFERRED_SIZE, 44, javax.swing.GroupLayout.PREFERRED_SIZE)
                                .addPreferredGap(javax.swing.LayoutStyle.ComponentPlacement.RELATED)
                                .addComponent(jLabel16))
                            .addGroup(jPanel1Layout.createSequentialGroup()
                                .addComponent(jLabel17)
                                .addPreferredGap(javax.swing.LayoutStyle.ComponentPlacement.RELATED)
                                .addComponent(nilaiTransferPenelitian, javax.swing.GroupLayout.PREFERRED_SIZE, 42, javax.swing.GroupLayout.PREFERRED_SIZE)
                                .addPreferredGap(javax.swing.LayoutStyle.ComponentPlacement.RELATED)
                                .addComponent(jLabel18))
                            .addGroup(jPanel1Layout.createSequentialGroup()
                                .addComponent(jLabel4)
                                .addPreferredGap(javax.swing.LayoutStyle.ComponentPlacement.UNRELATED)
                                .addComponent(nilaiPP, javax.swing.GroupLayout.PREFERRED_SIZE, 48, javax.swing.GroupLayout.PREFERRED_SIZE)
                                .addPreferredGap(javax.swing.LayoutStyle.ComponentPlacement.RELATED)
                                .addComponent(jLabel8))
                            .addGroup(jPanel1Layout.createSequentialGroup()
                                .addGap(162, 162, 162)
                                .addComponent(jLabel1))
                            .addGroup(jPanel1Layout.createSequentialGroup()
                                .addComponent(jLabel21)
                                .addPreferredGap(javax.swing.LayoutStyle.ComponentPlacement.RELATED)
                                .addComponent(nilaiPendidikanSDSMP, javax.swing.GroupLayout.PREFERRED_SIZE, 41, javax.swing.GroupLayout.PREFERRED_SIZE)
                                .addPreferredGap(javax.swing.LayoutStyle.ComponentPlacement.RELATED)
                                .addComponent(jLabel22))
                            .addGroup(jPanel1Layout.createSequentialGroup()
                                .addComponent(jLabel19)
                                .addPreferredGap(javax.swing.LayoutStyle.ComponentPlacement.RELATED)
                                .addComponent(nilaiPendidikanSMK, javax.swing.GroupLayout.PREFERRED_SIZE, 41, javax.swing.GroupLayout.PREFERRED_SIZE)
                                .addPreferredGap(javax.swing.LayoutStyle.ComponentPlacement.UNRELATED)
                                .addComponent(jLabel20))
                            .addGroup(jPanel1Layout.createSequentialGroup()
                                .addComponent(jLabel23)
                                .addPreferredGap(javax.swing.LayoutStyle.ComponentPlacement.RELATED)
                                .addComponent(nilaiKPPajak, javax.swing.GroupLayout.PREFERRED_SIZE, 46, javax.swing.GroupLayout.PREFERRED_SIZE)
                                .addPreferredGap(javax.swing.LayoutStyle.ComponentPlacement.RELATED)
                                .addComponent(jLabel24, javax.swing.GroupLayout.PREFERRED_SIZE, 17, javax.swing.GroupLayout.PREFERRED_SIZE))
                            .addGroup(jPanel1Layout.createSequentialGroup()
                                .addComponent(jLabel25)
                                .addPreferredGap(javax.swing.LayoutStyle.ComponentPlacement.RELATED)
                                .addComponent(nilaiKPEkonomi, javax.swing.GroupLayout.PREFERRED_SIZE, 47, javax.swing.GroupLayout.PREFERRED_SIZE)
                                .addPreferredGap(javax.swing.LayoutStyle.ComponentPlacement.RELATED)
                                .addComponent(jLabel26))
                            .addGroup(jPanel1Layout.createSequentialGroup()
                                .addComponent(jLabel27)
                                .addPreferredGap(javax.swing.LayoutStyle.ComponentPlacement.UNRELATED)
                                .addComponent(nilaiKeuanganKewirausahaan, javax.swing.GroupLayout.PREFERRED_SIZE, 42, javax.swing.GroupLayout.PREFERRED_SIZE)
                                .addPreferredGap(javax.swing.LayoutStyle.ComponentPlacement.RELATED)
                                .addComponent(jLabel28))))
                    .addGroup(jPanel1Layout.createSequentialGroup()
                        .addContainerGap()
                        .addComponent(jLabel2)))
                .addContainerGap(86, Short.MAX_VALUE))
        );
        jPanel1Layout.setVerticalGroup(
            jPanel1Layout.createParallelGroup(javax.swing.GroupLayout.Alignment.LEADING)
            .addGroup(jPanel1Layout.createSequentialGroup()
                .addContainerGap()
                .addComponent(jLabel1)
                .addGap(18, 18, 18)
                .addComponent(jLabel2)
                .addPreferredGap(javax.swing.LayoutStyle.ComponentPlacement.UNRELATED)
                .addComponent(jLabel3)
                .addGap(18, 18, 18)
                .addGroup(jPanel1Layout.createParallelGroup(javax.swing.GroupLayout.Alignment.BASELINE)
                    .addComponent(jLabel27)
                    .addComponent(nilaiKeuanganKewirausahaan, javax.swing.GroupLayout.PREFERRED_SIZE, javax.swing.GroupLayout.DEFAULT_SIZE, javax.swing.GroupLayout.PREFERRED_SIZE)
                    .addComponent(jLabel28))
                .addGap(18, 18, 18)
                .addGroup(jPanel1Layout.createParallelGroup(javax.swing.GroupLayout.Alignment.BASELINE)
                    .addComponent(jLabel25)
                    .addComponent(nilaiKPEkonomi, javax.swing.GroupLayout.PREFERRED_SIZE, javax.swing.GroupLayout.DEFAULT_SIZE, javax.swing.GroupLayout.PREFERRED_SIZE)
                    .addComponent(jLabel26))
                .addGap(18, 18, 18)
                .addGroup(jPanel1Layout.createParallelGroup(javax.swing.GroupLayout.Alignment.BASELINE)
                    .addComponent(jLabel23)
                    .addComponent(nilaiKPPajak, javax.swing.GroupLayout.PREFERRED_SIZE, javax.swing.GroupLayout.DEFAULT_SIZE, javax.swing.GroupLayout.PREFERRED_SIZE)
                    .addComponent(jLabel24))
                .addGap(18, 18, Short.MAX_VALUE)
                .addGroup(jPanel1Layout.createParallelGroup(javax.swing.GroupLayout.Alignment.BASELINE)
                    .addComponent(jLabel4)
                    .addComponent(nilaiPP, javax.swing.GroupLayout.PREFERRED_SIZE, javax.swing.GroupLayout.DEFAULT_SIZE, javax.swing.GroupLayout.PREFERRED_SIZE)
                    .addComponent(jLabel8))
                .addGap(18, 18, 18)
                .addGroup(jPanel1Layout.createParallelGroup(javax.swing.GroupLayout.Alignment.BASELINE)
                    .addComponent(jLabel21)
                    .addComponent(nilaiPendidikanSDSMP, javax.swing.GroupLayout.PREFERRED_SIZE, javax.swing.GroupLayout.DEFAULT_SIZE, javax.swing.GroupLayout.PREFERRED_SIZE)
                    .addComponent(jLabel22))
                .addGap(18, 18, 18)
                .addGroup(jPanel1Layout.createParallelGroup(javax.swing.GroupLayout.Alignment.BASELINE)
                    .addComponent(jLabel19)
                    .addComponent(nilaiPendidikanSMK, javax.swing.GroupLayout.PREFERRED_SIZE, javax.swing.GroupLayout.DEFAULT_SIZE, javax.swing.GroupLayout.PREFERRED_SIZE)
                    .addComponent(jLabel20))
                .addGap(18, 18, 18)
                .addGroup(jPanel1Layout.createParallelGroup(javax.swing.GroupLayout.Alignment.BASELINE)
                    .addComponent(jLabel17)
                    .addComponent(nilaiTransferPenelitian, javax.swing.GroupLayout.PREFERRED_SIZE, javax.swing.GroupLayout.DEFAULT_SIZE, javax.swing.GroupLayout.PREFERRED_SIZE)
                    .addComponent(jLabel18))
                .addGap(16, 16, 16)
                .addGroup(jPanel1Layout.createParallelGroup(javax.swing.GroupLayout.Alignment.LEADING)
                    .addComponent(jLabel15, javax.swing.GroupLayout.Alignment.TRAILING)
                    .addGroup(jPanel1Layout.createParallelGroup(javax.swing.GroupLayout.Alignment.BASELINE)
                        .addComponent(nilaiInfrastrukturKomersial, javax.swing.GroupLayout.PREFERRED_SIZE, javax.swing.GroupLayout.DEFAULT_SIZE, javax.swing.GroupLayout.PREFERRED_SIZE)
                        .addComponent(jLabel16)))
                .addPreferredGap(javax.swing.LayoutStyle.ComponentPlacement.UNRELATED)
                .addGroup(jPanel1Layout.createParallelGroup(javax.swing.GroupLayout.Alignment.BASELINE)
                    .addComponent(jLabel5)
                    .addComponent(nilaiDP, javax.swing.GroupLayout.PREFERRED_SIZE, javax.swing.GroupLayout.DEFAULT_SIZE, javax.swing.GroupLayout.PREFERRED_SIZE)
                    .addComponent(jLabel9))
                .addPreferredGap(javax.swing.LayoutStyle.ComponentPlacement.RELATED)
                .addComponent(jLabel14)
                .addPreferredGap(javax.swing.LayoutStyle.ComponentPlacement.UNRELATED)
                .addGroup(jPanel1Layout.createParallelGroup(javax.swing.GroupLayout.Alignment.BASELINE)
                    .addComponent(jLabel12)
                    .addComponent(nilaiKeterbukaanPasar, javax.swing.GroupLayout.PREFERRED_SIZE, javax.swing.GroupLayout.DEFAULT_SIZE, javax.swing.GroupLayout.PREFERRED_SIZE)
                    .addComponent(jLabel13))
                .addGap(18, 18, 18)
                .addGroup(jPanel1Layout.createParallelGroup(javax.swing.GroupLayout.Alignment.BASELINE)
                    .addComponent(jLabel7)
                    .addComponent(nilaiIFA, javax.swing.GroupLayout.PREFERRED_SIZE, javax.swing.GroupLayout.DEFAULT_SIZE, javax.swing.GroupLayout.PREFERRED_SIZE)
                    .addComponent(jLabel11))
                .addPreferredGap(javax.swing.LayoutStyle.ComponentPlacement.UNRELATED)
                .addGroup(jPanel1Layout.createParallelGroup(javax.swing.GroupLayout.Alignment.BASELINE)
                    .addComponent(jLabel6)
                    .addComponent(nilaiNSB, javax.swing.GroupLayout.PREFERRED_SIZE, javax.swing.GroupLayout.DEFAULT_SIZE, javax.swing.GroupLayout.PREFERRED_SIZE)
                    .addComponent(jLabel10))
                .addGap(35, 35, 35)
                .addGroup(jPanel1Layout.createParallelGroup(javax.swing.GroupLayout.Alignment.BASELINE)
                    .addComponent(nextButton)
                    .addComponent(backButton))
                .addContainerGap())
        );

        javax.swing.GroupLayout layout = new javax.swing.GroupLayout(getContentPane());
        getContentPane().setLayout(layout);
        layout.setHorizontalGroup(
            layout.createParallelGroup(javax.swing.GroupLayout.Alignment.LEADING)
            .addGroup(layout.createSequentialGroup()
                .addContainerGap()
                .addComponent(jPanel1, javax.swing.GroupLayout.DEFAULT_SIZE, javax.swing.GroupLayout.DEFAULT_SIZE, Short.MAX_VALUE)
                .addContainerGap())
        );
        layout.setVerticalGroup(
            layout.createParallelGroup(javax.swing.GroupLayout.Alignment.LEADING)
            .addGroup(layout.createSequentialGroup()
                .addContainerGap()
                .addComponent(jPanel1, javax.swing.GroupLayout.DEFAULT_SIZE, javax.swing.GroupLayout.DEFAULT_SIZE, Short.MAX_VALUE)
                .addContainerGap())
        );

        pack();
    }// </editor-fold>                        

    private void backButtonActionPerformed(java.awt.event.ActionEvent evt) {                                           
        // TODO add your handling code here:
        this.hide();
        TampilanKondisiKetetanggaan ki = new TampilanKondisiKetetanggaan();
        ki.setVisible(true);
    }                                          

    private void nextButtonActionPerformed(java.awt.event.ActionEvent evt) {                                           
        // TODO add your handling code here:


    }                                          

    private void nilaiPPActionPerformed(java.awt.event.ActionEvent evt) {                                        

    }                                       

    private void nilaiDPActionPerformed(java.awt.event.ActionEvent evt) {                                        

    }                                       

    private void nilaiNSBActionPerformed(java.awt.event.ActionEvent evt) {                                         
        // TODO add your handling code here:

    }                                        

    private void nilaiIFAActionPerformed(java.awt.event.ActionEvent evt) {                                         

    }                                        

    private void nextButtonMouseClicked(java.awt.event.MouseEvent evt) {                                        
        boolean checker = true;
//        String isiNilaiDP;
//        String isiNilaiIFA;
//        String isiNilaiNSB;
//        String isiNilaiPP;
        double isiNilaiKK = 0.0;
        double isiNilaiIK = 0.0;
        double isiNilaiKPE = 0.0;
        double isiNilaiKPP = 0.0;
        double isiNilaiKP = 0.0;
        double isiNilaiPSS = 0.0;
        double isiNilaiPS = 0.0;
        double isiNilaiTP = 0.0;
        double[] kumpulanNilaiPF = new double[12];
        double isiNilaiDP = 0.0;
        double isiNilaiIFA = 0.0;
        double isiNilaiNSB = 0.0;
        double isiNilaiPP = 0.0;
        if (nilaiDP.getText().equals("")) {
            InputDataHandler.inputDataEksternal("dinamikaPasar", null);
            checker = false;

        } else if (nilaiIFA.getText().equals("")) {
            InputDataHandler.inputDataEksternal("InfrastrukturListrik", null);
            checker = false;
        } else if (nilaiNSB.getText().equals("")) {
            InputDataHandler.inputDataEksternal("NormaSosialBudaya", null);
            checker = false;
        } else if (nilaiPP.getText().equals("")) {
            InputDataHandler.inputDataEksternal("ProgramPemerintah", null);
            checker = false;
        } else if (nilaiInfrastrukturKomersial.getText().equals("")) {
            InputDataHandler.inputDataEksternal("InfrastrukturKomersial", null);
            checker = false;
        } else if (nilaiKPEkonomi.getText().equals("")) {
            InputDataHandler.inputDataEksternal("NilaiKPEkonomi", null);
            checker = false;
        } else if (nilaiKPPajak.getText().equals("")) {
            InputDataHandler.inputDataEksternal("nilaiKPPajak", null);
            checker = false;
        } else if (nilaiKeterbukaanPasar.getText().equals("")) {
            InputDataHandler.inputDataEksternal("nilaiKeterbukaanPasar", null);
            checker = false;
        } else if (nilaiKeuanganKewirausahaan.getText().equals("")) {
            InputDataHandler.inputDataEksternal("nilaiKeuanganKewirausahaan", null);
            checker = false;
        } else if (nilaiPendidikanSDSMP.getText().equals("")) {
            InputDataHandler.inputDataEksternal("nilaiPendidikanSDSMP", null);
            checker = false;
        } else if (nilaiPendidikanSMK.getText().equals("")) {
            InputDataHandler.inputDataEksternal("nilaiPendidikanSMK", null);
            checker = false;
        } else if (nilaiTransferPenelitian.getText().equals("")) {
            InputDataHandler.inputDataEksternal("nilaiTransferPenelitian", null);
            checker = false;
        } else {
            if (!nilaiKeuanganKewirausahaan.equals("")) {
                isiNilaiKK = Double.parseDouble(nilaiKeuanganKewirausahaan.getText()) / 100.0;
                String nilaiKK = Double.toString(isiNilaiKK);
                InputDataHandler.inputDataEksternal("nilaiKeuanganKewirausahaan", nilaiKK);
                kumpulanNilaiPF[0] = Double.parseDouble(InputDataHandler.getValue("nilaiKeuanganKewirausahaan"));
                if (!nilaiKPEkonomi.getText().equals("")) {
                    isiNilaiKPE = Double.parseDouble(nilaiKPEkonomi.getText()) / 100.0;
                    String nilaiKP = Double.toString(isiNilaiKPE);
                    InputDataHandler.inputDataEksternal("nilaiKPEkonomi", nilaiKP);
                    kumpulanNilaiPF[1] = Double.parseDouble(InputDataHandler.getValue("nilaiKPEkonomi"));
                    if (!nilaiKPPajak.getText().equals("")) {
                        isiNilaiKPP = Double.parseDouble(nilaiKPPajak.getText()) / 100.0;
                        String nilaiKPP = Double.toString(isiNilaiKPP);
                        InputDataHandler.inputDataEksternal("nilaiKPPajak", nilaiKPP);
                        kumpulanNilaiPF[2] = Double.parseDouble(InputDataHandler.getValue("nilaiKPPajak"));
                        if (!nilaiPP.getText().equals("")) {
                            isiNilaiPP = Double.parseDouble(nilaiPP.getText()) / 100.0;
                            String nilaiPP = Double.toString(isiNilaiPP);
                            InputDataHandler.inputDataEksternal("ProgramPemerintah", nilaiPP);
                            kumpulanNilaiPF[3] = Double.parseDouble(InputDataHandler.getValue("ProgramPemerintah"));

                            if (!nilaiPendidikanSDSMP.getText().equals("")) {
                                isiNilaiPSS = Double.parseDouble(nilaiPendidikanSDSMP.getText()) / 100.0;
                                String nilaiPSS = Double.toString(isiNilaiPSS);
                                InputDataHandler.inputDataEksternal("nilaiPendidikanSDSMP", nilaiPSS);
                                kumpulanNilaiPF[4] = Double.parseDouble(InputDataHandler.getValue("nilaiPendidikanSDSMP"));
                                if (!nilaiPendidikanSMK.getText().equals("")) {
                                    isiNilaiPS = Double.parseDouble(nilaiPendidikanSMK.getText()) / 100.0;
                                    String nilaiPS = Double.toString(isiNilaiPS);
                                    InputDataHandler.inputDataEksternal("nilaiPendidikanSMK", nilaiPS);
                                    kumpulanNilaiPF[5] = Double.parseDouble(InputDataHandler.getValue("nilaiPendidikanSMK"));
                                    if (!nilaiTransferPenelitian.getText().equals("")) {
                                        isiNilaiTP = Double.parseDouble(nilaiTransferPenelitian.getText()) / 100.0;
                                        String nilaiTP = Double.toString(isiNilaiTP);
                                        InputDataHandler.inputDataEksternal("nilaiTransferPenelitian", nilaiTP);
                                        kumpulanNilaiPF[6] = Double.parseDouble(InputDataHandler.getValue("nilaiTransferPenelitian"));
                                        if (!nilaiInfrastrukturKomersial.getText().equals("")) {
                                            isiNilaiIK = Double.parseDouble(nilaiInfrastrukturKomersial.getText()) / 100.0;
                                            String nilaiIK = Double.toString(isiNilaiIK);
                                            InputDataHandler.inputDataEksternal("nilaiInfrastrukturKomersial", nilaiIK);
                                            kumpulanNilaiPF[7] = Double.parseDouble(InputDataHandler.getValue("nilaiInfrastrukturKomersial"));
                                            if (!nilaiDP.getText().equals("")) {
                                                isiNilaiDP = Double.parseDouble(nilaiDP.getText()) / 100.0;
                                                String nilaiDP = Double.toString(isiNilaiDP);
                                                InputDataHandler.inputDataEksternal("DinamikaPasar", nilaiDP);
                                                kumpulanNilaiPF[8] = Double.parseDouble(InputDataHandler.getValue("DinamikaPasar"));
                                                if (!nilaiKeterbukaanPasar.getText().equals("")) {
                                                    isiNilaiKP = Double.parseDouble(nilaiKeterbukaanPasar.getText()) / 100.0;
                                                    String nilaiKPas = Double.toString(isiNilaiKP);
                                                    InputDataHandler.inputDataEksternal("nilaiKeterbukaanPasar", nilaiKPas);
                                                    kumpulanNilaiPF[9] = Double.parseDouble(InputDataHandler.getValue("nilaiKeterbukaanPasar"));
                                                    if (!nilaiIFA.getText().equals("")) {
                                                        isiNilaiIFA = Double.parseDouble(nilaiIFA.getText()) / 100.0;
                                                        String nilaiIFA = Double.toString(isiNilaiIFA);
                                                        InputDataHandler.inputDataEksternal("InfrastrukturListrik", nilaiIFA);
                                                        kumpulanNilaiPF[10] = Double.parseDouble(InputDataHandler.getValue("InfrastrukturListrik"));
                                                        if (!nilaiNSB.getText().equals("")) {
                                                            isiNilaiNSB = Double.parseDouble(nilaiNSB.getText()) / 100.0;
                                                            String nilaiNSB = Double.toString(isiNilaiNSB);
                                                            InputDataHandler.inputDataEksternal("NormaSosialBudaya", nilaiNSB);
                                                            kumpulanNilaiPF[11] = Double.parseDouble(InputDataHandler.getValue("NormaSosialBudaya"));
                                                        }
                                                    }
                                                }
                                            }
                                        }
                                    }

                                }
                            }
                        }
                    }
                }
            }
        }
        int totalNilai=0;
        for (int i = 0; i < kumpulanNilaiPF.length; i++) {
            totalNilai+=kumpulanNilaiPF[i]*100;
        }
        
        if ( totalNilai != 100) {
            JOptionPane.showMessageDialog(null, "The sum of text fields must 100%!");
            checker = false;
        }
        if (checker == true) {
            this.hide();
            TampilanDataWirausaha ks = new TampilanDataWirausaha();
            ks.setVisible(true);
        } else {
            JOptionPane.showMessageDialog(null, "You must fill the text field!");
        }
        InputDataHandler.setDataEksternal(kumpulanNilaiPF);
    }                                       

    private void nilaiKeterbukaanPasarActionPerformed(java.awt.event.ActionEvent evt) {                                                      
        // TODO add your handling code here:
    }                                                     

    private void nilaiInfrastrukturKomersialActionPerformed(java.awt.event.ActionEvent evt) {                                                            
        // TODO add your handling code here:
    }                                                           

    private void nilaiPendidikanSDSMPActionPerformed(java.awt.event.ActionEvent evt) {                                                     
        // TODO add your handling code here:
    }                                                    

    /**
     * @param args the command line arguments
     */
    public static void main(String args[]) {
        /* Set the Nimbus look and feel */
        //<editor-fold defaultstate="collapsed" desc=" Look and feel setting code (optional) ">
        /* If Nimbus (introduced in Java SE 6) is not available, stay with the default look and feel.
         * For details see http://download.oracle.com/javase/tutorial/uiswing/lookandfeel/plaf.html 
         */
        try {
            for (javax.swing.UIManager.LookAndFeelInfo info : javax.swing.UIManager.getInstalledLookAndFeels()) {
                if ("Nimbus".equals(info.getName())) {
                    javax.swing.UIManager.setLookAndFeel(info.getClassName());
                    break;
                }
            }
        } catch (ClassNotFoundException ex) {
            java.util.logging.Logger.getLogger(TampilanKondisiEksternal.class.getName()).log(java.util.logging.Level.SEVERE, null, ex);
        } catch (InstantiationException ex) {
            java.util.logging.Logger.getLogger(TampilanKondisiEksternal.class.getName()).log(java.util.logging.Level.SEVERE, null, ex);
        } catch (IllegalAccessException ex) {
            java.util.logging.Logger.getLogger(TampilanKondisiEksternal.class.getName()).log(java.util.logging.Level.SEVERE, null, ex);
        } catch (javax.swing.UnsupportedLookAndFeelException ex) {
            java.util.logging.Logger.getLogger(TampilanKondisiEksternal.class.getName()).log(java.util.logging.Level.SEVERE, null, ex);
        }
        //</editor-fold>

        /* Create and display the form */
        java.awt.EventQueue.invokeLater(new Runnable() {
            public void run() {
                new TampilanKondisiEksternal().setVisible(true);
            }
        });
    }

    // Variables declaration - do not modify                     
    private javax.swing.JButton backButton;
    private javax.swing.JLabel jLabel1;
    private javax.swing.JLabel jLabel10;
    private javax.swing.JLabel jLabel11;
    private javax.swing.JLabel jLabel12;
    private javax.swing.JLabel jLabel13;
    private javax.swing.JLabel jLabel14;
    private javax.swing.JLabel jLabel15;
    private javax.swing.JLabel jLabel16;
    private javax.swing.JLabel jLabel17;
    private javax.swing.JLabel jLabel18;
    private javax.swing.JLabel jLabel19;
    private javax.swing.JLabel jLabel2;
    private javax.swing.JLabel jLabel20;
    private javax.swing.JLabel jLabel21;
    private javax.swing.JLabel jLabel22;
    private javax.swing.JLabel jLabel23;
    private javax.swing.JLabel jLabel24;
    private javax.swing.JLabel jLabel25;
    private javax.swing.JLabel jLabel26;
    private javax.swing.JLabel jLabel27;
    private javax.swing.JLabel jLabel28;
    private javax.swing.JLabel jLabel3;
    private javax.swing.JLabel jLabel4;
    private javax.swing.JLabel jLabel5;
    private javax.swing.JLabel jLabel6;
    private javax.swing.JLabel jLabel7;
    private javax.swing.JLabel jLabel8;
    private javax.swing.JLabel jLabel9;
    private javax.swing.JPanel jPanel1;
    public javax.swing.JButton nextButton;
    private javax.swing.JTextField nilaiDP;
    private javax.swing.JTextField nilaiIFA;
    private javax.swing.JTextField nilaiInfrastrukturKomersial;
    private javax.swing.JTextField nilaiKPEkonomi;
    private javax.swing.JTextField nilaiKPPajak;
    private javax.swing.JTextField nilaiKeterbukaanPasar;
    private javax.swing.JTextField nilaiKeuanganKewirausahaan;
    private javax.swing.JTextField nilaiNSB;
    private javax.swing.JTextField nilaiPP;
    private javax.swing.JTextField nilaiPendidikanSDSMP;
    private javax.swing.JTextField nilaiPendidikanSMK;
    private javax.swing.JTextField nilaiTransferPenelitian;
    // End of variables declaration                   
}

\end{lstlisting}

\begin{lstlisting}[language=Java, caption=TampilanDataWirausaha.java]
/*
 * To change this license header, choose License Headers in Project Properties.
 * To change this template file, choose Tools | Templates
 * and open the template in the editor.
 */
package ecasimulatorjframe;

import java.io.BufferedReader;
import java.io.File;
import java.io.FileReader;
import java.io.IOException;
import java.util.logging.Level;
import java.util.logging.Logger;
import javax.swing.JFileChooser;
import javax.swing.JOptionPane;
import javax.swing.table.DefaultTableModel;

/**
 *
 * @author Vanessa
 */
public class TampilanDataWirausaha extends javax.swing.JFrame {

    /**
     * Creates new form TampilanSimulasi
     */
    CA ca;
    public BufferedReader br;
    private final JFileChooser openFileChooser;

    public TampilanDataWirausaha() {
        initComponents();
        openFileChooser = new JFileChooser();
        openFileChooser.setSelectedFile(new File("D:\\text.txt"));
    }

    /**
     * This method is called from within the constructor to initialize the form.
     * WARNING: Do NOT modify this code. The content of this method is always
     * regenerated by the Form Editor.
     */
    @SuppressWarnings("unchecked")
    // <editor-fold defaultstate="collapsed" desc="Generated Code">                          
    private void initComponents() {

        jPanel1 = new javax.swing.JPanel();
        jPanel2 = new javax.swing.JPanel();
        jLabel1 = new javax.swing.JLabel();
        jLabel2 = new javax.swing.JLabel();
        openFileButton = new javax.swing.JButton();
        messageLabel = new javax.swing.JLabel();
        jScrollPane1 = new javax.swing.JScrollPane();
        jTable1 = new javax.swing.JTable();
        nextButton = new javax.swing.JButton();
        backButton = new javax.swing.JButton();

        javax.swing.GroupLayout jPanel1Layout = new javax.swing.GroupLayout(jPanel1);
        jPanel1.setLayout(jPanel1Layout);
        jPanel1Layout.setHorizontalGroup(
            jPanel1Layout.createParallelGroup(javax.swing.GroupLayout.Alignment.LEADING)
            .addGap(0, 100, Short.MAX_VALUE)
        );
        jPanel1Layout.setVerticalGroup(
            jPanel1Layout.createParallelGroup(javax.swing.GroupLayout.Alignment.LEADING)
            .addGap(0, 100, Short.MAX_VALUE)
        );

        setDefaultCloseOperation(javax.swing.WindowConstants.EXIT_ON_CLOSE);

        jLabel1.setFont(new java.awt.Font("Tahoma", 1, 14)); // NOI18N
        jLabel1.setText("SIMULATOR ECA");

        jLabel2.setText("Data Simulasi :");

        openFileButton.setText("OPEN FILE");
        openFileButton.addMouseListener(new java.awt.event.MouseAdapter() {
            public void mouseClicked(java.awt.event.MouseEvent evt) {
                openFileButtonMouseClicked(evt);
            }
        });
        openFileButton.addActionListener(new java.awt.event.ActionListener() {
            public void actionPerformed(java.awt.event.ActionEvent evt) {
                openFileButtonActionPerformed(evt);
            }
        });

        jTable1.setModel(new javax.swing.table.DefaultTableModel(
            new Object [][] {

            },
            new String [] {
                "Jenis Kelamin", "Umur", "Usia Bisnis (bulan)", "Kategori", "Sub Kategori", "Pendidikan", "Lokasi", "Pendapatan", "Level", "Point"
            }
        ));
        jTable1.setCursor(new java.awt.Cursor(java.awt.Cursor.DEFAULT_CURSOR));
        jScrollPane1.setViewportView(jTable1);

        nextButton.setText("NEXT");
        nextButton.addMouseListener(new java.awt.event.MouseAdapter() {
            public void mouseClicked(java.awt.event.MouseEvent evt) {
                nextButtonMouseClicked(evt);
            }
        });
        nextButton.addActionListener(new java.awt.event.ActionListener() {
            public void actionPerformed(java.awt.event.ActionEvent evt) {
                nextButtonActionPerformed(evt);
            }
        });

        backButton.setText("BACK");
        backButton.addActionListener(new java.awt.event.ActionListener() {
            public void actionPerformed(java.awt.event.ActionEvent evt) {
                backButtonActionPerformed(evt);
            }
        });

        javax.swing.GroupLayout jPanel2Layout = new javax.swing.GroupLayout(jPanel2);
        jPanel2.setLayout(jPanel2Layout);
        jPanel2Layout.setHorizontalGroup(
            jPanel2Layout.createParallelGroup(javax.swing.GroupLayout.Alignment.LEADING)
            .addGroup(jPanel2Layout.createSequentialGroup()
                .addContainerGap()
                .addGroup(jPanel2Layout.createParallelGroup(javax.swing.GroupLayout.Alignment.LEADING)
                    .addGroup(jPanel2Layout.createSequentialGroup()
                        .addGroup(jPanel2Layout.createParallelGroup(javax.swing.GroupLayout.Alignment.LEADING)
                            .addGroup(jPanel2Layout.createSequentialGroup()
                                .addComponent(jLabel2)
                                .addPreferredGap(javax.swing.LayoutStyle.ComponentPlacement.UNRELATED)
                                .addGroup(jPanel2Layout.createParallelGroup(javax.swing.GroupLayout.Alignment.LEADING)
                                    .addComponent(openFileButton)
                                    .addGroup(jPanel2Layout.createSequentialGroup()
                                        .addGap(111, 111, 111)
                                        .addComponent(messageLabel, javax.swing.GroupLayout.PREFERRED_SIZE, 122, javax.swing.GroupLayout.PREFERRED_SIZE)))
                                .addGap(0, 0, Short.MAX_VALUE))
                            .addComponent(jScrollPane1, javax.swing.GroupLayout.Alignment.TRAILING, javax.swing.GroupLayout.DEFAULT_SIZE, 579, Short.MAX_VALUE))
                        .addContainerGap())
                    .addGroup(javax.swing.GroupLayout.Alignment.TRAILING, jPanel2Layout.createSequentialGroup()
                        .addGap(0, 0, Short.MAX_VALUE)
                        .addComponent(jLabel1)
                        .addGap(235, 235, 235))))
            .addGroup(javax.swing.GroupLayout.Alignment.TRAILING, jPanel2Layout.createSequentialGroup()
                .addGap(48, 48, 48)
                .addComponent(backButton)
                .addPreferredGap(javax.swing.LayoutStyle.ComponentPlacement.RELATED, javax.swing.GroupLayout.DEFAULT_SIZE, Short.MAX_VALUE)
                .addComponent(nextButton)
                .addGap(53, 53, 53))
        );
        jPanel2Layout.setVerticalGroup(
            jPanel2Layout.createParallelGroup(javax.swing.GroupLayout.Alignment.LEADING)
            .addGroup(jPanel2Layout.createSequentialGroup()
                .addContainerGap()
                .addComponent(jLabel1)
                .addGap(23, 23, 23)
                .addGroup(jPanel2Layout.createParallelGroup(javax.swing.GroupLayout.Alignment.BASELINE)
                    .addComponent(jLabel2)
                    .addComponent(openFileButton)
                    .addComponent(messageLabel))
                .addPreferredGap(javax.swing.LayoutStyle.ComponentPlacement.UNRELATED)
                .addComponent(jScrollPane1, javax.swing.GroupLayout.PREFERRED_SIZE, 319, javax.swing.GroupLayout.PREFERRED_SIZE)
                .addPreferredGap(javax.swing.LayoutStyle.ComponentPlacement.RELATED, 65, Short.MAX_VALUE)
                .addGroup(jPanel2Layout.createParallelGroup(javax.swing.GroupLayout.Alignment.BASELINE)
                    .addComponent(nextButton)
                    .addComponent(backButton))
                .addGap(31, 31, 31))
        );

        javax.swing.GroupLayout layout = new javax.swing.GroupLayout(getContentPane());
        getContentPane().setLayout(layout);
        layout.setHorizontalGroup(
            layout.createParallelGroup(javax.swing.GroupLayout.Alignment.LEADING)
            .addGroup(layout.createSequentialGroup()
                .addContainerGap()
                .addComponent(jPanel2, javax.swing.GroupLayout.DEFAULT_SIZE, javax.swing.GroupLayout.DEFAULT_SIZE, Short.MAX_VALUE)
                .addContainerGap())
        );
        layout.setVerticalGroup(
            layout.createParallelGroup(javax.swing.GroupLayout.Alignment.LEADING)
            .addGroup(layout.createSequentialGroup()
                .addContainerGap()
                .addComponent(jPanel2, javax.swing.GroupLayout.DEFAULT_SIZE, javax.swing.GroupLayout.DEFAULT_SIZE, Short.MAX_VALUE)
                .addContainerGap())
        );

        pack();
    }// </editor-fold>                        

    private void openFileButtonActionPerformed(java.awt.event.ActionEvent evt) {                                               

    }                                              

    private void nextButtonActionPerformed(java.awt.event.ActionEvent evt) {                                           

        if (!openFileChooser.getSelectedFile().exists()) {
            JOptionPane.showMessageDialog(null, "You must choose the file first!");
            return;
        }
        this.hide();
        TampilanSimulasi ts = new TampilanSimulasi(this.ca);
        ts.setVisible(true);
    }                                          

    private void backButtonActionPerformed(java.awt.event.ActionEvent evt) {                                           
        this.hide();
        TampilanKondisiEksternal ke = new TampilanKondisiEksternal();
        ke.setVisible(true);
    }                                          

    private void openFileButtonMouseClicked(java.awt.event.MouseEvent evt) {                                            
        double[] kumpulanBobot;
        int returnValue = openFileChooser.showOpenDialog(this);
        StringBuilder sb = new StringBuilder();
        if (returnValue == JFileChooser.APPROVE_OPTION) {
            try {
                br = new BufferedReader(new FileReader(openFileChooser.getSelectedFile()));
                DefaultTableModel model = (DefaultTableModel) jTable1.getModel();

                Object[] tableLines = br.lines().toArray();
                ca = new CA(tableLines.length, InputDataHandler.getKetetanggaan(), 4);
                kumpulanBobot = new double[InputDataHandler.getKetetanggaan()];
                int i;
                for (i = 0; i < tableLines.length; i++) {
                    String lines = tableLines[i].toString().trim();
                    String[] dataRow = lines.split(",");
                    model.addRow(dataRow);

                    //memasukkan data dari fileInput ke kelas Entrepreneurs ca
                    ca.E[i] = new Entrepreneurs();
                    if (dataRow[0].equals("false")) {
                        ca.E[i].sex = false; // pria
                    } else {
                        ca.E[i].sex = true; //wanita
                    }
                    ca.E[i].age = Integer.parseInt(dataRow[1]);
                    ca.E[i].b_age = Integer.parseInt(dataRow[2]);
                    ca.E[i].b_category = Integer.parseInt(dataRow[3]);
                    ca.E[i].b_area = Integer.parseInt(dataRow[4]);
                    ca.E[i].education = Integer.parseInt(dataRow[5]);
                    ca.E[i].location = Integer.parseInt(dataRow[6]);
                    ca.E[i].income = Integer.parseInt(dataRow[7]);
                    ca.E[i].level = Integer.parseInt(dataRow[8]);
                    ca.E[i].point = 0.0;
                }
            } catch (IOException e) {
                //messageLabel.setText("failed to load the file!");
                Logger.getLogger(TampilanDataWirausaha.class.getName()).log(Level.SEVERE, null, e);
            }
        }
    }                                           

    private void nextButtonMouseClicked(java.awt.event.MouseEvent evt) {                                        

    }                                       

    /**
     * @param args the command line arguments
     */
    public static void main(String args[]) {
        /* Set the Nimbus look and feel */
        //<editor-fold defaultstate="collapsed" desc=" Look and feel setting code (optional) ">
        /* If Nimbus (introduced in Java SE 6) is not available, stay with the default look and feel.
         * For details see http://download.oracle.com/javase/tutorial/uiswing/lookandfeel/plaf.html 
         */
        try {
            for (javax.swing.UIManager.LookAndFeelInfo info : javax.swing.UIManager.getInstalledLookAndFeels()) {
                if ("Nimbus".equals(info.getName())) {
                    javax.swing.UIManager.setLookAndFeel(info.getClassName());
                    break;
                }
            }
        } catch (ClassNotFoundException ex) {
            java.util.logging.Logger.getLogger(TampilanDataWirausaha.class.getName()).log(java.util.logging.Level.SEVERE, null, ex);
        } catch (InstantiationException ex) {
            java.util.logging.Logger.getLogger(TampilanDataWirausaha.class.getName()).log(java.util.logging.Level.SEVERE, null, ex);
        } catch (IllegalAccessException ex) {
            java.util.logging.Logger.getLogger(TampilanDataWirausaha.class.getName()).log(java.util.logging.Level.SEVERE, null, ex);
        } catch (javax.swing.UnsupportedLookAndFeelException ex) {
            java.util.logging.Logger.getLogger(TampilanDataWirausaha.class.getName()).log(java.util.logging.Level.SEVERE, null, ex);
        }
        //</editor-fold>
        //</editor-fold>

        /* Create and display the form */
        java.awt.EventQueue.invokeLater(new Runnable() {
            public void run() {
                new TampilanDataWirausaha().setVisible(true);
            }
        });
    }

    // Variables declaration - do not modify                     
    private javax.swing.JButton backButton;
    private javax.swing.JLabel jLabel1;
    private javax.swing.JLabel jLabel2;
    private javax.swing.JPanel jPanel1;
    private javax.swing.JPanel jPanel2;
    private javax.swing.JScrollPane jScrollPane1;
    private javax.swing.JTable jTable1;
    private javax.swing.JLabel messageLabel;
    public javax.swing.JButton nextButton;
    private javax.swing.JButton openFileButton;
    // End of variables declaration                   
}

\end{lstlisting}

\begin{lstlisting}[language=Java, caption=Entrepreneurs.java]
/*
 * To change this license header, choose License Headers in Project Properties.
 * To change this template file, choose Tools | Templates
 * and open the template in the editor.
 */
package ecasimulatorjframe;

import java.io.BufferedWriter;
import java.io.File;
import java.io.FileNotFoundException;
import java.io.FileWriter;
import java.io.IOException;
import java.io.PrintWriter;
import java.util.logging.Level;
import java.util.logging.Logger;
import javax.swing.JOptionPane;

/**
 *
 * @author Vanessa
 */
public class TampilanSimulasi extends javax.swing.JFrame {

    /**
     * Creates new form TampilanSimulasi
     */
    CA ca;

    public TampilanSimulasi(CA ca) {
        initComponents();
        this.ca = ca;
    }

    /**
     * This method is called from within the constructor to initialize the form.
     * WARNING: Do NOT modify this code. The content of this method is always
     * regenerated by the Form Editor.
     */
    @SuppressWarnings("unchecked")
    // <editor-fold defaultstate="collapsed" desc="Generated Code">                          
    private void initComponents() {

        jLabel1 = new javax.swing.JLabel();
        jLabel2 = new javax.swing.JLabel();
        jLabel3 = new javax.swing.JLabel();
        jLabel4 = new javax.swing.JLabel();
        jLabel5 = new javax.swing.JLabel();
        jLabel6 = new javax.swing.JLabel();
        nilaiA = new javax.swing.JTextField();
        nilaiB = new javax.swing.JTextField();
        nilaiC = new javax.swing.JTextField();
        nilaiThreshold = new javax.swing.JTextField();
        jLabel7 = new javax.swing.JLabel();
        simulateButton = new javax.swing.JButton();
        nilaiPeriode = new javax.swing.JTextField();
        jLabel8 = new javax.swing.JLabel();

        setDefaultCloseOperation(javax.swing.WindowConstants.EXIT_ON_CLOSE);

        jLabel1.setFont(new java.awt.Font("Tahoma", 1, 14)); // NOI18N
        jLabel1.setText("SIMULATOR ECA");

        jLabel2.setText("Simulasi :");

        jLabel3.setText("a :");

        jLabel4.setText("b :");

        jLabel5.setText("c :");

        jLabel6.setText("Threshold :");

        nilaiA.addActionListener(new java.awt.event.ActionListener() {
            public void actionPerformed(java.awt.event.ActionEvent evt) {
                nilaiAActionPerformed(evt);
            }
        });

        nilaiB.addActionListener(new java.awt.event.ActionListener() {
            public void actionPerformed(java.awt.event.ActionEvent evt) {
                nilaiBActionPerformed(evt);
            }
        });

        nilaiC.addActionListener(new java.awt.event.ActionListener() {
            public void actionPerformed(java.awt.event.ActionEvent evt) {
                nilaiCActionPerformed(evt);
            }
        });

        nilaiThreshold.addActionListener(new java.awt.event.ActionListener() {
            public void actionPerformed(java.awt.event.ActionEvent evt) {
                nilaiThresholdActionPerformed(evt);
            }
        });

        jLabel7.setText("Periode :");

        simulateButton.setText("SIMULATE");
        simulateButton.addMouseListener(new java.awt.event.MouseAdapter() {
            public void mouseClicked(java.awt.event.MouseEvent evt) {
                simulateButtonMouseClicked(evt);
            }
        });
        simulateButton.addActionListener(new java.awt.event.ActionListener() {
            public void actionPerformed(java.awt.event.ActionEvent evt) {
                simulateButtonActionPerformed(evt);
            }
        });

        nilaiPeriode.addActionListener(new java.awt.event.ActionListener() {
            public void actionPerformed(java.awt.event.ActionEvent evt) {
                nilaiPeriodeActionPerformed(evt);
            }
        });

        jLabel8.setText(" bulan");

        javax.swing.GroupLayout layout = new javax.swing.GroupLayout(getContentPane());
        getContentPane().setLayout(layout);
        layout.setHorizontalGroup(
            layout.createParallelGroup(javax.swing.GroupLayout.Alignment.LEADING)
            .addGroup(layout.createSequentialGroup()
                .addGap(33, 33, 33)
                .addGroup(layout.createParallelGroup(javax.swing.GroupLayout.Alignment.LEADING, false)
                    .addComponent(jLabel2)
                    .addGroup(layout.createSequentialGroup()
                        .addComponent(jLabel3)
                        .addPreferredGap(javax.swing.LayoutStyle.ComponentPlacement.RELATED)
                        .addComponent(nilaiA)))
                .addPreferredGap(javax.swing.LayoutStyle.ComponentPlacement.RELATED)
                .addComponent(jLabel4)
                .addPreferredGap(javax.swing.LayoutStyle.ComponentPlacement.RELATED)
                .addComponent(nilaiB, javax.swing.GroupLayout.PREFERRED_SIZE, 29, javax.swing.GroupLayout.PREFERRED_SIZE)
                .addPreferredGap(javax.swing.LayoutStyle.ComponentPlacement.RELATED)
                .addComponent(jLabel5)
                .addPreferredGap(javax.swing.LayoutStyle.ComponentPlacement.RELATED)
                .addGroup(layout.createParallelGroup(javax.swing.GroupLayout.Alignment.LEADING)
                    .addComponent(jLabel1)
                    .addGroup(layout.createSequentialGroup()
                        .addGroup(layout.createParallelGroup(javax.swing.GroupLayout.Alignment.TRAILING)
                            .addComponent(simulateButton)
                            .addGroup(layout.createSequentialGroup()
                                .addComponent(nilaiC, javax.swing.GroupLayout.PREFERRED_SIZE, 31, javax.swing.GroupLayout.PREFERRED_SIZE)
                                .addGap(18, 18, 18)
                                .addComponent(jLabel6)))
                        .addGap(13, 13, 13)
                        .addComponent(nilaiThreshold, javax.swing.GroupLayout.PREFERRED_SIZE, 34, javax.swing.GroupLayout.PREFERRED_SIZE)
                        .addPreferredGap(javax.swing.LayoutStyle.ComponentPlacement.UNRELATED)
                        .addComponent(jLabel7)))
                .addPreferredGap(javax.swing.LayoutStyle.ComponentPlacement.UNRELATED)
                .addComponent(nilaiPeriode, javax.swing.GroupLayout.PREFERRED_SIZE, 37, javax.swing.GroupLayout.PREFERRED_SIZE)
                .addPreferredGap(javax.swing.LayoutStyle.ComponentPlacement.UNRELATED)
                .addComponent(jLabel8)
                .addContainerGap(javax.swing.GroupLayout.DEFAULT_SIZE, Short.MAX_VALUE))
        );
        layout.setVerticalGroup(
            layout.createParallelGroup(javax.swing.GroupLayout.Alignment.LEADING)
            .addGroup(layout.createSequentialGroup()
                .addContainerGap()
                .addComponent(jLabel1)
                .addGap(30, 30, 30)
                .addComponent(jLabel2)
                .addPreferredGap(javax.swing.LayoutStyle.ComponentPlacement.UNRELATED)
                .addGroup(layout.createParallelGroup(javax.swing.GroupLayout.Alignment.BASELINE)
                    .addComponent(jLabel3)
                    .addComponent(jLabel4)
                    .addComponent(jLabel5)
                    .addComponent(jLabel6)
                    .addComponent(nilaiA, javax.swing.GroupLayout.PREFERRED_SIZE, javax.swing.GroupLayout.DEFAULT_SIZE, javax.swing.GroupLayout.PREFERRED_SIZE)
                    .addComponent(nilaiB, javax.swing.GroupLayout.PREFERRED_SIZE, javax.swing.GroupLayout.DEFAULT_SIZE, javax.swing.GroupLayout.PREFERRED_SIZE)
                    .addComponent(nilaiC, javax.swing.GroupLayout.PREFERRED_SIZE, javax.swing.GroupLayout.DEFAULT_SIZE, javax.swing.GroupLayout.PREFERRED_SIZE)
                    .addComponent(nilaiThreshold, javax.swing.GroupLayout.PREFERRED_SIZE, javax.swing.GroupLayout.DEFAULT_SIZE, javax.swing.GroupLayout.PREFERRED_SIZE)
                    .addComponent(jLabel7)
                    .addComponent(nilaiPeriode, javax.swing.GroupLayout.PREFERRED_SIZE, javax.swing.GroupLayout.DEFAULT_SIZE, javax.swing.GroupLayout.PREFERRED_SIZE)
                    .addComponent(jLabel8))
                .addPreferredGap(javax.swing.LayoutStyle.ComponentPlacement.RELATED, 27, Short.MAX_VALUE)
                .addComponent(simulateButton)
                .addGap(21, 21, 21))
        );

        pack();
    }// </editor-fold>                        

    private void nilaiAActionPerformed(java.awt.event.ActionEvent evt) {                                       

    }                                      

    private void simulateButtonActionPerformed(java.awt.event.ActionEvent evt) {                                               

    }                                              

    private void nilaiBActionPerformed(java.awt.event.ActionEvent evt) {                                       

    }                                      

    private void nilaiCActionPerformed(java.awt.event.ActionEvent evt) {                                       

    }                                      

    private void nilaiThresholdActionPerformed(java.awt.event.ActionEvent evt) {                                               

    }                                              

    private void nilaiPeriodeActionPerformed(java.awt.event.ActionEvent evt) {                                             

    }                                            

    private void simulateButtonMouseClicked(java.awt.event.MouseEvent evt) {                                            
        boolean checker = true;
        double a = 0.0;
        double b = 0.0;
        double c = 0.0;
        double[] kumpulanBobot = new double[3];
        int m = 0;
        if (nilaiA.getText().equals("")) {
            InputDataHandler.inputDataSimulasi("nilaiA", null);
            checker = false;
        } else if (nilaiB.getText().equals("")) {
            InputDataHandler.inputDataSimulasi("nilaiB", null);
            checker = false;
        } else if (nilaiC.getText().equals("")) {
            InputDataHandler.inputDataSimulasi("nilaiC", null);
            checker = false;
        } else if (nilaiPeriode.getText().equals("")) {
            InputDataHandler.inputDataSimulasi("periode", null);
            checker = false;
        } else if (nilaiThreshold.getText().equals("")) {
            InputDataHandler.inputDataSimulasi("threshold", null);
            checker = false;
        } else {
            if (!nilaiA.getText().equals("")) {
                InputDataHandler.inputDataSimulasi("nilaiA", nilaiA.getText());
                a = Double.parseDouble(nilaiA.getText());
                kumpulanBobot[m] = a;
                m++;
                if (!nilaiB.getText().equals("")) {
                    InputDataHandler.inputDataSimulasi("nilaiB", nilaiB.getText());
                    b = Double.parseDouble(nilaiB.getText());
                    kumpulanBobot[m] = b;
                    m++;
                    if (!nilaiC.getText().equals("")) {
                        InputDataHandler.inputDataSimulasi("nilaiC", nilaiC.getText());
                        c = Double.parseDouble(nilaiC.getText());
                        kumpulanBobot[m] = c;
                        m++;
                        if (!nilaiPeriode.getText().equals("")) {
                            InputDataHandler.inputDataSimulasi("periode", nilaiPeriode.getText());
                            if (!nilaiThreshold.getText().equals("")) {
                                InputDataHandler.inputDataSimulasi("threshold", nilaiThreshold.getText());
                            }
                        }
                    }
                }
            }
        }
        int totalNilai = 0;
        for (int i = 0; i < kumpulanBobot.length; i++) {
            totalNilai += kumpulanBobot[i]*100;
        }
        if (totalNilai!=100) {
            JOptionPane.showMessageDialog(null, "The sum of a,b and c's value must 1!");
        }

        if (checker == false) {
            JOptionPane.showMessageDialog(null, "You must fill the text field first!");
        }

        double[] composition = new double[]{a, b, c};

        double[] POAf = new double[]{8.6, 17.7, 28.4, 29.5, 15.8}; // female
        double[] POAm = new double[]{8.3, 14.5, 26.7, 36.2, 14.3}; // male

        // Perceived Opportunities Education
        double[] POEf = new double[]{1.8, 17.4, 23.4, 49.8, 7.4, 0.1};
        double[] POEm = new double[]{0.7, 11.8, 19.9, 54.7, 12.6, 0.3};

        // Perceived Opportunities Location
        double[] POLf = new double[]{0.3, 6.4, 4.8, 2.8, 1.4, 3.5, 1.7, 46.3, 9.6, 6.1, 9.5, 2.5, 1.1, 1.0, 0.6, 2.4}; 
        double[] POLm = new double[]{0.5, 4.4, 4.5, 2.3, 1.9, 3.8, 2.1, 47.6, 11.1, 6.3, 8.4, 2.7, 0.9, 1.1, 0.5, 2.0};

        // Perceived Opportunities Income
        double[] POIf = new double[]{42.7, 41.5, 10.8, 2.8, 1.5, 0.3, 0, 0.5}; 
        double[] POIm = new double[]{42.1, 41.7, 11.0, 3.4, 0.7, 0.3, 0.5, 0.2};

        // Perceived Capabilities Age
        double[] PCAf = new double[]{8.9, 16.1, 28.2, 31.6, 15.1}; 
        double[] PCAm = new double[]{8.5, 17.3, 26.1, 33.4, 14.7};    

        // Perceived Capabilities Education
        double[] PCEf = new double[]{1.7, 15.4, 22.8, 51.5, 8.2, 0.4}; 
        double[] PCEm = new double[]{0.9, 12.4, 17.4, 56.9, 12.0, 0.5}; 

        // Perceived Capabilities Income
        double[] PCIf = new double[]{41.4, 43.0, 10.2, 3.1, 1.4, 0.2, 0.2, 0.4}; 
        double[] PCIm = new double[]{42.9, 42.1, 10.5, 3.1, 0.8, 0.3, 0.2, 0.2};

        // Perceived Capabilities Location
        double[] PCLf = new double[]{0.4, 7.5, 3.9, 2.4, 2.0, 3.2, 1.6, 41.1, 10.8, 6.9, 9.0, 3.7, 1.2, 1.0, 0.8, 4.4}; 
        double[] PCLm = new double[]{0.7, 5.4, 3.4, 2.6, 3.0, 3.8, 1.8, 41.1, 11.5, 7.2, 8.7, 3.5, 1.2, 0.9, 0.7, 4.3}; 
        // Role Model Age
        double[] RMAf = new double[]{7.5, 17.6, 26.8, 31.0, 17.1};    
        double[] RMAm = new double[]{9.1, 16.9, 25.3, 34.4, 14.3};   

        // Role Model Income
        double[] RMIf = new double[]{43.1, 41.8, 9.7, 3.0, 1.6, 0.4, 0.1, 0.4}; 
        double[] RMIm = new double[]{42.9, 42.1, 10.4, 3.0, 0.7, 0.2, 0.5, 0.2};

        // Fear of Failuer Age
        double[] FFAf = new double[]{8.2, 16.4, 23.5, 32.4, 19.5};
        double[] FFAm = new double[]{7.2, 14.3, 23.6, 36.4, 18.6};

        // Fear of Failure Education
        double[] FFEf = new double[]{2.3, 13.9, 22.9, 51.7, 8.7, 0.5};
        double[] FFEm = new double[]{0.7, 12.1, 18.4, 57.4, 11.3, 0};

        // Fear of Failure Location
        double[] FFLf = new double[]{0.7, 10.3, 3.4, 3.8, 3.1, 4.2, 2.5, 36.9, 2.6, 7.8, 12.1, 4.6, 1.9, 1.0, 0.4, 4.8};
        double[] FFLm = new double[]{0.7, 8.9, 3.4, 2.4, 4.2, 5.4, 2.9, 35.4, 1.8, 7.1, 13.2, 4.7, 2.2, 1.0, 0.5, 6.1};

        // Media Attention Location
        double[] MALf = new double[]{0.7, 9.9, 3.6, 3.5, 5.4, 4.1, 1.9, 41.1, 6.3, 9.1, 4.7, 2.9, 1.1, 1.0, 0.7, 4.0};
        double[] MALm = new double[]{0.9, 8.1, 3.4, 3.6, 4.7, 4.9, 1.9, 41.8, 6.5, 8.2, 5.8, 2.9, 1.3, 1.1, 0.7, 4.1};

        // Media Attention Income
        double[] MAIf = new double[]{44.4, 41.6, 9.1, 2.7, 1.4, 0.2, 0.2, 0.4};
        double[] MAIm = new double[]{44.0, 40.0, 11.4, 3.1, 0.6, 0.2, 0.4, 0.2};

        // High Status Successful Income
        double[] HSSIf = new double[]{45.6, 41.7, 8.5, 2.2, 1.2, 0.3, 0.1, 0.3};
        double[] HSSIm = new double[]{46.2, 39.5, 10.2, 2.8, 0.6, 0.2, 0.4, 0.2};

        //High Status Successful Location
        double[] HSSLf = new double[]{0.7, 9.0, 2.7, 2.6, 5.4, 4.5, 1.8, 35.0, 9.8, 8.3, 10.8, 2.8, 1.5, 1.0, 0.8, 3.4};
        double[] HSSLm = new double[]{0.8, 7.2, 2.3, 3.2, 4.8, 4.5, 2.2, 37.0, 10.2, 7.6, 10.7, 2.9, 1.6, 1.1, 0.8, 3.1};

        // High Status Successful Age
        double[] HSSAf = new double[]{10, 17, 26, 31, 17};
        double[] HSSAm = new double[]{9, 16, 25, 33, 16};

        // High Status Successful Education
        double[] HSSEf = new double[]{2, 15, 23, 52, 8, 0};
        double[] HSSEm = new double[]{1, 12, 19, 56, 11, 0};

        // Faktor Publik
        double[] pfs = new double[]{3.06, 2.69, 2.22, 2.53, 2.54, 3.3, 2.31, 3.25, 3.92, 2.82, 3.45, 3.29};
        double[] pfw = InputDataHandler.getDataEksternal();

        double[] nw = InputDataHandler.getBobot();
        int[] nr = InputDataHandler.getRelation();

        ca.pub.setFactors(pfs);
        ca.pub.setWeights(pfw);

        ca.N.setWeight(nw);
        ca.N.setRelation(nr);

        int maxIter = Integer.parseInt(InputDataHandler.getValue("periode")); // masukan periode

        Entrepreneurs[][] e = new Entrepreneurs[maxIter][];
        String[] line = new String[maxIter];
        try {
            PrintWriter pw = new PrintWriter(new File("D:\\output.csv"));
            StringBuilder sb = new StringBuilder();
            for (int i = 0; i < maxIter; i++) {
                sb.append("Bulan ke-" + i);
                sb.append('\n');

                ca.NeighborhoodDefinition();
                ca.calculatePoint(POAm, POAf, POEm, POEf, POLm, POLf, POIm, POIf, PCAm, PCAf, PCEm, PCEf, PCLm, PCLf, PCIm, PCIf, RMAm, RMAf, RMIm, RMIf, FFAf, FFAm, FFEf, FFEm, FFLf, FFLm, MALf, MALm, MAIf, MAIm, HSSIf, HSSIm, HSSLf, HSSLm, HSSAf, HSSAm, HSSEf, HSSEm);
                Entrepreneurs[] nE;
                e[i] = ca.stateTransition(ca, composition);

                for (int j = 0; j < e[i].length; j++) {
                    sb.append(e[i][j].toString2());
                    sb.append('\n');

                }
                // perubahan disimpan dulu
                ca.E = e[i];
                // lalu baru diprint
                line[i] = ca.print(i);

            }
            pw.write(sb.toString());
            pw.close();
        } catch (FileNotFoundException ex) {
            Logger.getLogger(TampilanSimulasi.class.getName()).log(Level.SEVERE, null, ex);
        }
        this.hide();
        TampilanHasil th = new TampilanHasil(line);

        th.setVisible(true);
    }                                           

    /**
     * @param args the command line arguments
     */
    public static void main(String args[]) {
        /* Set the Nimbus look and feel */
        //<editor-fold defaultstate="collapsed" desc=" Look and feel setting code (optional) ">
        /* If Nimbus (introduced in Java SE 6) is not available, stay with the default look and feel.
         * For details see http://download.oracle.com/javase/tutorial/uiswing/lookandfeel/plaf.html 
         */
        try {
            for (javax.swing.UIManager.LookAndFeelInfo info : javax.swing.UIManager.getInstalledLookAndFeels()) {
                if ("Nimbus".equals(info.getName())) {
                    javax.swing.UIManager.setLookAndFeel(info.getClassName());
                    break;

                }
            }
        } catch (ClassNotFoundException ex) {
            java.util.logging.Logger.getLogger(TampilanSimulasi.class
                    .getName()).log(java.util.logging.Level.SEVERE, null, ex);
        } catch (InstantiationException ex) {
            java.util.logging.Logger.getLogger(TampilanSimulasi.class
                    .getName()).log(java.util.logging.Level.SEVERE, null, ex);
        } catch (IllegalAccessException ex) {
            java.util.logging.Logger.getLogger(TampilanSimulasi.class
                    .getName()).log(java.util.logging.Level.SEVERE, null, ex);
        } catch (javax.swing.UnsupportedLookAndFeelException ex) {
            java.util.logging.Logger.getLogger(TampilanSimulasi.class
                    .getName()).log(java.util.logging.Level.SEVERE, null, ex);
        }
        //</editor-fold>

        /* Create and display the form */
        java.awt.EventQueue.invokeLater(new Runnable() {
            public void run() {
//                new TampilanSimulasi(this.ca).setVisible(true);
            }
        });
    }

    // Variables declaration - do not modify                     
    private javax.swing.JLabel jLabel1;
    private javax.swing.JLabel jLabel2;
    private javax.swing.JLabel jLabel3;
    private javax.swing.JLabel jLabel4;
    private javax.swing.JLabel jLabel5;
    private javax.swing.JLabel jLabel6;
    private javax.swing.JLabel jLabel7;
    private javax.swing.JLabel jLabel8;
    private javax.swing.JTextField nilaiA;
    private javax.swing.JTextField nilaiB;
    private javax.swing.JTextField nilaiC;
    private javax.swing.JTextField nilaiPeriode;
    private javax.swing.JTextField nilaiThreshold;
    public javax.swing.JButton simulateButton;
    // End of variables declaration                   
}

\end{lstlisting}

\begin{lstlisting}[language=Java, caption=TampilanHasil.java]
/*
 * To change this license header, choose License Headers in Project Properties.
 * To change this template file, choose Tools | Templates
 * and open the template in the editor.
 */
package ecasimulatorjframe;

import javax.swing.table.DefaultTableModel;

/**
 *
 * @author Vanessa
 */
public class TampilanHasil extends javax.swing.JFrame {

    /**
     * Creates new form TampilanHasil
     */
    String[] res;

    public TampilanHasil(String[] res) {
        initComponents();
        this.res = res;
    }

    /**
     * This method is called from within the constructor to initialize the form.
     * WARNING: Do NOT modify this code. The content of this method is always
     * regenerated by the Form Editor.
     */
    @SuppressWarnings("unchecked")
    // <editor-fold defaultstate="collapsed" desc="Generated Code">                          
    private void initComponents() {

        jScrollPane1 = new javax.swing.JScrollPane();
        jTable1 = new javax.swing.JTable();
        jLabel1 = new javax.swing.JLabel();

        setDefaultCloseOperation(javax.swing.WindowConstants.EXIT_ON_CLOSE);
        addWindowListener(new java.awt.event.WindowAdapter() {
            public void windowActivated(java.awt.event.WindowEvent evt) {
                formWindowActivated(evt);
            }
            public void windowOpened(java.awt.event.WindowEvent evt) {
                formWindowOpened(evt);
            }
        });

        jTable1.setModel(new javax.swing.table.DefaultTableModel(
            new Object [][] {

            },
            new String [] {
                "Iterasi", "Potential", "Nascent", "New Business Manager", "Established", "Retired"
            }
        ));
        jScrollPane1.setViewportView(jTable1);

        jLabel1.setFont(new java.awt.Font("Tahoma", 0, 18)); // NOI18N
        jLabel1.setText("HASIL SIMULASI");

        javax.swing.GroupLayout layout = new javax.swing.GroupLayout(getContentPane());
        getContentPane().setLayout(layout);
        layout.setHorizontalGroup(
            layout.createParallelGroup(javax.swing.GroupLayout.Alignment.LEADING)
            .addGroup(layout.createSequentialGroup()
                .addGroup(layout.createParallelGroup(javax.swing.GroupLayout.Alignment.LEADING)
                    .addGroup(layout.createSequentialGroup()
                        .addGap(23, 23, 23)
                        .addComponent(jScrollPane1, javax.swing.GroupLayout.PREFERRED_SIZE, 632, javax.swing.GroupLayout.PREFERRED_SIZE))
                    .addGroup(layout.createSequentialGroup()
                        .addGap(267, 267, 267)
                        .addComponent(jLabel1)))
                .addContainerGap(21, Short.MAX_VALUE))
        );
        layout.setVerticalGroup(
            layout.createParallelGroup(javax.swing.GroupLayout.Alignment.LEADING)
            .addGroup(javax.swing.GroupLayout.Alignment.TRAILING, layout.createSequentialGroup()
                .addGap(29, 29, 29)
                .addComponent(jLabel1)
                .addPreferredGap(javax.swing.LayoutStyle.ComponentPlacement.RELATED, 34, Short.MAX_VALUE)
                .addComponent(jScrollPane1, javax.swing.GroupLayout.PREFERRED_SIZE, javax.swing.GroupLayout.DEFAULT_SIZE, javax.swing.GroupLayout.PREFERRED_SIZE)
                .addGap(27, 27, 27))
        );

        pack();
    }// </editor-fold>                        

    private void formWindowActivated(java.awt.event.WindowEvent evt) {                                     

    }                                    

    private void formWindowOpened(java.awt.event.WindowEvent evt) {                                  
        DefaultTableModel model = (DefaultTableModel) jTable1.getModel();
        for (int i = 0; i < res.length; i++) {
            String[] dataRow1 = res[i].split(",");
            model.addRow(dataRow1);
        }
    }                                 

    /**
     * @param args the command line arguments
     */
    public static void main(String args[]) {
        /* Set the Nimbus look and feel */
        //<editor-fold defaultstate="collapsed" desc=" Look and feel setting code (optional) ">
        /* If Nimbus (introduced in Java SE 6) is not available, stay with the default look and feel.
         * For details see http://download.oracle.com/javase/tutorial/uiswing/lookandfeel/plaf.html 
         */
        try {
            for (javax.swing.UIManager.LookAndFeelInfo info : javax.swing.UIManager.getInstalledLookAndFeels()) {
                if ("Nimbus".equals(info.getName())) {
                    javax.swing.UIManager.setLookAndFeel(info.getClassName());
                    break;
                }
            }
        } catch (ClassNotFoundException ex) {
            java.util.logging.Logger.getLogger(TampilanHasil.class.getName()).log(java.util.logging.Level.SEVERE, null, ex);
        } catch (InstantiationException ex) {
            java.util.logging.Logger.getLogger(TampilanHasil.class.getName()).log(java.util.logging.Level.SEVERE, null, ex);
        } catch (IllegalAccessException ex) {
            java.util.logging.Logger.getLogger(TampilanHasil.class.getName()).log(java.util.logging.Level.SEVERE, null, ex);
        } catch (javax.swing.UnsupportedLookAndFeelException ex) {
            java.util.logging.Logger.getLogger(TampilanHasil.class.getName()).log(java.util.logging.Level.SEVERE, null, ex);
        }
        //</editor-fold>

        /* Create and display the form */
        java.awt.EventQueue.invokeLater(new Runnable() {
            public void run() {
//                new TampilanHasil().setVisible(true);
            }
        });

    }

    // Variables declaration - do not modify                     
    private javax.swing.JLabel jLabel1;
    private javax.swing.JScrollPane jScrollPane1;
    private javax.swing.JTable jTable1;
    // End of variables declaration                   

}

\end{lstlisting}
%\lstinputlisting[language=Java, caption=MyCode.java]{./Lampiran/MyCode.java} 

