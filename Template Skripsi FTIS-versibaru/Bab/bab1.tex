%versi 2 (8-10-2016) 
\chapter{Pendahuluan}
\label{chap:intro}
   
\section{Latar Belakang}
\label{sec:label}

Pada saat ini, lapangan kerja pada suatu negara tidak bisa diprediksi, tetapi kenyataannya lapangan kerja dari tahun ke tahun semakin terbatas \cite{LBwirausaha}. Dengan melihat situasi tersebut maka bisa dipastikan tingkat pengangguran di suatu negara akan semakin tinggi. Solusi terbaik untuk mengurangi permasalahan tersebut adalah dengan berwirausaha. Kewirausahaan adalah kemampuan seseorang untuk membuat suatu usaha yang dimulai dari 0 yang dirintis hingga usaha tersebut benar-benar sukses. Tentu saja hal ini memberikan pengaruh positif terhadap pertumbuhan ekonomi suatu negara, karena kewirausahaan juga sekaligus membuka lapangan kerja bagi masyarakat. Jika usaha yang dirintis semakin besar, otomatis perusahaan tersebut akan merekrut tenaga kerja yang semakin banyak lagi. 

 
Pada zaman sekarang, sudah banyak sekali orang yang lebih memilih untuk berwirausaha daripada bekerja di kantor atau di sebuah perusahaan. Alasan mengapa banyak orang lebih memilih berwirausaha pun bervariasi contohnya orang tersebut tidak terlalu menyukai waktu kerjanya diatur oleh orang lain melainkan ia lebih menyukai waktu kerjanya diatur oleh dirinya sendiri. Tidak hanya pada jaman sekarang, dari jaman dahulu juga sudah ada wirausahawan yang namanya tidak asing lagi didengar oleh telinga kita salah satunya yaitu Bob Sadino. Untuk menjadi wirausahawan yang sukses seperti Bob Sadino tidaklah mudah, pasti ada beberapa faktor dari luar maupun dalam yang mempengaruhi keberlangsungan wirausaha. Dalam berwirausaha dibutuhkan usaha yang besar untuk menjadi sukses, usaha tersebut juga harus dijaga kekonsistenannya agar tidak mengalami kebangkrutan.


Kewirausahaan sangat diperlukan guna mendorong perekonomian suatu negara karena dapat mengurangi tingkat pengangguran di Indonesia. Secara ekonomis, kewirausahaan akan membantu meningkatkan pendapatan masyarakat atau meningkatkan kesejahteraan melalui penciptaan produk baru, serta mengurangi kemiskinan.  Ideal besarnya populasi wirausaha dalam suatu negara adalah 2\% dari total penduduk suatu negara. Saat ini Indonesia baru mencapai 1.5\% pengusaha dari total penduduk \cite{ECA}. Maka dari itu, kondisi wirausaha ini perlu dipantau terus-menerus perkembangannya agar dapat memajukan perekonomian di Indonesia. Pemantauan ini dilakukan oleh pemerintah dan lembaga-lembaga swasta yang berkepentingan. Salah satu lembaga yang memantau kewirausahaan adalah GEM (Global Entrepreneurship Monitor). GEM merupakan konsorsium yang bertujuan untuk mengukur dan memantau kegiatan kewirausahaan. 


Selain pemantauan terhadap kondisi riil, salah satu kegiatan yang mendukung pemantauan adalah pengamatan secara tidak langsung. Salah satu pengamatan tidak langsung adalah dengan membuat model matematika dari pertumbuhan wirausaha dan kemudian melakukan simulasi terhadap model tersebut. Salah satu model matematika yang dapat digunakan untuk memodelkan pertumbuhan wirausaha adalah \textit{Entrepreneurial Cellular Automata} (ECA) yang diusulkan oleh Nugraheni dan Natali \cite{ECA}. ECA adalah pengembangan dari \textit{Cellular Automata} standar dari Ulam dan von Neumann. \textit{Cellular Automata} (CA) sendiri merupakan suatu model matematika yang digunakan untuk memodelkan suatu sistem dinamis. Pada \cite{ECA} dijelaskan bagaimana struktur dari ECA dan diberikan illustrasi bagaimana menggunakan ECA untuk memprediksi pertumbuhan wirausaha berdasarkan parameter wirausaha dari GEM. 


Dalam hasil penelitian ECA setiap wirausahawan mempunyai beberapa atribut yang bersifat statis maupun dinamis. Contoh atribut yang bersifat statis yaitu bidang usaha, kategori usaha, lokasi geografis dan jenis kelamin. Sementara contoh untuk atribut dinamis adalah usia, level wirausaha dan usia usaha. Diantara atribut dinamis, level wirausaha menjadi atribut penting karena atribut ini yang akan menjadi acuan untuk menentukan perkembangan dari kewirausahaan. \textit{Continuity Index} digunakan untuk menentukan apakah seorang wirausahawan pada suatu saat tertentu akan meneruskan usahanya pada waktu selanjutnya.


Akan tetapi ECA yang telah dibuat belum bisa menggambarkan kepada \textit{user} awam atau pemantau tentang parameter mana saja yang dibutuhkan untuk dapat melihat pertumbuhan wirausaha dalam waktu tertentu. Selain untuk melihat pertumbuhan wirausaha, dengan adanya simulator pemantau juga dapat mengetahui faktor apa saja yang paling berpengaruh dalam menaikkan pertumbuhan wirausaha dan faktor apa saja yang membuat pertumbuhan wirausaha menurun.
Skripsi ini bertujuan untuk membangun sebuah simulator ECA dengan memperhitungkan beberapa parameter yang belum diperhatikan pada ECA dan menampilkan hasil simulasi dalam bentuk tabel.



\section{Rumusan Masalah}
\label{sec:rumusan}
Berikut adalah susunan permasalahan yang akan dibahas pada penelitian ini:


\begin{enumerate}
	\item Faktor apa saja yang mempengaruhi keberlangsungan wirausaha?
	\item Bagaimana memodelkan pertumbuhan wirausaha dengan \textit{Entrepreneurial Cellular Automata}?
	\item Bagaimana membangun simulator keberlangsungan wirausaha dengan \textit{Entrepreneurial Cellular Automata}?
\end{enumerate}



\section{Tujuan}
\label{sec:tujuan}
Berdasarkan rumusan masalah yang telah dibuat, maka tujuan penelitian ini dijelaskan ke dalam poin-poin sebagai berikut :


\begin{enumerate}
	\item Mempelajari faktor yang berpengaruh pada keberlangsungan wirausaha.
	\item Memodelkan pertumbuhan wirausaha dengan \textit{ Entrepreneurial Cellular Automata}.
	\item Membangun simulator keberlangsungan wirausaha dengan \textit{Entrepreneurial Cellular Automata}.
\end{enumerate}

\section{Batasan Masalah}
\label{sec:batasan}
\begin{enumerate}
	\item Tidak bertujuan untuk menguji kualitas atau kebenaran dari ECA, tetapi hanya membangun simulator untuk ECA saja.
	\item Perangkat lunak yang dibuat hanya bisa dijalankan pada komputer / \textit{laptop}.
	\item Hanya mempelajari perkembangan wirausaha dari GEM.
	\item Data bersifat statis artinya hanya menangani wirausaha yang sudah ada, tidak menangani wirausaha baru yang muncul pada saat periode tertentu. 
	\item Data wirausaha yang diuji bukan data nyata.
	\item Nilai-nilai beberapa konstanta yang digunakan pada simulator didasarkan pada data dari GEM.
\end{enumerate}


\section{Metodologi}
\label{sec:metlit}
Langkah-langkah yang akan dijalani untuk menyelesaikan penelitian ini :
\begin{enumerate}
	\item Melakukan studi pustaka untuk hal-hal berikut :
		\begin{enumerate}
			\item \textit{Cellular Automata} khususnya ECA
			\item Kewirausahaan khususnya GEM
		\end{enumerate}
	\item Menganalisis masalah kewirausahaan untuk membangun simulator pertumbuhan wirausaha menggunakan \textit{Entrepreneurial Cellular Automata}.
	\item Merancang perangkat lunak berdasarkan hasil pemodelan.
	\item Mengimplementasikan perangkat lunak sesuai rancangan.
	\item Menguji perangkat lunak yang dibuat.
	\item Menulis dokumen skripsi.
\end{enumerate}


\section{Sistematika Pembahasan}
\label{sec:sispem}
Setiap bab dalam penelitian ini memiliki sistematika penulisan yang dijelasan ke dalam poin-poin sebagai berikut :
\begin{enumerate}
	\item Bab 1: Pendahuluan berisi latar belakang masalah, rumusan masalah, tujuan, batasan masalah, metodologi dan sistematika pembahasan.
	\item Bab 2: Landasan Teori yaitu akan membahas mengenai arti kewirausahaan, penjelasan \textit{cellular automata}, penjelasan \textit{entrepreneurial cellular automata} dan penjelasan tentang graf.
	\item Bab 3: Analisis, yaitu berisi analisis pertumbuhan wirausaha, analisis pemodelan \textit{entrepreneurial cellular automata}, analisis model pertumbuhan wirausaha dengan \textit{entrepreneurial cellular automata}, deskripsi perangkat lunak dan analisis perangkat lunak.
	\item Bab 4: Perancangan, membahas mengenai diagram kelas, rancangan antarmuka dan rancangan \textit{file} input.
	\item Bab 5: Implementasi dan Pengujian, pada bab ini berisi tentang implementasi, hasil implementasi dan contoh simulasi. Dalam pengujian akan dijelaskan pengujian fungsional, pengujian pembacaan parameter, pengujian pembacaan \textit{file} dan pengujian hasil dari simulasi.
	\item Bab 6: Kesimpulan dan Saran, yaitu membahas hasil kesimpulan dari keseluruhan penelitian ini dan saran-saran yang dapat diberikan untuk penelitian berikutnya.
\end{enumerate}



