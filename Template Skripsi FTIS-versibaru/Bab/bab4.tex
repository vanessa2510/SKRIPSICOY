\chapter{Perancangan}
\label{chap:perancangan}

Pada bab ini akan dijelaskan perancangan mengenai simulator yang akan dibangun untuk pertumbuhan wirausaha. Perancangan yang dibuat akan meliputi diagram kelas beserta penjelasannya, rancangan antarmuka dari perangkat lunak, serta rancangan \textit{file} input.


\section{Diagram Kelas}
\label{sec:perancangankelas}

Dalam membuat simulator diperlukan sebuah GUI atau Interface untuk bisa menggambarkan kinerja suatu sistem. Berdasarkan diagram kelas pada bab analisis \ref{fig:CD1}, dibuatlah perubahan diagram kelas rinci yang dibuat oleh penulis untuk memenuhi kebutuhan dalam membangun simulator. Perubahan pada diagram kelas berupa penambahan kelas baru yaitu kelas InputDataHandler dan perubahan pada kelas ECA dan kelas Entrepreneurs. Deskripsi kelas beserta fungsinya akan dijelaskan pada subbab selanjutnya. (Gambar \ref{fig:classdiagram2})

\subsection{Kelas CA}
Pada kelas CA, \textit{method} \texttt{calculatePoint()} di kelas diagram \ref{fig:classdiagram2}tidak dituliskan parameternya karena sangat banyak, maka dari itu akan dijelaskan lebih lanjut pada nomor ke 3. Dilakukan perubahan pada tiga \textit{method} di kelas CA yaitu :
\begin{itemize}
	\item \texttt{public Entrepreneur[] stateTransition(CA model, double[] composition)}\\
	Perubahan yang dilakukan adalah pada saat menambahkan umur wirausaha. Umur wirausaha akan ditambah jika bulannya sudah mencapai 12 bulan atau kelipatan 12 bulan. Dilakukan perubahan agar pada setiap iterasi (bulan), umur wirausaha tidak bertambah secara terus-menerus melainkan ditambah pada saat sudah 1 tahun (12 bulan).
	\item \texttt{public void NeighborhoodDefinition() }\\
	Perubahan yang dilakukan adalah penambahan pada faktor (umur, pendidikan, pendapatan dan jenis kelamin) dan relasi (lebih dari sama dengan).
	\item \texttt{public void calculatePoint(double[] POAm, double[] POAf, double[] POEf, double[] POEm, double[] POLm, double[] POLf, double[] POIm, double[] POIf, double[] PCAf, double[] PCAm, double[] PCEm, double[] PCEf, double[] PCLm, double[] PCLf, double[] PCIm, double[] PCIf, double[] RMAm, double[] RMAf, double[] RMIm, double[] RMIf, double[] FFAf, double[] FFAm, double[] FFEf, double[] FFEm, double[] FFLf, double[] FFLm, double[] MALf, double[] MALm, double[] MAIf, double[] MAIm, double[] HSSIf, double[] HSSIm, double[] HSSLf, double[] HSSLm, double[] HSSAf, double[] HSSAm, double[] HSSEf, double[] HSSEm)}\\
		Perubahan yang dilakukan adalah penambahan pada indikator yang mendukung intensi masyarakat untuk memulai usaha. Indikator-indikator tersebut yaitu Entrepreneurial Intentions (High Status Successful Entrepreneurship, Media Attention) dan Fear of Failure.
\end{itemize}

\begin{figure} [H]
	\centering  
	\includegraphics[width=18cm, height=14cm]{diagramKelas1}
	\caption[Diagram Kelas Simulator ECA]{Diagram Kelas Simulator ECA} 
	\label{fig:classdiagram2} 
\end{figure}






\subsection{Kelas Tampilan Bobot Ketetanggaan}
Kelas ini merupakan kelas untuk menampilkan seluruh atribut umum dari seorang wirausaha yang dapat dipilih menggunakan \textit{checkbox}, atribut yang dipilih nantinya akan mempengaruhi ketetanggaan antara wirausaha yang satu dengan wirausaha lainnya. Setelah itu, \textit{user} diminta mengisi bobot untuk masing-masing atribut yang sudah dichecklist melalui \textit{textfield}.

\subsection{Kelas Tampilan Kondisi Ketetanggaan}
Kelas ini merupakan kelas untuk menampilkan atribut yang sudah dipilih dari kelas TampilanKondisiInternal. \textit{User} dapat memilih atribut mana saja yang akan ditetapkan menjadi kondisi ketetanggaan untuk satu wirausaha ke wirausaha lainnya. Selain itu, \textit{user} diminta untuk mengisi hubungan ketetanggaan khusus untuk 4 atribut yaitu umur, level, pendapatan dan pendidikan jika \textit{user} men-checklist salah satu atau bahkan keempat-empatnya dari atribut tersebut. Untuk atribut jenis kelamin, lokasi usaha dan bidang usaha tidak dapat ditetapkan menjadi 3 jenis karena jenisnya hanya satu yaitu sama dengan. Alasan ketiga atribut tersebut tidak bisa ditetapkan menjadi 3 jenis karena ketiga atribut tersebut tidak bisa diurutkan atau dibandingkan seperti atribut a lebih besar dari atribut b.

\subsection{Kelas Tampilan Kondisi Eksternal}
Kelas ini merupakan kelas untuk menampilkan faktor eksternal yang mempengaruhi pertumbuhan wirausaha. Dalam kasus ini ditetapkan 12 faktor publik yaitu keuangan terkait dengan kewirausahaan, kebijakan pemerintah terkait ekonomi, kebijakan pemerintah terkait pajak, program pemerintah, pendidikan kewirausahaan pada SD dan SMP, pendidikan kewirausahaan pada SMK, professional dan universitas, transfer penelitian dan pengembangan, infrastruktur komersial dan legal, dinamika pasar, keterbukaan pasar, infrastruktur fisik dan akses layanan, serta norma sosial dan budaya. \textit{User} diminta untuk mengisi bobot untuk setiap faktor dan total dari semua bobot harus 100\%.

\subsection{Kelas Data Wirausaha}
Kelas ini merupakan kelas untuk membuka \textit{file} data wirausaha yang akan disimulasikan, lalu menampilkannya ke tabel. Isi datanya berupa :
\begin{enumerate}
	\item Jenis Kelamin
	\item Umur
	\item Usia Bisnis
	\item Kategori Usaha
	\item Subkategori
	\item Pendidikan
	\item Lokasi
	\item Pendapatan
	\item Level
	\item Point
\end{enumerate}

\subsection{Kelas Tampilan Simulasi}
Kelas ini berfungsi untuk mengisi nilai a, b, c, \textit{threshold} dan periode. Nilai a,b,c dan \textit{threshold} bertipe double, sedangkan periode bertipe integer. Periode ini dihitung dalam bulan. Kelas ini juga untuk menghitung \textit{Continuity Index} yang hasil iterasinya akan dikirim ke kelas TampilanHasil dalam bentuk tabel. Selain itu, kelas ini juga akan menampilkan hasil perubahan setiap individu wirausaha dalam setiap bulannya pada \textit{file} CSV.

\subsection{Kelas Tampilan Hasil}
Kelas ini berfungsi untuk menampilkan iterasi (per bulan) banyaknya wirausaha yang berada pada level tertentu dalam bentuk tabel. Untuk hasil keluaran yang dikeluarkan pada \textit{file} CSV dapat dibuka pada Microsoft Excel.

\subsection{Kelas Input Data Handler}
Kelas ini merupakan kelas untuk mengambil dan menyimpan data masukan dari \textit{user} yang nantinya akan dipakai untuk menghitung \textit{Continuity Index}.
Berikut penjelasan method-method yang ada di kelas InputDataHandler :
	\begin{itemize}
		\item \texttt{public static void inputDataInternal(String key, String value)}\\
		Berfungsi untuk menyimpan masukan pada kelas TampilanBobotKetetanggaan.\\
		Parameter :
		\begin{itemize}
			\item \texttt{key} merupakan kata kunci dari setiap masukan.
			\item \texttt{value} merupakan nilai dari kata kunci.
		\end{itemize}
		
		\item \texttt{public static boolean checkKey(String key)}\\
		Berfungsi untuk memeriksa isi nilai dari kata kunci. Return \textit{true} jika kata kunci tersebut mempunyai nilai. Return \textit{false} jika kata kunci tersebut tidak mempunyai nilai.\\
		Parameter :
		\begin{itemize}
			\item \texttt{key} merupakan kata kunci dari setiap masukan.
		\end{itemize}
		
		\item \texttt{public static void inputDataKetetanggaan(String key, String value)}\\
		Berfungsi untuk menyimpan masukan pada kelas TampilanKondisiKetetanggaan.\\
		Parameter :
		\begin{itemize}
			\item \texttt{key} merupakan kata kunci dari setiap masukan.
			\item \texttt{value} merupakan nilai dari kata kunci.
		\end{itemize}
		
		\item \texttt{public static void jmlCheckList()}\\
		Berfungsi untuk menambahkan jumlah \textit{checklist} pada kelas TampilanBobotKetetanggaan.
		
		\item \texttt{public static int getKetetanggaan()}\\
		Berfungsi untuk mengambil nilai ketetanggaan.
		
		\item \texttt{public static void inputDataEksternal(String key, String value)}\\
		Berfungsi untuk menyimpan masukan dari kelas TampilanKondisiEksternal.\\
		Parameter:
		\begin{itemize}
			\item \texttt{key} merupakan kata kunci dari setiap masukan.
			\item \texttt{value} merupakan nilai dari kata kunci.
		\end{itemize}
		
		\item \texttt{public static void setDataEksternal(double[] kumpulanNilaiPF)}\\
		Berfungsi untuk mengubah nilai-nilai dari faktor publik.\\
		Parameter:
		\begin{itemize}
			\item \texttt{kumpulanNilaiPF} merupakan kumpulan nilai faktor publik.
		\end{itemize}
		
		\item \texttt{public static double[] getDataEksternal()}\\
		Berfungsi untuk mengambil nilai-nilai dari faktor publik.
		
		\item \texttt{public static String getValue(String key)}\\
		Berfungsi untuk mengambil nilai dari kata kunci.\\
		Parameter:
		\begin{itemize}
			\item \texttt{key} merupakan kata kunci dari setiap masukan.
		\end{itemize}
		
		\item \texttt{public static void setBobot(double[] bobotAtribut)}\\
		Berfungsi untuk mengubah nilai-nilai bobot dari setiap atribut.\\
		Parameter:
		\begin{itemize}
			\item \texttt{bobotAtribut} merupakan kumpulan bobot dari setiap atribut.
		\end{itemize}
		
		\item \texttt{public static void getBobot()}\\
		Berfungsi untuk mengambil nilai dari bobot.
		
		\item \texttt{public static void setRelation(int[] nilaiRelation)}\\
		Berfungsi untuk mengubah nilai-nilai dari setiap relasi.\\
		Parameter:
		\begin{itemize}
			\item \texttt{nilaiRelation} merupakan kumpulan nilai dari setiap relasi.
		\end{itemize}
		
		\item \texttt{public static int[] getRelation()}\\
		Berfungsi untuk mengambil nilai dari setiap relasi.
		
		\item \texttt{public static void setPopulation(int jmlData)}\\
		Berfungsi untuk mengubah nilai dari populasi.\\
		Parameter:
		\begin{itemize}
			\item \texttt{jmlData} merupakan jumlah dari data masukan \textit{user}.
		\end{itemize}
		
		\item \texttt{public static int getPopulation()}\\
		Berfungsi untuk mengembalikan nilai dari populasi.
		
		\item \texttt{public static void inputDataSimulasi(String key, String value)}\\
		Berfungsi untuk menyimpan masukan dari kelas TampilanSimulasi.\\
		Parameter :
		\begin{itemize}
			\item \texttt{key} merupakan kata kunci dari setiap masukan.
			\item \texttt{value} merupakan nilai dari kata kunci.
		\end{itemize}
	\end{itemize}


\section{Rancangan Antarmuka}
\label{sec:rancanganantarmuka}

\subsection{Tampilan Bobot Ketetanggaan}

\begin{figure} [H]
	\centering  
	\includegraphics[width=11cm, height=12cm]{mockup1} 
	\caption[Gambar Tampilan Bobot Ketetanggaan]{Gambar Tampilan Bobot Ketetanggaan}
	\label{fig:kondisiInternal} 
\end{figure}

Dapat dilihat pada gambar \ref{fig:kondisiInternal}, pada kondisi awal terdapat 7 atribut umum (umur, level wirausaha, pendidikan, pendapatan, jenis kelamin, lokasi dan bidang usaha) dari seorang wirausahawan yang dapat dipilih oleh \textit{user} melalui \textit{checkbox}. \textit{User} dapat memilih lebih dari 1 atribut. Jika \textit{user} tidak mengisi \textit{checkbox} terlebih dahulu, \textit{user} tidak akan bisa mengisi bobot atribut. Atribut yang dipilih melalui \textit{checkbox}, akan menjadi atribut ketetanggaan dari wirausahawan satu dengan wirausahawan lainnya. Setelah \textit{user} memilih atribut wirausaha, \textit{user} harus mengisi bobot dari masing-masing atribut melalui \textit{text field}. Total dari bobot atribut yang dipilih jumlahnya harus 100\%. Jika \textit{user} tidak mengisi seluruh \textit{checkbox}, \textit{user} tidak akan bisa melanjutkan ke proses selanjutnya. Begitu juga jika \textit{user} tidak mengisi bobot berdasarkan atribut yang sudah dipilih, \textit{user} tidak dapat melanjutkan ke proses selanjutnya.

\subsection{Tampilan Kondisi Ketetanggaan}

\begin{figure} [H]
	\centering  
	\includegraphics[width=11cm, height=12cm]{mockup2} 
	\caption[Gambar Tampilan Kondisi Ketetanggaan]{Gambar Tampilan Kondisi Ketetanggaan}
	\label{fig:kondisiTetangga} 
\end{figure}

Dapat dilihat pada gambar \ref{fig:kondisiTetangga}, terdapat 7 atribut tetangga yang telah dipilih oleh \textit{user} pada kelas TampilanBobotKetetanggaan. Atribut yang ditampilkan pada kondisi ketetanggaan bergantung pada pemilihan atribut pada tampilan sebelumnya \ref{fig:kondisiInternal}. Contohnya jika \textit{user} memilih 3 atribut (level wirausaha, jenis kelamin dan lokasi) pada tampilan sebelumnya, atribut yang akan ditampilkan pada kondisi ketetanggaan hanya 3 atribut (level wirausaha, jenis kelamin dan lokasi). Pada tampilan ini \textit{user} diminta untuk mengisi relasi ketetanggaan khususnya pada atribut umur, level, pendapatan dan pendidikan. 3 atribut lainnya tidak terdapat relasi ketetanggaan, hal ini dikarenakan ketiga atribut tersebut tidak bisa dibanding-bandingkan. Contohnya seperti lokasi, wirausaha A membangun usahanya di kota Jakarta, sedangkan wirausaha B membangun usahanya di kota Bandung. Tentu saja hal ini tidak dapat ditetapkan sebagai kota Jakarta lebih dari kota Bandung atau kota Bandung kurang dari kota Jakarta.

\subsection{Tampilan Kondisi Eksternal}

\begin{figure} [H]
	\centering  
	\includegraphics[width=11cm, height=13cm]{mockup3} 
	\caption[Gambar Tampilan Kondisi Eksternal]{Gambar Tampilan Kondisi Eksternal}
	\label{fig:kondisiEksternal} 
\end{figure}

Pada tampilan kondisi eksternal terdapat 12 faktor publik yang mempengaruhi pertumbuhan wirausaha di Indonesia. Keduabelas faktor publik ini didapatkan dari data GEM 2013. Untuk keduabelas faktor ini, \textit{user} harus mengisi bobot setiap faktor publik yang total bobotnya harus 100\%. Untuk memeriksa jumlah masukan bobot, \textit{user} bisa melihatnya di total. 

\subsection{Tampilan Data Wirausaha}

\begin{figure} [H]
	\centering  
	\includegraphics[width=16cm, height=12cm]{mockup4-1} 
	\caption[Gambar Tampilan Data Wirausaha]{Gambar Tampilan Data Wirausaha}
	\label{fig:kondisiDataWirausaha} 
\end{figure}

Pada tampilan ini, \textit{user} akan memasukkan \textit{file} masukan data wirausaha dalam format teks, rancangan \textit{file input} akan dibahas pada subbab \ref{rancanganFile}. Setelah \textit{file} dipilih, data wirausaha akan ditampilkan pada tabel.
\subsection{Tampilan Simulasi}

\begin{figure} [H]
	\centering  
	\includegraphics[width=12cm, height=7cm]{mockup5} 
	\caption[Gambar Tampilan Simulasi]{Gambar Tampilan Simulasi}
	\label{fig:simulasi} 
\end{figure}

Pada tampilan simulasi, \textit{user} diminta untuk mengisi nilai a,b,c, \textit{threshold} dan periode. Nilai-nilai tersebut digunakan untuk menghitung CIDx wirausahawan. Total dari nilai a,b dan c harus 1. Periode merupakan berapa lama iterasi tersebut akan berjalan (dalam bulan). Sedangkan \textit{button} "SIMULATE" berfungsi untuk menjalankan simulasi yang hasilnya akan ditampilkan dalam bentuk tabel.

\subsection{Tampilan Hasil}

Pada tampilan hasil, akan ditampilkan hasil simulasi berupa tabel, yang masing-masing kolomnya berisi iterasi (bulan), jumlah wirausaha pada level \textit{potential}, jumlah wirausaha pada level \textit{nascent}, jumlah wirausaha pada level \textit{new\_bm}, jumlah wirausaha pada level \textit{est\_bm}, jumlah wirausaha pada level \textit{retired}. Untuk kolom pertama, di setiap barisnya akan berisi lamanya iterasi (dalam bulan). Contoh periode 5 bulan maka setiap barisnya berisi bulan ke-0, bulan ke-1, bulan ke-2 sampai bulan ke-4. Iterasi dimulai dari bulan ke-0 yang artinya bulan ke-1, bulan ke-1 yang artinya bulan ke-2, dst.
\begin{figure} [H]
	\centering  
	\includegraphics[width=13cm, height=11cm]{mockup6} 
	\caption[Gambar Tampilan Hasil]{Gambar Tampilan Hasil}
	\label{fig:tampilanHasil} 
\end{figure}

\section{Rancangan File Input}
\label{rancanganFile}
Perancangan \textit{file} input yang akan disimulasikan terdiri dari jenis kelamin, umur, kategori usaha, sub kategori usaha, pendidikan, lokasi, pendapatan, level wirausaha dan point untuk setiap barisnya.
Tipe dari masing-masing atribut yaitu :
\begin{enumerate}
	\item Jenis kelamin bertipe boolean.
	\begin{itemize}
		\item True untuk pria
		\item False untuk wanita
	\end{itemize}
	\item Umur bertipe bilangan bulat (dalam tahun).
	\item Usia bisnis bertipe bilangan bulat (dalam bulan).
	\item Kategori usaha bertipe bilangan bulat, masing-masing angka mendeskripsikan kategori usaha yang berbeda, yaitu :
		\begin{itemize}
			\item 0 untuk makanan
			\item 1 untuk minuman
			\item 2 untuk tas
			\item 3 untuk pakaian
		\end{itemize}
	\item Sub kategori usaha bertipe bilangan bulat.
		\begin{itemize}
			\item Kategori makanan :
				\begin{itemize}
					\item 0 untuk makanan ringan
					\item 1 untuk makanan berat
					\item 2 untuk makanan cepat saji
				\end{itemize}
			\item Kategori minuman :
				\begin{itemize}
					\item 0 untuk minuman sehat
					\item 1 untuk minuman bersoda
					\item 2 untuk minuman \textit{sachet}
				\end{itemize}
			\item Kategori tas :
				\begin{itemize}
					\item 0 untuk tas pria
					\item 1 untuk tas anak-anak
					\item 2 untuk tas wanita
				\end{itemize}
		\end{itemize}
		\item Pendidikan bertipe bilangan bulat, masing-masing angka mendeskripsikan tingkat pendidikan yang berbeda, yaitu :
		\begin{itemize}
					\item 0 untuk tingkat pendidikan rendah
					\item 1 untuk sekolah dasar
					\item 2 untuk sekolah menengah pertama
					\item 3 untuk sekolah menengah ke atas
					\item 4 untuk sarjana (S1)
					\item 5 untuk diploma (S2)
					\item 6 untuk profesor (S3)
				\end{itemize}
		\item Lokasi, bertipe bilangan bulat yang masing-masing angkanya mendeskripsikan lokasi yang berbeda, yaitu :
			\begin{itemize}
				\item 0 untuk Banda Aceh
				\item 1 untuk Medan
				\item 2 untuk Padang
				\item 3 untuk Pekanbaru
				\item 4 untuk Palembang
				\item 5 untuk Bandar Lampung
				\item 6 untuk Serang
				\item 7 untuk Jakarta
				\item 8 untuk Bandung
				\item 9 untuk Semarang dan Surakarta
				\item 10 untuk Surabaya
				\item 11 untuk Denpasar
				\item 12 untuk Mataram
				\item 13 untuk Kupang
				\item 14 untuk Pontianak
				\item 15 untuk Makassar
			\end{itemize}
		\item Pendapatan bertipe bilangan bulat, masing-masing angka mendeskripsikan tingkat pendapatan yang berbeda yaitu :
			\begin{itemize}
			\item 0 untuk pendapatan dibawah 3 juta rupiah
			\item 1 untuk pendapatan 3 juta rupiah sampai 5 juta rupiah
			\item 2 untuk pendapatan 5 juta rupiah sampai 7 juta rupiah
			\item 3 untuk pendapatan 7 juta rupiah sampai 9 juta rupiah
			\item 4 untuk pendapatan 9 juta rupiah sampai 11 juta rupiah
			\item 5 untuk pendapatan 11 juta rupiah sampai 13 juta rupiah
			\item 6 untuk pendapatan 13 juta rupiah sampai 15 juta rupiah
			\item 7 untuk pendapatan diatas 15 juta rupiah
			\end{itemize}
		\item Level, bertipe bilangan bulat, masing-masing angka mendeskripsikan level yang berbeda yaitu :
			\begin{itemize}
			\item 0 untuk level potential
			\item 1 untuk level nascent
			\item 2 untuk level new business manager
			\item 3 untuk level established 
			\item 4 untuk level retired
			\end{itemize}
		\item Point merupakan nilai dari kondisi internal individu wirausaha. Point mempunyai tipe data double.
\end{enumerate}
Berikut contoh untuk \textit{file input} :

\begin{figure} [H]
	\centering  
	\includegraphics[width=6cm, height=4cm]{formatFile} 
	\caption[Contoh Format File Data Wirausaha]{Contoh Format File Data Wirausaha}
	\label{fig:formatFile} 
\end{figure}

Berdasarkan contoh di atas \ref{fig:formatFile}, terdapat 3 data wirausahawan. Baris pertama berisi pria, berumur 18 tahun, usia bisnisnya 0 bulan, kategori atau bidang usahanya minuman, sub kategori minuman bersoda, pendidikan SMP, usahanya berlokasi di Medan, pendapatan 5 sampai 7 juta rupiah, level usahanya \textit{nascent} dan nilai dari kondisi internalnya masih 0.
