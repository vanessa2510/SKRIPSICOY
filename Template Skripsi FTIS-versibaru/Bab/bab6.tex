\chapter{Kesimpulan dan Saran}
\label{chap:kesimpulan}

Pada bab ini akan diberikan kesimpulan terhadap simulator yang telah dibuat, juga saran-saran untuk penelitian ini.

\section{Kesimpulan}
Berdasarkan hasil penelitian yang telah dilakukan, didapatkan kesimpulan-kesimpulan sebagai berikut :
\begin{enumerate}
	\item Faktor yang mempengaruhi keberlangsungan wirausaha terdiri dari faktor luar dan faktor dalam. Faktor luar dibagi menjadi dua yaitu faktor publik dan faktor tetangga.\\
Berikut akan dijelaskan secara detail :
	\begin{itemize}
		\item Faktor Dalam\\
		Faktor dalam berasal dari atribut wirausaha itu sendiri, atribut wirausaha dibagi menjadi dua macam yaitu atribut umum (jenis kelamin, umur, level wirausaha, pendapatan, pendidikan, bidang usaha, lokasi) dan atribut psikologis (\textit{Perceived Opportunities, Perceived Capabilities, Role Model, Fear of Failure, Entrepreneurial of Intention}).
		\item Faktor Luar
		\begin{itemize}
			\item Faktor publik\\
			Faktor publik terdiri dari:
			\begin{enumerate}
				\item Keuangan terkait dengan kewirausahaan
				\item Kebijakan pemerintah terkait ekonomi
				\item Kebijakan pemerintah terkait pajak
				\item Program Pemerintah
				\item Pendidikan kewirausahaan pada SD dan SMP
				\item Pendidikan kewirausahan pada SMK, professional dan universitas
				\item Transfer penelitian dan pengembangan
				\item Infrastruktur komersial dan legal
				\item Keterbukaan Pasar
				\item Norma, Sosial dan Budaya
				\item Infrastruktur Fisik dan Akses Layanan
				\item Dinamika Pasar
			\end{enumerate}
			\item Faktor tetangga\\
			Faktor tetangga berasal dari relasi individu wirausaha dengan wirausaha lainnya. Relasi tersebut yaitu lebih dari sama dengan, sama dengan dan kurang dari sama dengan. 
		\end{itemize}
	\end{itemize}
	\item Dalam memodelkan pertumbuhan wirausaha dengan \textit{Entrepreneur Cellular Automata} dibutuhkan beberapa proses yaitu :\\
	\begin{enumerate}
		\item Menyesuaikan data wirausaha yang diberikan dengan nilai masing-masing atribut yang ada di GEM 2013.
		\item Menghitung nilai \textit{Continuity Index} yang dibagi menjadi 3 bagian yaitu:\\
		\begin{itemize}
			\item Menghitung faktor dalam dengan cara menjumlahkan nilai atribut pada setiap atribut psikologis lalu dikali dengan bobot atribut psikologis.
		\end{itemize}
	\end{enumerate}
\end{enumerate}