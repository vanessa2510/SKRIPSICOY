\chapter{Analisis}
\label{chap:analisis}


Pada bab ini akan dilakukan analisis mengenai pembangunan simulator pertumbuhan wirausaha dengan \textit{Entrepreneurial Cellular Automata}. Pembahasan akan dimulai dari analisa pertumbuhan wirausaha di Indonesia yang menjadi pokok permasalahan. Lalu dari analisis ini akan dilanjutkan dengan analisis kebutuhan perangkat lunak agar mampu memodelkan pertumbuhan wirausaha di Indonesia.

\section{Analisis Pertumbuhan Wirausaha}
\label{sec:analisisPertumbuhanWirausaha}

Seperti yang sudah dijelaskan pada bab \ref{chap:teori}, kewirausahaan dalam negara berkembang seperti Indonesia memang sangat diperlukan untuk membantu meningkatkan pertumbuhan ekonomi. GEM melakukan penelitiannya berdasarkan :
\begin{enumerate}
	\item Keadaan ekonomi negara,
	\item Kemampuan dan motivasi individu serta cara pandang masyarakat mengenai wirausaha,
	\item Pertumbuhan kewirausahaan dan persaingan antar negara tentang seberapa inovatif usaha tersebut.
\end{enumerate}  

Kewirausahaan menurut GEM merupakan sebuah proses yang memiliki tahapan-tahapan yang berbeda. Tahapan yang pertama yaitu individu yang bisa melihat peluang baik dalam berwirausaha dan memiliki kemampuan untuk berwirausaha (wirausaha \textit{potential}). Kedua, individu yang sudah menjalankan usahanya dalam waktu kurang dari tiga bulan (wirausaha \textit{nascent}). Ketiga, individu yang sudah menjalankan usahanya selama lebih dari tiga bulan tetapi tidak lebih dari tiga setengah tahun (wirausaha \textit{new business manager}). Keempat, individu yang sudah menjalankan usahanya lebih dari tiga setengah tahun (wirausaha \textit{established}). Penjelasan lebih lanjut dapat dilihat pada bab \ref{chap:teori} subbab \ref{sec:ECA}. Digunakan \textit{new\_bm} untuk new business manager dan \textit{est\_bm} untuk established business. 


Dalam pertumbuhan wirausaha tentu ada beberapa faktor yang mempengaruhi keberlangsungan pertumbuhan wirausaha. Secara umum, atribut atau faktor yang mempengaruhi pertumbuhan wirausaha yaitu terbagi menjadi 2 jenis yaitu atribut statis dan dinamis. Atribut dinamis yaitu umur, level wirausaha dan usia usaha. Di antara atribut dinamis, level wirausaha menjadi atribut penting karena atribut ini yang akan dijadikan sebagai acuan untuk menentukan perkembangan dari kewirausahaan. Atribut statis yaitu bidang usaha, kategori usaha, jenis kelamin dan lokasi geografis. Sedangkan atribut secara psikologis menurut GEM yaitu Perceived Opportunities, Perceived Capabilities, Entrepreneurial Intention  (High Status of Successful dan Media Attention)dan Fear of Failure Rate. 


\section{Analisis Pemodelan Entrepreneurial Cellular Automata}
\label{analisisCA}
Pada penelitian ini akan menggunakan cellular automata berbasis graf. Hal ini dikarenakan jumlah wirausaha di Indonesia yang tidak sedikit, sebab jika menggunakan cellular automata satu atau dua dimensi jumlahnya terbatas.\\
Dalam ECA, sel akan direpresentasikan sebagai satu wirausaha, sedangkan ketetanggaan merepresentasikan hubungan antar wirausaha satu dengan wirausaha lainnya. Hubungan antar wirausaha ada 3 jenis yaitu kurang dari sama dengan, sama dengan dan lebih dari sama dengan.


Perubahan transisi dari individu wirausahawan dapat diketahui melalui angka yang disebut \textit{Continuity Index} (CIdx).

\begin{displaymath}
\label{RumusCIDx}
	CIdx_{i}(t) = a.Cint_{i}(t) + b.Cneg_{i}(t) + c.Cpub(t)
\end{displaymath}
dimana a,b,c merupakan bilangan riil sedemikian sehingga $0\leq a,b,c \leq 1$ dan $a+b+c=$ 1.0 dan $Cint_{i}(t)$ dan $Cneg_{i}(t)$ melambangkan kondisi internal dan kondisi ketetanggaan dari sebuah individu i pada saat t dan $Cpub(t)$ melambangkan kondisi publik pada saat t nilai dari $CIdx$ dari individu i pada saat t.\\
Kondisi internal dari wirausaha berisi atribut-atribut sebagai berikut :
\begin{enumerate}
	\item Atribut Umum
		\begin{enumerate}
			\item Umur
			\item Level Wirausaha
			\item Bidang Usaha
			\item Jenis Kelamin
			\item Pendidikan
			\item Pendapatan
			\item Lokasi
		\end{enumerate}
	\item Atribut Psikologis
		 \begin{enumerate}
			\item Perceived Opportunities
			\item Perceived Capabilities
			\item Role Model
			\item Entrepreneurial of Intention (High Status Successful Entrepreneurship dan Public Media Attention for Entrepreneurship)
			\item Fear of Failure
		 \end{enumerate}
		\end{enumerate}
		
Kondisi ketetanggaan wirausaha berisi tentang hasil relasi  wirausaha dengan wirausaha lainnya. Relasinya yaitu lebih dari sama dengan, kurang dari sama dengan dan sama dengan. Relasi antar wirausaha berdasarkan pada atribut umum dari wirausahawan yaitu umur, level wirausaha, pendapatan dan pendidikan.\\ Contoh relasi kurang dari sama dengan adalah wirausaha A mempunyai umur 24 tahun, sedangkan wirausaha B mempunyai umur 30 tahun. Artinya yaitu wirausaha A mempunyai relasi umur kurang dari sama dengan wirausaha B, hal ini dikarenakan umur wirausaha A kurang dari umur wirausaha B. Contoh relasi sama dengan adalah wirausaha A mempunyai level wirausaha \textit{nascent}, sedangkan wirausaha B mempunyai level wirausaha \textit{nascent}. Artinya wirausaha A memiliki relasi level sama dengan wirausaha B, hal ini dikarenakan mereka berada pada kedudukan level wirausaha yang sama. Contoh relasi lebih dari sama dengan adalah wirausaha A berada pada tingkat pendidikan SMP, sedangkan wirausaha B berada pada tingkat pendidikan SMA. Artinya wirausaha B memiliki relasi pendidikan lebih besar sama dengan wirausaha A, hal ini dikarenakan wirausaha B memiliki tingkat pendidikan lebih besar atau lebih tinggi dibandingkan wirausaha A.\\
Kondisi faktor publik berisi tentang :
\begin{enumerate}
			\item Keuangan terkait dengan kewirausahaan
			\item Kebijakan pemerintah terkait ekonomi
			\item Kebijakan pemerintah terkait pajak
			\item Program Pemerintah
			\item Pendidikan kewirausahaan pada SD dan SMP
			\item Pendidikan kewirausahan pada SMK, professional dan universitas
			\item Transfer penelitian dan pengembangan
			\item Infrastruktur komersial dan legal
			\item Keterbukaan Pasar
			\item Norma, Sosial dan Budaya
			\item Infrastruktur Fisik dan Akses Layanan
			\item Dinamika Pasar
		 \end{enumerate}
		
Seperti yang sudah dijelaskan bahwa nilai dari \textit{Continuity Index} sangat berpengaruh pada perubahan level wirausaha. Nilai dari \textit{Continuity Index} akan dievaluasi terlebih dahulu menggunakan tabel transisi wirausaha (\ref{tabelLW}). Pada tabel \ref{tabelLW} akan dijelaskan mengenai transisi level dengan menggunakan lambang-lambang \textit{CIdx, bl, a ,b} dan \textit{th} untuk menyatakan \textit{Continuity Index}, level , usia individu, usia usaha dan nilai ambang.
\begin{table}[H]
\centering
\caption{Transisi Level Wirausaha}
\begin{tabular}{|c|c|}
\hline
Waktu sekarang & Waktu berikutnya \\
\hline
\textit{bl} = potential, $ \textit{CIdx} < \textit{th}, \textit{a} < 64 \times 12$ & \textit{bl} = potential \\
\hline
\textit{bl} = potential, $\textit{CIdx} \geq \textit{th}, \textit{a} < 64 \times 12$ & \textit{bl} = nascent \\
\hline
\textit{bl} = potential, $\textit{a} \geq 64 \times 12$ & \textit{bl} = retired \\
\hline
\textit{bl} = nascent, $\textit{CIdx} < \textit{th}, \textit{a} <64 \times 12$ & \textit{bl} = potential \\
\hline
\textit{bl} = nascent, $\textit{CIdx} \geq \textit{th}, \textit{b} < 3$ & \textit{bl} = nascent \\
\hline
\textit{bl} = nascent, $\textit{a} \geq 64 \times 12$ & \textit{bl} = retired \\
\hline
\textit{bl} = new\_bm, $\textit{CIdx} < \textit{th}, \textit{a} < 64 \times 12$ & \textit{bl} = potential \\
\hline
\textit{bl} = new\_bm, $\textit{CIdx} \geq \textit{th}, \textit{b} < 42$ & \textit{bl} = potential \\
\hline
\textit{bl} = new\_bm, $\textit{a} \geq 64 \times 12$ & \textit{bl} = retired \\
\hline
\textit{bl} = est\_bm, $\textit{CIdx} < \textit{th}, \textit{a} < 64 \times 12$ & \textit{bl} = potential \\
\hline
\textit{bl} = est\_bm, $\textit{CIdx} \geq \textit{th}, \textit{a} < 64 \times 12$ & \textit{bl} = est\_bm \\
\hline
\textit{bl} = est\_bm, $\textit{a} \geq 64 \times 12$ & \textit{bl} = retired \\
\hline
\textit{bl} = retired, $\textit{a} \geq 64 \times 12$ & \textit{bl} = retired \\
\hline
\end{tabular}
\label{tabelLW}
\end{table}

Dapat disimpulkan bahwa perubahan level wirausaha bukan hanya ditentukan oleh nilai \textit{Continuity Index} melainkan ditentukan juga oleh umur wirausaha dan usia bisnisnya.

% Seorang wirausahawan akan meneruskan usahanya jika nilai CIdx nya memenuhi nilai ambang. Jika nilai dari \textit{Continuity Index} sudah sama atau lebih dari nilai ambang, level wirausaha akan berubah. Sebaliknya, jika nilai dari \textit{Continuity Index }kurang dari sama dengan nilai ambang, level wirausaha bisa saja berubah dan bisa saja tidak berubah.

\section{Analisis Model Pertumbuhan Wirausaha dengan Entrepreneurial Cellular Automata}
\label{analisismodelCA}

Analisis model pertumbuhan wirausaha bergantung terhadap nilai \textit{Continuity Index}, nilai ambang (\textit{threshold}), umur dan usia bisnis. Seperti yang sudah dijelaskan pada bab \ref{chap:teori}, \textit{Continuity Index} adalah angka yang menentukan seorang wirausaha akan meneruskan usahanya atau tidak. Sedangkan nilai ambang berfungsi untuk acuan (patokan) perubahan wirausaha dari waktu ke waktu. (Rumus CIDx : \ref{RumusCIDx}).


Untuk mempermudah pemahaman mengenai \textit{Continuity Index}, akan diberikan contoh simulasi dari data tidak real, yaitu terdapat nilai a = 0.5, b = 0.4 dan c = 0.1, nilai ambangnya 15, serta periodenya dalam waktu 5 bulan. Nilai dari faktor psikologis diasumsikan Perceived Opportunities bernilai 0.2, Perceived Capabilities bernilai 0.25, High Status of Successful bernilai 0.1, Public Media Attention bernilai 0.05, Role Model bernilai 0.3 dan Fear of Failure bernilai 0.1. Diasumsikan terdapat tiga wirausahawan dan berikut data dari masing-masing wirausaha :
				
\begin{table} [H]
\centering
\caption{Data wirausahawan}
\begin{tabular}{|c|p{1cm}|p{1cm}|p{1cm}|p{2cm}|p{2cm}|p{2cm}|p{2cm}|p{1cm}|c|}
\hline
& Jenis Kelamin & Umur & Usia Bisnis & Kategori & Sub Kategori & Pendidikan & Lokasi & \textit{Income} & Level\\
\hline
E1 & P & 18th & 0 bulan & Minuman & Minuman bersoda & SMP & Medan & 5-7jt & Nascent\\
\hline
E2 & W & 30th & 0 bulan & Tas & Tas anak-anak  & SMA & Pekanbaru & 3-5jt & New\_bm\\
\hline
E3 & W & 45th & 0 bulan & Makanan & Makanan berat & SD & Palembang & 7-9jt & New\_bm\\
\hline
\end{tabular}
\end{table}

Asumsi ketetanggaan antara wirausaha satu dengan wirausaha lainnya hanya 3 atribut yaitu :

\begin{table} [H]
\centering
\caption{Data Bobot Atribut}
\begin{tabular}{|c|c|}
\hline
Atribut & Bobot\\
\hline
Level Wirausaha & 30\% \\
\hline
Pendidikan & 40\% \\
\hline
Jenis Kelamin & 30\% \\
\hline
\end{tabular}
\end{table}

Masing-masing tetangga relasinya yaitu sama dengan.

	\begin{figure} [H]
		\centering  
		\includegraphics[width=16cm, height=12cm]{t=0} 
		\caption[Gambar ketetanggaan tiga entrepreneur pada saat awal]{Gambar ketetanggaan tiga entrepreneur pada saat awal} 
		\label{fig:t0} 
	\end{figure}


Dalam simulasi ini terdapat 12 faktor publik yaitu :

\begin{table} [H]
\centering
\caption{Faktor Publik}
\begin{tabular}{|c|c|}
\hline
Faktor Publik & Bobot\\
\hline
Keuangan terkait dengan kewirausahaan & 3.06 \\
\hline
Kebijakan pemerintah terkait ekonomi & 2.69 \\
\hline
Kebijakan pemerintah terkait pajak & 2.22 \\
\hline
Program Pemerintah & 2.53\\
\hline
Pendidikan kewirausahaan pada SD dan SMP & 2.54\\
\hline
Pendidikan kewirausahaan pada SMK, professional dan universitas & 3.3\\
\hline
Transfer penelitian dan pengembangan & 2.31\\
\hline
Infrastruktur komersial dan legal & 3.25\\
\hline
Dinamika Pasar & 3.92\\
\hline
Keterbukaan Pasar & 2.82\\
\hline
Infrastruktur fisik dan akses layanan & 3.45\\
\hline 
Norma sosial dan budaya & 3.29\\
\hline
\end{tabular}
\end{table}
	
	
Untuk perhitungan pada faktor eksternal :
\begin{multline}
	CIDx(eksternal) = 0.1 \times ((3.06 \times 0.1) + (2.69 \times 0.1) + (2.22 \times 0.1) + (2.53 \times 0.05) + (2.54 \times 0.1) + (3.3 \times 0.1) \\ + (2.31 \times 0.05) + (3.25 \times 0.05) + (3.92 \times 0.1) + (2.82 \times 0.05) + (3.45 \times 0.1) + (3.29 \times 0.1)) = 0.29925 / 12 = 0.0249375 
\end{multline}

Perhitungan CIDx(t=0)\\
\begin{multline}
	CIdx_{1}(t=0) = 0.5 \times (((14.3+4.4+19.9+11) \times 0.2) + ((14.7+17.4+5.4+10.5) \times 0.25) + ((14.3+10.4) \times 0.3) \\ + ((16+19+7.2+10.2) \times 0.1) + ((8.1+11.4) \times 0.05) + ((18.6+18.4+8.9) \times 0.1) ) + 0.4 \times (0 + 0 + 0)\\ + 0.0249375 = 20.09243
\end{multline}	

\begin{multline}
	CIdx_{2}(t=0) = 0.5 \times (((29.5+49.8+2.8+41.5) \times 0.2) + ((31.6+51.5+2.4+43) \times 0.25) + ((31+41.8) \times 0.3)\\ + ((31+52+2.6+41.7) \times 0.1) + ((3.5+41.6) \times 0.05) + ((32.4+51.7 + 3.8) \times 0.1)) + 0.4 \times ((\frac {1} {2} \times 0.3) + 0 +  (\frac {1} {2} \times 0.3))\\ + 0.0249375 = 51.3749
\end{multline}


\begin{multline}
	CIdx_{3}(t=0) = 0.5 \times (((17.7+17.4+1.4+2.8) \times 0.2) + ((16.1+15.4+3.1+2) \times 0.25) + ((17.6+3) \times 0.3)\\ + ((17+15+5.4+2.2) \times 0.1) + ((5.4+2.7) \times 0.05) + ((16.4+13.9) \times 0.1)) + 0.4 \times ((\frac {1} {2} \times 0.3) + 0 +  (\frac {1} {2} \times 0.3))\\ + 0.0249375 = 15.4374
\end{multline}

	\begin{figure} [H]
		\centering  
		\includegraphics[width=18cm, height=12cm]{t=0} 
		\caption[Gambar ketetanggaan tiga entrepreneur pada saat t = 0]{Gambar ketetanggaan tiga entrepreneur pada saat t = 0} 
		\label{fig:t0} 
	\end{figure}

Perhitungan CIDx (t=1)

\begin{multline}
	CIdx_{1}(t=1) = 0.5 \times (((14.3+4.4+19.9+11) \times 0.2) + ((14.7+17.4+5.4+10.5) \times 0.25) + ((14.3+10.4) \times 0.3) \\ + ((16+19+7.2+10.2) \times 0.1) + ((8.1+11.4) \times 0.05) + ((18.6+18.4+8.9) \times 0.1) ) + 0.4 \times (0 + 0 + 0)\\ +  0.0249375 = 20.09243
\end{multline}

\begin{multline}
	CIdx_{2}(t=1) = 0.5 \times (((29.5+49.8+2.8+41.5) \times 0.2) + ((31.6+51.5+2.4+43) \times 0.25) + ((31+41.8) \times 0.3)\\ + ((31+52+2.6+41.7) \times 0.1) + ((3.5+41.6) \times 0.05) + ((32.4+51.7 + 3.8) \times 0.1)) + 0.4 \times ((\frac {1} {2} \times 0.3) + 0 +  (\frac {1} {2} \times 0.3))\\ +  0.0249375 = 51.3749
\end{multline}

\begin{multline}
	CIdx_{3}(t=1) = 0.5 \times (((17.7+17.4+1.4+2.8) \times 0.2) + ((16.1+15.4+3.1+2) \times 0.25) + ((17.6+3) \times 0.3)\\ + ((17+15+5.4+2.2) \times 0.1) + ((5.4+2.7) \times 0.05) + ((16.4+13.9) \times 0.1)) + 0.4 \times ((\frac {1} {2} \times 0.3) + 0 +  (\frac {1} {2} \times 0.3))\\ +  0.0249375 = 15.4374
\end{multline}

	\begin{figure} [H]
		\centering  
		\includegraphics[width=18cm, height=12cm]{t=0} 
		\caption[Gambar ketetanggaan tiga entrepreneur pada saat t = 1]{Gambar ketetanggaan tiga entrepreneur pada saat t = 1} 
		\label{fig:t1} 
	\end{figure}

Perhitungan CIDx (t=2)


\begin{multline}
	CIdx_{1}(t=2) = 0.5 \times (((14.3+4.4+19.9+11) \times 0.2) + ((14.7+17.4+5.4+10.5) \times 0.25) + ((14.3+10.4) \times 0.3) \\ + ((16+19+7.2+10.2) \times 0.1) + ((8.1+11.4) \times 0.05) + ((18.6+18.4+8.9) \times 0.1) ) + 0.4 \times (0 + 0 + 0)\\ +  0.0249375 = 20.09243
\end{multline}

\begin{multline}
	CIdx_{2}(t=2) = 0.5 \times (((29.5+49.8+2.8+41.5) \times 0.2) + ((31.6+51.5+2.4+43) \times 0.25) + ((31+41.8) \times 0.3)\\ + ((31+52+2.6+41.7) \times 0.1) + ((3.5+41.6) \times 0.05) + ((32.4+51.7 + 3.8) \times 0.1)) + 0.4 \times ((\frac {1} {2} \times 0.3) + 0 +  (\frac {1} {2} \times 0.3))\\ +  0.0249375 = 51.3749
\end{multline}

\begin{multline}
	CIdx_{3}(t=2) = 0.5 \times (((17.7+17.4+1.4+2.8) \times 0.2) + ((16.1+15.4+3.1+2) \times 0.25) + ((17.6+3) \times 0.3)\\ + ((17+15+5.4+2.2) \times 0.1) + ((5.4+2.7) \times 0.05) + ((16.4+13.9) \times 0.1)) + 0.4 \times ((\frac {1} {2} \times 0.3) + 0 +  (\frac {1} {2} \times 0.3))\\ +  0.0249375 = 15.4374
\end{multline}

	\begin{figure} [H]
		\centering  
		\includegraphics[width=18cm, height=12cm]{t=0} 
		\caption[Gambar ketetanggaan tiga entrepreneur pada saat t = 2]{Gambar ketetanggaan tiga entrepreneur pada saat t = 2} 
		\label{fig:t2} 
	\end{figure}
	
Perhitungan CIDx (t=3)

\begin{multline}
	CIdx_{1}(t=3) = 0.5 \times (((14.3+4.4+19.9+11) \times 0.2) + ((14.7+17.4+5.4+10.5) \times 0.25) + ((14.3+10.4) \times 0.3)\\ + ((16+19+7.2+10.2) \times 0.1) + ((8.1+11.4) \times 0.05) + ((18.6+18.4+8.9) \times 0.1) ) + 0.4 \times (0 + 0 + \frac{2}{2} \times 0.3)\\ +  0.0249375 = 20.2124
\end{multline}

\begin{multline}
	CIdx_{2}(t=3) = 0.5 \times (((29.5+49.8+2.8+41.5) \times 0.2) + ((31.6+51.5+2.4+43) \times 0.25) + ((31+41.8) \times 0.3)\\ + ((31+52+2.6+41.7) \times 0.1) + ((3.5+41.6) \times 0.05) + ((32.4+51.7 + 3.8) \times 0.1)) + 0.4 \times ((\frac {1} {2} \times 0.3) + 0 +  (\frac {2} {2} \times 0.3))\\ +  0.0249375 = 51.4349
\end{multline}

\begin{multline}
	CIdx_{3}(t=3) = 0.5 \times (((17.7+17.4+1.4+2.8) \times 0.2) + ((16.1+15.4+3.1+2) \times 0.25) + ((17.6+3) \times 0.3)\\ + ((17+15+5.4+2.2) \times 0.1) + ((5.4+2.7) \times 0.05) + ((16.4+13.9) \times 0.1)) + 0.4 \times ((\frac {1} {2} \times 0.3) + 0 +  (\frac {2} {2} \times 0.3))\\ +  0.0249375 = 15.4974
\end{multline}

	\begin{figure} [H]
		\centering  
		\includegraphics[width=18cm, height=12cm]{t=3} 
		\caption[Gambar ketetanggaan tiga entrepreneur pada saat t = 3]{Gambar ketetanggaan tiga entrepreneur pada saat t = 3} 
		\label{fig:t3} 
	\end{figure}
	
	Perhitungan CIDx (t=4)

\begin{multline}
	CIdx_{1}(t=4) = 0.5 \times (((14.3+4.4+19.9+11) \times 0.2) + ((14.7+17.4+5.4+10.5) \times 0.25) + ((14.3+10.4) \times 0.3)\\ + ((16+19+7.2+10.2) \times 0.1) + ((8.1+11.4) \times 0.05) + ((18.6+18.4+8.9) \times 0.1) ) + 0.4 \times (0 + 0 + \frac{2}{2} \times 0.3)\\ +  0.0249375 = 20.48675
\end{multline}

\begin{multline}
	CIdx_{2}(t=4) = 0.5 \times (((29.5+49.8+2.8+41.5) \times 0.2) + ((31.6+51.5+2.4+43) \times 0.25) + ((31+41.8) \times 0.3)\\ + ((31+52+2.6+41.7) \times 0.1) + ((3.5+41.6) \times 0.05) + ((32.4+51.7 + 3.8) \times 0.1)) + 0.4 \times ((\frac {1} {2} \times 0.3) + 0 +  (\frac {2} {2} \times 0.3))\\ +  0.0249375 = 51.70925
\end{multline}

\begin{multline}
	CIdx_{3}(t=4) = 0.5 \times (((17.7+17.4+1.4+2.8) \times 0.2) + ((16.1+15.4+3.1+2) \times 0.25) + ((17.6+3) \times 0.3)\\ + ((17+15+5.4+2.2) \times 0.1) + ((5.4+2.7) \times 0.05) + ((16.4+13.9) \times 0.1)) + 0.4 \times ((\frac {1} {2} \times 0.3) + 0 +  (\frac {2} {2} \times 0.3))\\ +  0.0249375 = 15.77175
\end{multline}

	\begin{figure} [H]
		\centering  
		\includegraphics[width=18cm, height=12cm]{t=3} 
		\caption[Gambar ketetanggaan tiga entrepreneur pada saat t = 4]{Gambar ketetanggaan tiga entrepreneur pada saat t = 4} 
		\label{fig:t3} 
	\end{figure}
	
Jadi hasil dari simulasi ini adalah pada bulan pertama wirausaha 1 berada pada level \textit{nascent} dan wirausaha 2 dan 3 berada pada level \textit{new\_bm}. Bulan kedua dan ketiga masih sama, bulan keempat mengalami perubahan pada level wirausaha 1 yaitu dari \textit{nascent} berubah menjadi \textit{new\_bm} sehingga ketiga wirausaha pada bulan keempat berada pada level wirausaha yang sama, begitu juga pada bulan kelima.
	
\section{Deskripsi Perangkat Lunak}
\label{dpl}

Dalam skripsi ini penulis merancang sebuah simulator dari Entrepreneurial Cellular Automata (ECA) yang sebelumnya telah dikembangkan oleh Cecilia Esti Nugraheni dan Vania Natali \cite{ECA}. Simulator ini dinamakan Simulator Pertumbuhan Wirausaha Berbasis Cellular Automata.\\
Perangkat lunak ini dibuat untuk memberi gambaran kepada pemerintah atau lembaga umum mengenai pergerakkan wirausaha dalam waktu tertentu. Masukan dari simulator ECA ini yaitu berupa parameter-parameter simulasi yang terdiri dari bobot atribut, relasi antar wirausaha dan nilai a,b,c, \textit{threshold} dan periode. Proses yang dijalankan yaitu pada perhitungan \textit{Continuity Index} yang perhitungannya terbagi menjadi 3 tahap yaitu perhitungan pada faktor internal, perhitungan pada faktor ketetanggaan dan perhitungan pada faktor publik. Hasil keluaran dari simulator ini terdiri dari dua keluaran yaitu keluaran yang ditampilkan pada layar yang berupa jumlah wirausaha pada level tertentu yang ditampilkan per bulan, hasil keluaran kedua yaitu berupa perubahan setiap individu wirausaha dalam setiap bulannya pada \textit{file} CSV yang dapat dibuka pada Microsoft Excel.


\section{Analisis Perangkat Lunak}
\label{analisisPL}

\subsection{Diagram \textit{Use Case}}

Pada diagram \textit{use case} hanya terdapat satu aktor yaitu pemerintah sebagai \textit{user}. Diagram \textit{use case} dapat dilihat pada gambar \ref{fig:usecase}.

	\begin{figure} [H]
		\centering  
		\includegraphics[width=14cm, height=14cm]{UseCase2} 
		\caption[Use Case ECA]{Use Case ECA} 
		\label{fig:usecase} 
	\end{figure}
	
Berdasarkan hasil analisis, dibentuk 3 \textit{use case} dengan 1 aktor, yaitu :
\begin{enumerate}
	\item \textbf{Memasukkan parameter simulasi}
	
	\textit{User} dapat memasukkan parameter seperti bobot setiap ketetanggaan, relasi ketetanggaan, bobot faktor publik, mengisi nilai a,b,c dan \textit{threshold} serta periode.
	\item \textbf{Memasukkan file data wirausaha dalam format text}
	
	\textit{User} dapat memasukkan data wirausaha yang akan disimulasikan berupa \textit{file} text.
	\item \textbf{Menjalankan simulasi}
	
	\textit{User} dapat menjalankan simulasi dan melihat hasil simulasi setiap bulannya.
\end{enumerate}

\textbf{Skenario \textit{Use Case}}
\begin{enumerate}
	\item Memasukkan parameter simulasi
	
		\begin{itemize}
			\item Nama : Memasukkan Parameter Simulasi
			\item Aktor : \textit{User}
			\item Deskripsi : Memasukkan bobot untuk setiap atribut dan parameter penting dalam simulasi.
			\item Kondisi awal : \textit{User} belum mengisi bobot untuk setiap atribut dan parameter dalam simulasi.
			\item Kondisi akhir : \textit{User} telah mengisi bobot untuk setiap atribut dan parameter dalam simulasi.
			\item Skenario utama :
		\end{itemize}
		
\begin{table}[H]
\centering
\caption{Tabel Skenario Memasukkan Parameter Simulasi}
\begin{tabular}{|c|p{7cm}|p{7cm}|}
\hline
No & Aksi & Reaksi Sistem\\
\hline
1 & \textit{User} memasukkan parameter simulasi & Sistem akan menyimpan masukan parameter dari \textit{user}.\\
\hline
\end{tabular}
\label{tabelSkenario1}
\end{table}

	\item Memasukkan \textit{File} Data Wirausaha Dalam Format Text
	\begin{itemize}
		\item Nama : Memasukkan \textit{file} data wirausaha dalam format text.
		\item Aktor : \textit{User}.
		\item Deskripsi : Memasukkan \textit{file} data wirausaha yang akan disimulasikan.
		\item Kondisi awal : \textit{User} memasukkan \textit{file} data wirausaha dalam format text.
		\item Kondisi akhir : Sistem akan menampilkan isi data pada tabel.
		\item Skenario utama:
	\end{itemize}
	
	\begin{table}[H]
\centering
\caption{Tabel Skenario Memasukkan \textit{file} data wirausaha dalam format text}
\begin{tabular}{|c|p{7cm}|p{7cm}|}
\hline
No & Aksi & Reaksi Sistem\\
\hline
1 & \textit{User} memilih \textit{file} dan memasukkan \textit{file} data wirausaha dalam format text. & Sistem akan menampilkan isi data pada tabel. \\
\hline
\end{tabular}
\label{tabelSkenario2}
\end{table}

	\item Menjalankan Simulasi
		\begin{itemize}
			\item Nama : Menjalankan Simulasi
			\item Aktor : \textit{User}
			\item Deskripsi : Menjalankan simulasi dan melihat hasil simulasi
			\item Kondisi awal : \textit{User} menjalankan program
			\item Kondisi akhir : Sistem akan menampilkan hasil di tabel dan sistem juga akan mengeluarkan hasil rincian perubahan individu wirausaha pada \textit{file} CSV.
			\item Skenario utama:
		\end{itemize}
		
\begin{table}[H]
\centering
\caption{Tabel Skenario Menjalankan Simulasi}
\begin{tabular}{|c|p{7cm}|p{7cm}|}
\hline
No & Aksi & Reaksi Sistem\\
\hline
1 & \textit{User} menjalankan program & Sistem akan menampilkan hasil di tabel dan sistem juga akan mengeluarkan hasil rincian perubahan individu wirausaha pada \textit{file} CSV \\
\hline
\end{tabular}
\label{tabelSkenario3}
\end{table}
		
\end{enumerate}


\subsection{Diagram Kelas}


Pada bagian ini akan diberikan diagram kelas ECA beserta penjelasannya.

	\begin{figure} [H]
		\centering  
		\includegraphics[width=18cm, height=12cm]{diagramKelas0} 
		\caption[Diagram Kelas ECA]{Diagram Kelas ECA} 
		\label{fig:CD1} 
	\end{figure}
	

\subsection{Kelas EGM}
	Kelas EGM merupakan kelas untuk menjalankan perhitungan CIDx, CIDx merupakan angka yang mengindikasikan kemungkinan seorang wirausahawan untuk meneruskan usahanya. Perhitungan CIDx ini menggunakan data dari GEM 2013.
	
\subsection{Kelas CA} 
Kelas CA merupakan kelas yang merepresentasikan \textit{cellular automata}. Berikut akan dijelaskan beberapa \textit{method} yang ada di kelas CA :
		\begin{enumerate}
			\item \texttt{public Entrepreneurs[] stateTransition(CA model, double[] composition)}\\
			Merupakan method untuk menentukan perubahan transisi pada seorang wirausaha yang bergantung pada umur dan nilai ambang.\\			Parameter:
			\begin{itemize}
				\item \texttt{model} merupakan objek dari kelas CA.
				\item \texttt{composition} merupakan nilai a,b dan c.
			\end{itemize}
			
			\item \texttt{public double getNeighborIndex(CA model, int idxEnt)}\\
			Merupakan method untuk menghitung nilai dari kondisi ketetanggaan setiap wirausaha.\\
			Parameter:
			\begin{itemize}
				\item \texttt{model} merupakan objek dari kelas CA.
				\item \texttt{idxEnt} merupakan indeks dari wirausaha.
			\end{itemize}
		
			\item \texttt{public void nextLevel(Entrepreneurs ne, int i, CA model, double[] composition)}\\
			Merupakan method untuk menentukan perubahan level usaha dari seorang wirausaha.\\
			Parameter:
			\begin{itemize}
				\item \texttt{ne} merupakan objek dari kelas Entrepreneurs.
				\item \texttt{i} merupakan indeks.
				\item \texttt{model} merupakan objek dari kelas CA.
				\item \texttt{composition} merupakan nilai dari a,b dan c.
			\end{itemize}
			
			\item \texttt{public double getIndex(int i, CA model, double[] composition)}\\
			Merupakan method untuk menghitung CIDx.\\
			Parameter:
			\begin{itemize}
				\item \texttt{i} merupakan indeks.
				\item \texttt{model} merupakan objek dari kelas CA.
				\item \texttt{composition} merupakan nilai dari a,b dan c.
			\end{itemize}
			
			\item \texttt{public void NeighborhoodDefinition()}\\
			Merupakan method untuk mendefinisikan jenis-jenis ketetanggaan seperti lebih dari sama dengan, sama dengan dan lebih kecil sama dengan.\\
			
			\item \texttt{public void genDummyEntrepreneurs()}\\
			Merupakan method untuk membuat data \textit{dummy} wirausaha.
			
			\item \texttt{public void genSimulationData()}\\
			Merupakan method untuk membuat data wirausaha secara \textit{random}.
			
			\item \texttt{public void writeSimulationData(String namaFile)}\\
			Merupakan method untuk menampilkan hasil simulasi ke dalam suatu file.\\
			Parameter:
			\begin{itemize}
				\item \texttt{namaFile} merupakan file tempat hasil simulasi akan ditampilkan.
			\end{itemize}
			
			\item \texttt{public void readSimulationData(String fileName)}\\
			Merupakan method untuk membaca dan memasukkan data file yang akan yang akan disimulasi.\\
			Parameter:
				\begin{itemize}
				\item \texttt{fileName} merupakan file untuk menyimpan hasil simulasi.
			\end{itemize}
			
			\item \texttt{public void print(int iter,PrintWriter out)}\\
			Merupakan method untuk menampilkan jumlah dari masing-masing level wirausaha.\\
			Parameter:
			\begin{itemize}
				\item \texttt{iter} merupakan iterasi per bulan.
				\item \texttt{out} untuk menge-\textit{print} hasil.
			\end{itemize}
			
			\item \texttt{calculatePoint(double[] POAm, double[] POAf, double[] POEf, double[] POEm, double[] POLm, double[] POLf, double[] POIm, double[] POIf, double[] PCAf, double[] PCAm, double[] PCEm, double[] PCEf, double[] PCLm, double[] PCLf, double[] PCIm, double[] PCIf, double[] RMAm, double[] RMAf, double[] RMIm, double[] RMIf)}\\
			Merupakan method untuk menghitung kondisi internal dari seorang wirausaha.\\
			Parameter:
			\begin{itemize}
				\item \texttt{POAm} merupakan kumpulan nilai dari Perceived Opportunities berdasarkan umur (pria).
				\item \texttt{POAf} merupakan kumpulan nilai dari Perceived Opportunities berdasarkan umur (wanita).
				\item \texttt{POEm} merupakan kumpulan nilai dari Perceived Opportunities berdasarkan pendidikan (pria).
				\item \texttt{POEf} merupakan kumpulan nilai dari Perceived Opportunities berdasarkan pendidikan (wanita).
				\item \texttt{POLm} merupakan kumpulan nilai dari Perceived Opportunities berdasarkan lokasi (pria).
				\item \texttt{POLf} merupakan kumpulan nilai dari Perceived Opportunities berdasarkan lokasi (wanita).
				\item \texttt{POIm} merupakan kumpulan nilai dari Perceived Opportunities berdasarkan pendapatan (pria).
				\item \texttt{POIf} merupakan kumpulan nilai dari Perceived Opportunities berdasarkan pendapatan (wanita).
				\item \texttt{PCAm} merupakan kumpulan nilai dari Perceived Capabilities berdasarkan umur (pria).
				\item \texttt{PCAf} merupakan kumpulan nilai dari Perceived Capabilities berdasarkan umur (wanita).
				\item \texttt{PCEm} merupakan kumpulan nilai dari Perceived Capabilities berdasarkan pendidikan (pria).
				\item \texttt{PCEf} merupakan kumpulan nilai dari Perceived Capabilities berdasarkan pendidikan (wanita).
				\item \texttt{PCLm} merupakan kumpulan nilai dari Perceived Capabilities berdasarkan lokasi (pria).
				\item \texttt{PCLf} merupakan kumpulan nilai dari Perceived Capabilities berdasarkan lokasi (wanita).
				\item \texttt{PCIm} merupakan kumpulan nilai dari Perceived Capabilities berdasarkan pendapatan (pria).
				\item \texttt{PCIf} merupakan kumpulan nilai dari Perceived Capabilities berdasarkan pendapatan (wanita).
				\item \texttt{RMAm} merupakan kumpulan nilai dari Role Model berdasarkan umur (pria).
				\item \texttt{RMAf} merupakan kumpulan nilai dari Role Model berdasarkan umur (wanita).
				\item \texttt{RMIm} merupakan kumpulan nilai dari Role Model berdasarkan pendapatan (pria).
				\item \texttt{RMIf} merupakan kumpulan nilai dari Role Model berdasarkan pendapatan (wanita).
			\end{itemize}
			
			\item \texttt{public int getAgeRange(int a)}\\
			Merupakan method untuk membedakan rentang usia yang telah ditentukan oleh GEM 2013.\cite{GEM2013}\\
			Parameter:
			\begin{itemize}
				\item \texttt{a} merupakan umur wirausaha.
			\end{itemize}
		\end{enumerate}
		
\subsection{Kelas Entrepreneur} 
	Kelas Entrepreneur merupakan kelas untuk merepresentasikan individu wirausahawan.
\subsection{Kelas Neighbor}
	Kelas Neighbor merupakan kelas untuk merepresentasikan ketetanggaan untuk satu aspek tertentu. Setiap aspeknya didefinisikan sebagai satu neighbor yang berupa adjacency matrix.
\subsection{Kelas Neighborhood}
	Kelas Neighborhood merupakan kelas untuk merepresentasikan himpunan ketetanggaan yang tersusun atas sejumlah ketetanggaan.
\subsection{Kelas PublicFactor}
	Kelas PublicFactor merupakan kelas untuk merepresentasikan faktok publik.
\subsection{Kelas State}
	Kelas State merupakan kelas untuk memberi nilai untuk setiap level wirausaha.


