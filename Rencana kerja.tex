\documentclass[a4paper,twoside]{article}
\usepackage[T1]{fontenc}
\usepackage[bahasa]{babel}
\usepackage{graphicx}
\usepackage{graphics}
\usepackage{float}
\usepackage[cm]{fullpage}
\pagestyle{myheadings}
\usepackage{etoolbox}
\usepackage{setspace} 
\usepackage{lipsum} 
\setlength{\headsep}{30pt}
\usepackage[inner=2cm,outer=2.5cm,top=2.5cm,bottom=2cm]{geometry} %margin
% \pagestyle{empty}

\makeatletter
\renewcommand{\@maketitle} {\begin{center} {\LARGE \textbf{ \textsc{\@title}} \par} \bigskip {\large \textbf{\textsc{\@author}} }\end{center} }
\renewcommand{\thispagestyle}[1]{}
\markright{\textbf{\textsc{AIF401/AIF402 \textemdash Rencana Kerja Skripsi \textemdash Sem. Genap 2016/2017}}}

\onehalfspacing
 
\begin{document}

\title{\@judultopik}
\author{\nama \textendash \@npm} 

%tulis nama dan NPM anda di sini:
\newcommand{\nama}{Vanessa Sukamto}
\newcommand{\@npm}{2014730010}
\newcommand{\@judultopik}{Simulator Pertumbuhan Wirausaha Berbasis Cellular Automata} % Judul/topik anda
\newcommand{\jumpemb}{1} % Jumlah pembimbing, 1 atau 2
\newcommand{\tanggal}{01/01/1900}

% Dokumen hasil template ini harus dicetak bolak-balik !!!!

\maketitle

\pagenumbering{arabic}

\section{Deskripsi}
Pada saat ini, lapangan kerja pada suatu negara tidak bisa kita prediksi, tetapi kenyataan yang kita ketahui adalah lapangan kerja dari tahun ke tahun semakin terbatas. Dengan melihat situasi tersebut maka bisa dipastikan tingkat pengangguran di suatu negara akan semakin tinggi. Solusi terbaik untuk mengurangi permasalahan tersebut adalah dengan berwirausaha. Kewirausahaan adalah kemampuan seseorang untuk membuat suatu usaha yang dimulai dari 0 atau dimulai dari bawah yang dirintis hingga usaha tersebut benar-benar sukses. Tentu saja hal ini memberikan pengaruh positif terhadap pertumbuhan ekonomi suatu negara, karena kewirausahaan juga sekaligus membuka lapangan kerja bagi masyarakat lainnya.

 
Pada jaman sekarang, sudah banyak sekali orang yang lebih memilih untuk berwirausaha daripada bekerja di kantor atau di sebuah perusahaan.  Alasan mengapa banyak orang lebih memilih berwirausaha pun bervariasi contohnya orang tersebut tidak terlalu menyukai waktu kerjanya diatur oleh orang lain melainkan ia lebih menyukai waktu kerjanya diatur oleh dirinya sendiri. Tidak hanya pada jaman sekarang, dari jaman dahulu juga sudah ada wirausaha yang namanya tidak asing lagi didengar oleh telinga kita salah satunya yaitu Bob Sadino. Untuk menjadi sukses seperti Bob Sadino dibutuhkan proses untuk meraih usaha tersebut yang tentunya mengalami jatuh-bangun, sehingga keuletan dan semangat juang yang tinggi sangat diperlukan dalam berwirausaha.Akan tetapi, usaha yang sudah sukses juga harus terus dijaga kekonsistenannya. Jika tidak dijaga, usaha tersebut akan mengalami kebangkrutan. Selain itu, kebangkrutan juga bisa disebabkan oleh banyak faktor, baik yang berasal dari dalam maupun luar.


Kondisi wirausaha perlu dipantau terus-menerus perkembangannya. Pemantauan ini dilakukan oleh lembaga-lembaga swasta yang berkepentingan. Salah satu lembaga yang memantau adalah GEM (Global Entrepreneurship Monitor). GEM merupakan konsorsium yang bertujuan untuk mengukur dan memantau kegiatan kewirausahaan. Selain pemantauan terhadap kondisi riil, salah satu kegiatan yang mendukung pemantauan adalah pengamatan secara tidak langsung. Melalui \textit{Cellular Automata} (CA) ini dapat dikembangkan sebuah model untuk mempelajari perilaku dari wirausaha dan berbagai faktor yang mempengaruhi pertumbuhan kewirausahaan. Selain menggunakan \textit{Cellular Automata} , digunakan juga parameter yang digunakan untuk membangun model yaitu indikator fase kewirausahaan dari GEM. Demikian juga untuk data uji akan digunakan data yang disediakan oleh GEM. 


\section{Rumusan Masalah}
\begin{itemize}
	\item Faktor apa saja yang mempengaruhi keberlangsungan wirausaha?
	\item Bagaimana memodelkan pertumbuhan wirausaha dengan \textit{cellular automata}?
\end{itemize}

\section{Tujuan}
\begin{itemize}
	\item Mempelajari faktor yang berpengaruh pada keberlangsungan wirausaha.
	\item Mengembangkan model keberlangsungan wirausaha dengan \textit{cellular automata}.
\end{itemize}

\section{Deskripsi Perangkat Lunak}
Perangkat lunak akhir yang akan dibuat memiliki fitur minimal sebagai berikut:
\begin{itemize}
	\item Pengguna dapat memasukkan masukkan (\textit{input}) berupa data dari GEM.
	\item Pengguna dapat melihat hasil keluaran (\textit{output}) yang dihasilkan setelah memasukkan (\textit{input}) masukkan.
  \item PL dapat menerima data dari GEM.
	\item PL dapat menampilkan hasil keluaran (\textit{output}) berupa grafik pertumbuhan wirausaha.
\end{itemize}

\section{Detail Pengerjaan Skripsi}
Bagian-bagian pekerjaan skripsi ini adalah sebagai berikut :
	\begin{enumerate}
		\item Mempelajari kewirausahaan secara umum.
		\item Mempelajari GEM.
		\item Membaca laporan akhir ECA.
		\item Mempelajari teori tentang\textit{Cellular Automata}.
		\item Mempelajari aplikasi \textit{Cellular Automata}.
		\item Mengembangkan model pertumbuhan wirausaha dengan\textit{Cellular Automata}.
		\item Mengembangkan perangkat lunak yang mengimplementasikan model \textit{Cellular Automata} yang telah didefinisikan.
		\item Menguji perangkat lunak menggunakan data dari GEM.
		\item Mengambil kesimpulan tentang model yang telah dikembangkan.
		\item Menulis dokumen skripsi.
	\end{enumerate}

\section{Rencana Kerja}
Rincian capaian yang direncanakan di Skripsi 1 adalah sebagai berikut:
\begin{enumerate}
	\item
\end{enumerate}
Sedangkan yang akan diselesaikan di Skripsi 2 adalah sebagai berikut:
\begin{enumerate}
	\item
\end{enumerate}


\vspace{1cm}
\centering Bandung, \tanggal\\
\vspace{2cm} \nama \\ 
\vspace{1cm}

Menyetujui, \\
\ifdefstring{\jumpemb}{2}{
\vspace{1.5cm}
\begin{centering} Menyetujui,\\ \end{centering} \vspace{0.75cm}
\begin{minipage}[b]{0.45\linewidth}
% \centering Bandung, \makebox[0.5cm]{\hrulefill}/\makebox[0.5cm]{\hrulefill}/2013 \\
\vspace{2cm} Nama: \makebox[3cm]{\hrulefill}\\ Pembimbing Utama
\end{minipage} \hspace{0.5cm}
\begin{minipage}[b]{0.45\linewidth}
% \centering Bandung, \makebox[0.5cm]{\hrulefill}/\makebox[0.5cm]{\hrulefill}/2013\\
\vspace{2cm} Nama: \makebox[3cm]{\hrulefill}\\ Pembimbing Pendamping
\end{minipage}
\vspace{0.5cm}
}{
% \centering Bandung, \makebox[0.5cm]{\hrulefill}/\makebox[0.5cm]{\hrulefill}/2013\\
\vspace{2cm} Nama: \makebox[3cm]{\hrulefill}\\ Pembimbing Tunggal
}
\end{document}

